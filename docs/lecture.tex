\documentclass[]{book}
\usepackage{lmodern}
\usepackage{amssymb,amsmath}
\usepackage{ifxetex,ifluatex}
\usepackage{fixltx2e} % provides \textsubscript
\ifnum 0\ifxetex 1\fi\ifluatex 1\fi=0 % if pdftex
  \usepackage[T1]{fontenc}
  \usepackage[utf8]{inputenc}
\else % if luatex or xelatex
  \ifxetex
    \usepackage{mathspec}
  \else
    \usepackage{fontspec}
  \fi
  \defaultfontfeatures{Ligatures=TeX,Scale=MatchLowercase}
\fi
% use upquote if available, for straight quotes in verbatim environments
\IfFileExists{upquote.sty}{\usepackage{upquote}}{}
% use microtype if available
\IfFileExists{microtype.sty}{%
\usepackage{microtype}
\UseMicrotypeSet[protrusion]{basicmath} % disable protrusion for tt fonts
}{}
\usepackage[margin=1in]{geometry}
\usepackage{hyperref}
\hypersetup{unicode=true,
            pdftitle={ECON 6120I Topics in Empirical Industrial Organization},
            pdfauthor={Kohei Kawaguchi},
            pdfborder={0 0 0},
            breaklinks=true}
\urlstyle{same}  % don't use monospace font for urls
\usepackage{natbib}
\bibliographystyle{plainnat}
\usepackage{color}
\usepackage{fancyvrb}
\newcommand{\VerbBar}{|}
\newcommand{\VERB}{\Verb[commandchars=\\\{\}]}
\DefineVerbatimEnvironment{Highlighting}{Verbatim}{commandchars=\\\{\}}
% Add ',fontsize=\small' for more characters per line
\usepackage{framed}
\definecolor{shadecolor}{RGB}{248,248,248}
\newenvironment{Shaded}{\begin{snugshade}}{\end{snugshade}}
\newcommand{\KeywordTok}[1]{\textcolor[rgb]{0.13,0.29,0.53}{\textbf{#1}}}
\newcommand{\DataTypeTok}[1]{\textcolor[rgb]{0.13,0.29,0.53}{#1}}
\newcommand{\DecValTok}[1]{\textcolor[rgb]{0.00,0.00,0.81}{#1}}
\newcommand{\BaseNTok}[1]{\textcolor[rgb]{0.00,0.00,0.81}{#1}}
\newcommand{\FloatTok}[1]{\textcolor[rgb]{0.00,0.00,0.81}{#1}}
\newcommand{\ConstantTok}[1]{\textcolor[rgb]{0.00,0.00,0.00}{#1}}
\newcommand{\CharTok}[1]{\textcolor[rgb]{0.31,0.60,0.02}{#1}}
\newcommand{\SpecialCharTok}[1]{\textcolor[rgb]{0.00,0.00,0.00}{#1}}
\newcommand{\StringTok}[1]{\textcolor[rgb]{0.31,0.60,0.02}{#1}}
\newcommand{\VerbatimStringTok}[1]{\textcolor[rgb]{0.31,0.60,0.02}{#1}}
\newcommand{\SpecialStringTok}[1]{\textcolor[rgb]{0.31,0.60,0.02}{#1}}
\newcommand{\ImportTok}[1]{#1}
\newcommand{\CommentTok}[1]{\textcolor[rgb]{0.56,0.35,0.01}{\textit{#1}}}
\newcommand{\DocumentationTok}[1]{\textcolor[rgb]{0.56,0.35,0.01}{\textbf{\textit{#1}}}}
\newcommand{\AnnotationTok}[1]{\textcolor[rgb]{0.56,0.35,0.01}{\textbf{\textit{#1}}}}
\newcommand{\CommentVarTok}[1]{\textcolor[rgb]{0.56,0.35,0.01}{\textbf{\textit{#1}}}}
\newcommand{\OtherTok}[1]{\textcolor[rgb]{0.56,0.35,0.01}{#1}}
\newcommand{\FunctionTok}[1]{\textcolor[rgb]{0.00,0.00,0.00}{#1}}
\newcommand{\VariableTok}[1]{\textcolor[rgb]{0.00,0.00,0.00}{#1}}
\newcommand{\ControlFlowTok}[1]{\textcolor[rgb]{0.13,0.29,0.53}{\textbf{#1}}}
\newcommand{\OperatorTok}[1]{\textcolor[rgb]{0.81,0.36,0.00}{\textbf{#1}}}
\newcommand{\BuiltInTok}[1]{#1}
\newcommand{\ExtensionTok}[1]{#1}
\newcommand{\PreprocessorTok}[1]{\textcolor[rgb]{0.56,0.35,0.01}{\textit{#1}}}
\newcommand{\AttributeTok}[1]{\textcolor[rgb]{0.77,0.63,0.00}{#1}}
\newcommand{\RegionMarkerTok}[1]{#1}
\newcommand{\InformationTok}[1]{\textcolor[rgb]{0.56,0.35,0.01}{\textbf{\textit{#1}}}}
\newcommand{\WarningTok}[1]{\textcolor[rgb]{0.56,0.35,0.01}{\textbf{\textit{#1}}}}
\newcommand{\AlertTok}[1]{\textcolor[rgb]{0.94,0.16,0.16}{#1}}
\newcommand{\ErrorTok}[1]{\textcolor[rgb]{0.64,0.00,0.00}{\textbf{#1}}}
\newcommand{\NormalTok}[1]{#1}
\usepackage{longtable,booktabs}
\usepackage{graphicx,grffile}
\makeatletter
\def\maxwidth{\ifdim\Gin@nat@width>\linewidth\linewidth\else\Gin@nat@width\fi}
\def\maxheight{\ifdim\Gin@nat@height>\textheight\textheight\else\Gin@nat@height\fi}
\makeatother
% Scale images if necessary, so that they will not overflow the page
% margins by default, and it is still possible to overwrite the defaults
% using explicit options in \includegraphics[width, height, ...]{}
\setkeys{Gin}{width=\maxwidth,height=\maxheight,keepaspectratio}
\IfFileExists{parskip.sty}{%
\usepackage{parskip}
}{% else
\setlength{\parindent}{0pt}
\setlength{\parskip}{6pt plus 2pt minus 1pt}
}
\setlength{\emergencystretch}{3em}  % prevent overfull lines
\providecommand{\tightlist}{%
  \setlength{\itemsep}{0pt}\setlength{\parskip}{0pt}}
\setcounter{secnumdepth}{5}
% Redefines (sub)paragraphs to behave more like sections
\ifx\paragraph\undefined\else
\let\oldparagraph\paragraph
\renewcommand{\paragraph}[1]{\oldparagraph{#1}\mbox{}}
\fi
\ifx\subparagraph\undefined\else
\let\oldsubparagraph\subparagraph
\renewcommand{\subparagraph}[1]{\oldsubparagraph{#1}\mbox{}}
\fi

%%% Use protect on footnotes to avoid problems with footnotes in titles
\let\rmarkdownfootnote\footnote%
\def\footnote{\protect\rmarkdownfootnote}

%%% Change title format to be more compact
\usepackage{titling}

% Create subtitle command for use in maketitle
\newcommand{\subtitle}[1]{
  \posttitle{
    \begin{center}\large#1\end{center}
    }
}

\setlength{\droptitle}{-2em}

  \title{ECON 6120I Topics in Empirical Industrial Organization}
    \pretitle{\vspace{\droptitle}\centering\huge}
  \posttitle{\par}
    \author{Kohei Kawaguchi}
    \preauthor{\centering\large\emph}
  \postauthor{\par}
      \predate{\centering\large\emph}
  \postdate{\par}
    \date{Last updated: 2019-02-06}

\usepackage{booktabs}

\usepackage{amsthm}
\newtheorem{theorem}{Theorem}[chapter]
\newtheorem{lemma}{Lemma}[chapter]
\theoremstyle{definition}
\newtheorem{definition}{Definition}[chapter]
\newtheorem{corollary}{Corollary}[chapter]
\newtheorem{proposition}{Proposition}[chapter]
\theoremstyle{definition}
\newtheorem{example}{Example}[chapter]
\theoremstyle{definition}
\newtheorem{exercise}{Exercise}[chapter]
\theoremstyle{remark}
\newtheorem*{remark}{Remark}
\newtheorem*{solution}{Solution}
\begin{document}
\maketitle

{
\setcounter{tocdepth}{1}
\tableofcontents
}
\chapter{Syllabus}\label{syllabus}

\section{Instructor Information}\label{instructor-information}

\begin{itemize}
\tightlist
\item
  Instructor:

  \begin{itemize}
  \tightlist
  \item
    Name: Kohei Kawaguchi.
  \item
    Office: LSK6070, Monday 11:00-12:00.
  \end{itemize}
\item
  All questions related to the class have to be publicly asked on the
  discussion board of canvas rather than being privately asked in
  e-mail. The instructor usually does not reply in the evening,
  weekends, and holidays.
\end{itemize}

\section{General Information}\label{general-information}

\subsection{Class Time}\label{class-time}

\begin{itemize}
\tightlist
\item
  Date: Monday.
\item
  Time: 13:30-17:20.
\item
  Venue: CYTG001.
\end{itemize}

\subsection{Description}\label{description}

\begin{itemize}
\item
  This is a PhD-level course for empirical industrial organization. This
  course covers various econometric methods used in industrial
  organization that is often referred to as the structural estimation
  approach. These methods have been gradually developed since 1980s in
  parallel with the modernization of industrial organization based on
  the game theory and now widely applied in antitrust policy, business
  strategy, and neighboring fields such as labor economics and
  international economics.
\item
  This course presumes a good understanding of PhD-level microeconomics
  and microeconometrics. Participants are expected to understand at
  least UG-level industrial organization. This course requires
  participants to write programs mostly in R and sometimes in C++ to
  implement various econometric methods. In particular, all assignments
  will involve such a non-trivial programming task. Even though the
  understanding of these programming languages is not a prerequisite, a
  sharp learning curve will be required. Some experience in other
  programming languages will help. Audit without a credit is not
  admitted for students.
\end{itemize}

\subsection{Expectation and Goals}\label{expectation-and-goals}

\begin{itemize}
\tightlist
\item
  The goal of this course is to learn and practice econometric methods
  for empirical industrial organization. The lecture covers the
  econometric methods that have been developed between 80s and 00s to
  estimate primitive parameters governing imperfect competition among
  firms, such as production and cost function estimation, demand
  function estimation, merger simulation, entry and exit analysis, and
  dynamic decision models. The lecture also covers various new methods
  to recover model primitives in certain mechanisms such as auction,
  matching, network, and bargaining. The emphasis is put on the former
  group of methods, because they are the basis for other new methods.
  Participants will not only understand the theoretical background of
  the methods but also become able to implement these methods from
  scratches by writing their own programs. I will briefly discuss the
  latter class of new methods through reading recent papers. The
  participants will become able to understand and use these new methods.
\end{itemize}

\section{Required Environment}\label{required-environment}

\begin{itemize}
\tightlist
\item
  Participants should bring their laptop to the class. We have enough
  extension codes for students. The laptop should have sufficient RAM
  (at least \(\ge\) 8GB, \(\ge\) 16GB is recommended) and CPU power (at
  least Core i5 grade, Core i7 grade is recommended). Participants are
  fully responsible for their hardware issues. Operating System can be
  arbitrary. The instructor mainly uses OSX High Siera with iMac (Retina
  5K, 27-inch, Late 2015) and Macbook Pro (Retina, 15-inch, Early 2017).
\item
  Please install the following software by the first lecture. Technical
  issues related to the installment should be resolved by yourself,
  because it depends on your local environment. If you had an error,
  copy and paste the error message on a search engine, and find a
  solution. This solves 99.9\% of the problems.

  \begin{itemize}
  \tightlist
  \item
    R: \url{https://www.r-project.org/}
  \item
    RStudio: \url{https://www.rstudio.com/}
  \item
    LaTeX:

    \begin{itemize}
    \tightlist
    \item
      MixTex \url{https://miktex.org/}
    \item
      TeXLive \url{https://www.tug.org/texlive/}
    \item
      MacTeX \url{http://www.tug.org/mactex/}
    \end{itemize}
  \end{itemize}
\end{itemize}

\section{Evaluation}\label{evaluation}

\begin{itemize}
\tightlist
\item
  Assignments (80): In total 8 homework are assigned. Each homework
  involves the implementation of the methods covered in the class. Each
  homework has 10 points. The working hour for each homework will be
  around 10-20 hours.
\item
  Participation (10): Every time a participant asks a question in the
  class, after the class, during the office hour, or in the canvas. the
  participant gets one point, up to 10 points. The participant who asked
  the question writes the name, ID number, his/her question, and my
  answer in a discussion board on the course website to claim a point.
\item
  Referee report (10): Toward the end of the semester, a paper in
  industrial organization is randomly assigned to each participant. Each
  participant writes a critical referee report of the assigned paper in
  A4 2 pages that consists of the summary, contribution, strong and weak
  points of the paper.
\item
  Grading is based on the absolute scores: A+ with more than 80 points,
  A with more than 70 points, A- with more than 60 points, B+ with more
  than 50 points, B with more than 40 points, B- with more than 30
  points and C otherwise.
\end{itemize}

\section{Academic Integrity}\label{academic-integrity}

Without academic integrity, there is no serious learning. Thus you are
expected to hold the highest standard of academic integrity in the
course. You are encouraged to study and do homework in groups. However,
no cheating, plagiarism will be tolerated. Anyone caught cheating,
plagiarism will fail the course. Please make sure adhere to the HKUST
Academic Honor Code at all time (see
\url{http://www.ust.hk/vpaao/integrity/}).

\section{Schedule}\label{schedule}

\begin{itemize}
\tightlist
\item
  Introduction to structural estimation, R and RStudio
\item
  Production function estimation I
\item
  Production function estimation II
\item
  Demand function estimation I
\item
  Demand function estimation II
\item
  Merger Analysis
\item
  Entry and Exit
\item
  Single-Agent Dynamics I
\item
  Single-Agent Dynamics II, I change date due to my business trip
\item
  Dynamic Game I
\item
  Dynamic Game II
\item
  Canceled because of Easter Monday
\item
  Auction
\item
  Other Mechanisms
\end{itemize}

\section{Course Materials}\label{course-materials}

\subsection{R and RStudio}\label{r-and-rstudio}

\begin{itemize}
\tightlist
\item
  Grolemund, G., 2014, Hands-On Programming with R, O'Reilly.

  \begin{itemize}
  \tightlist
  \item
    Free online version is available:
    \url{https://rstudio-education.github.io/hopr/}.
  \end{itemize}
\item
  Wickham, H., \& Grolemund, G., 2017, R for Data Science, O'Reilly.

  \begin{itemize}
  \tightlist
  \item
    Free online version is available: \url{https://r4ds.had.co.nz/}.
  \end{itemize}
\item
  Boswell, D., \& Foucher, T., 2011, The Art of Readable Code: Simple
  and Practical Techniques for Writing Better Code, O'Reilly.
\end{itemize}

\subsection{Handbook Chapters}\label{handbook-chapters}

\begin{itemize}
\tightlist
\item
  Ackerberg, D., Benkard, C., Berry, S., \& Pakes, A. (2007).
  ``Econometric tools for analyzing market outcomes''. Handbook of
  econometrics, 6, 4171-4276.
\item
  Athey, S., \& Haile, P. A. (2007). ``Nonparametric approaches to
  auctions''. Handbook of Econometrics, 6, 3847-3965.
\item
  Berry, S., \& Reiss, P. (2007). ``Empirical models of entry and market
  structure''. Handbook of Industrial Organization, 3, 1845-1886.
\item
  Bresnahan, T. F. (1989). ``Empirical studies of industries with market
  power''. Handbook of Industrial Organization, 2, 1011-1057.
\item
  Hendricks, K., \& Porter, R. H. (2007). ``An empirical perspective on
  auctions''. Handbook of Industrial Organization, 3, 2073-2143.
\item
  Matzkin, R. L. (2007). ``Nonparametric identification''. Handbook of
  Econometrics, 6, 5307-5368.
\item
  Newey, W. K., \& McFadden, D. (1994). ``Large sample estimation and
  hypothesis testing''. Handbook of Econometrics, 4, 2111-2245.
\item
  Reiss, P. C., \& Wolak, F. A. (2007). ``Structural econometric
  modeling: Rationales and examples from industrial organization''.
  Handbook of Econometrics, 6, 4277-4415.
\end{itemize}

\subsection{Books}\label{books}

\begin{itemize}
\tightlist
\item
  Train, K. E. (2009). Discrete Choice Methods with Simulation,
  Cambridge university press.
\item
  Davis, P., \& Garces, E. (2010). Quantitative Techniques for
  Competition and Antitrust Analysis, Princeton University Press.
\item
  Tirole, J. (1988). The Theory of Industrial Organization, The MIT
  Press.
\end{itemize}

\subsection{Papers}\label{papers}

\begin{itemize}
\tightlist
\item
  The list of important papers are occasionally given during the course.
\end{itemize}

\chapter{Introduction}\label{intro}

\section{Structural Estimation and Counterfactual
Analysis}\label{structural-estimation-and-counterfactual-analysis}

\subsection{Example}\label{example}

\begin{itemize}
\item
  \citet{Igami2017} ``Estimating the Innovator's Dilemma: Structural
  Analysis of Creative Destruction in the Hard Disk Drive Industry,
  1981-1998''.
\item
  \textbf{Question}:
\item
  Does ``Innovator's Dilemma'' \citep{Christensen1997} exist?
\item
  Christensen argued that old winners tend to lag behind entrants even
  when introducing a new technology is not too difficult, with a case
  study of the HDD industry.

  \begin{itemize}
  \tightlist
  \item
    Apple's smartphone vs.~Nokia's feature phones.
  \item
    Amazon vs.~Borders.
  \item
    Kodak's digital camera.
  \end{itemize}
\item
  How do we empirically answer this question?
\end{itemize}

\begin{figure}

{\centering \includegraphics[width=0.8\linewidth]{figuretable/Igam2017Fig1} 

}

\caption{Figure 1 of Igam (2017)}\label{fig:unnamed-chunk-2}
\end{figure}

\begin{itemize}
\item
  \textbf{Hypotheses}:
\item
  Identify potentially competing hypotheses to explain the phenomenon.

  \begin{enumerate}
  \def\labelenumi{\arabic{enumi}.}
  \tightlist
  \item
    Cannibalization: Because of cannibalization, the benefits of
    introducing a new product are smaller for incumbents than for
    entrants.
  \item
    Different costs: The incumbents may have higher costs for innovation
    due to the organizational inertia, but at the same time they may
    have some cost advantage due to accumulated R\&D and better
    financial access.
  \item
    Preemption: The incumbents have additional incentive for innovation
    to preempt potential rivals.
  \item
    Institutional environment: The impacts of the three components
    differ across different institutional contexts such as the rules
    governing patents and market size.
  \end{enumerate}
\item
  Casual empiricists pick up their favorite factors to make up a story.
\item
  Serious empiricists should try to separate the contributions of each
  factor from data.
\item
  To do so, the author develops an economic model that explicitly
  incorporates the above mentioned factors, while keeping the model
  parameters flexible enough to let the data tell the sign and size of
  the effects of each factor on innovation.
\item
  \textbf{Economic model}:
\item
  The time is discrete with finite horizon \(t = 1, \cdots, T\).
\item
  In each year, there is a finite number of firms indexed by \(i\).
\item
  Each firm is in one of the technological states:

  \begin{equation}
  s_{it} \in \{\text{old only, both, new only, potential entrant}\},
  \end{equation}

  where the first two states are for incumbents (stick to the old
  technology or start using the new technology) and the last two states
  are for actual and potential entrants (enter with the new technology
  or stay outside the market).
\item
  In each year:

  \begin{itemize}
  \tightlist
  \item
    Pre-innovation incumbent (\(s_{it} =\) old): exit or innovate by
    paying a sunk cost \(\kappa^{inc}\) (to be \(s_{i, t + 1} =\) both).
  \item
    Post-innovation incumbent (\(s_{it} =\) both): exit or stay to be
    both.
  \item
    Potential entrant (\(s_{it} =\) potential entrant): give up entry or
    enter with the new technology by paying a sunk cost \(\kappa^{net}\)
    (to be \(s_{i, t + 1} =\) new).
  \item
    Actual entrant (\(s_{it} =\) new): exit or stay to be new.
  \end{itemize}
\item
  Given the industry state \(s_t = \{s_{it}\}_i\), the product market
  competition opens and the profit of firm \(i\),
  \(\pi_t(s_{it}, s_{-it})\), is realized for each active firm.
\item
  As the product market competition closes:

  \begin{itemize}
  \tightlist
  \item
    Pre-innovation incumbents draw private cost shocks and make
    decisions: \(a_t^{pre}\).
  \item
    Observing this, post-innovation incumbents draw private cost shocks
    and make decisions: \(a_t^{post}\).
  \item
    Observing this, actual entrants draw private cost shocks and make
    decisions: \(a_t^{act}\).
  \item
    Observing this, potential entrants draw private cost shocks and make
    decisions: \(a_t^{pot}\).
  \end{itemize}
\item
  This is a dynamic game. The equilibrium is defined by the concept of
  \textbf{Markov-perfecet equilibrium} \citep{Maskin1988}.
\item
  The representation of the competing theories in the model:

  \begin{itemize}
  \tightlist
  \item
    The existence of cannibalization is represented by the assumption
    that an incumbent maximizes the joint profits of old and new
    technology products.
  \item
    The size of cannibalization is captured by the shape of profit
    function.
  \item
    The difference in the cost of innovation is captured by the
    difference in the sunk costs of innovation.
  \item
    The preemptive incentive for incumbents are embodied in the dynamic
    optimization problem for each incumbent.
  \end{itemize}
\item
  \textbf{Econometric model}:
\item
  The author then turns the economic model into an econometric model.
\item
  This amounts to specify which part of the economic model is
  observed/known and which part is unobserved/unknown.
\item
  The author collects the data set of the HDD industry during 1977-99.
\item
  Based on the data, the author specify the identities of active firms
  and their products and the technologies embodied in the products in
  each year to code their \textbf{state variables}.
\item
  Moreover, by tracking the change in the state, the author code their
  \textbf{action variables}.
\item
  Thus, the state and action variables, \(s_t\) and \(a_t\). These are
  the \textbf{observables}.
\item
  The author does not observe:

  \begin{itemize}
  \tightlist
  \item
    Profit function \(\pi_t(\cdot)\).
  \item
    Sunk cost of innovation for pre-innovation incumbents
    \(\kappa^{inc}\).
  \item
    Sunk cost of entry for potential entrants \(\kappa^{net}\).
  \item
    Private cost shocks.
  \end{itemize}
\item
  These are the \textbf{unobservables}.
\item
  Among the unobservables, the profit function and sunk costs are the
  \textbf{parameter of interets} and the private cost shocks are
  \textbf{nuissance parameters} in the sense only the knowledge about
  the distribution of the latter is demanded.
\item
  \textbf{Identification}:
\item
  Can we infer the unobservables from the observables and the
  restrictions on the distribution of observable by the economic theory?
\item
  The profit function is identified from estimating the demand function
  for each firm's product, and estimating the cost function for each
  firm from using their price setting behavior.
\item
  The sunk costs of innovation are identified from the conditional
  probability of innovation across various states. If the cost is low,
  the probability should be high.
\item
  \textbf{Estimation}:
\item
  The identification established that in principle we can uncover the
  parameters of interests from observables under the restrictions of
  economic theory.
\item
  Finally, we apply a statistical method to the econometric model and
  infer the parameters of interest.
\item
  \textbf{Counterfactual analysis}:
\item
  If we can uncover the parameters of interest, we can conduct
  \textbf{comparative statics}: study the change in the endogenous
  variables when the exogenous variables including the model parameters
  are set different. In the current framework, this exercise is often
  called the \textbf{counterfactual analysis}.
\item
  What if there was no cannibalization?:

  \begin{itemize}
  \tightlist
  \item
    An incumbents separately maximizes the profit from old technology
    and new technology instead of jointly maximizing the profits. Solve
    the model under this new assumption everything else being equal.
  \item
    Free of cannibalization concerns, 8.95 incumbents start producing
    new HDDs in the first 10 years, compared with 6.30 in the baseline.
  \item
    The cumulative numbers of innovators among incumbents and entrants
    differ only by 2.8 compared with 6.45 in the baseline.
  \item
    Thus cannibalization can explain a significant part of the
    incumbent-entrant innovation gap.
  \end{itemize}
\item
  What if there was no preemption?:

  \begin{itemize}
  \tightlist
  \item
    A potential entrant ignores the incumbents' innovations upon making
    entry decisions.
  \item
    Without the preemptive motives, only 6.02 incumbents would innovate
    in the first 10 7ears, compared with 6.30 in the baseline.
  \item
    The cumulative incumbent-entrant innovation gap widen to 8.91
    compared with 6.45 in the baseline.
  \end{itemize}
\item
  The sunk cost of entry is smaller for incumbents than for entrants in
  the baseline.
\item
  \textbf{Interpretations and policy/managerial implication}:
\item
  Despite the cost advantage and the preemptive motives, the speed of
  innovation is slower among incumbents due to the strong
  cannibalization effect.
\item
  Incumbents that attempt to avoid the ``innovator's dilemma'' should
  separate the decision makings between old and new sections inside the
  organization so that it can avoid the concern for cannibalization.
\end{itemize}

\subsection{Recap}\label{recap}

\begin{itemize}
\tightlist
\item
  The structural approach in empirical industrial organization consists
  of the following components:
\end{itemize}

\begin{enumerate}
\def\labelenumi{\arabic{enumi}.}
\tightlist
\item
  Research question.
\item
  Competing hypotheses.
\item
  Economic model.
\item
  Econometric model
\item
  Identification.
\item
  Data collection.
\item
  Data cleaning.
\item
  Estimation.
\item
  Counterfactual analysis.
\item
  Coding.
\item
  Interpretations and policy/managerial implications.
\end{enumerate}

\begin{itemize}
\tightlist
\item
  The goal of this course is to be familiar with the standard
  methodology to complete this process.
\item
  The methodology covered in this class is mostly developed to analyze
  the standard framework to dynamic or oligopoly competition.
\item
  The policy implications are centered around competition policies.
\item
  But the basic idea can be extend to different class of situations such
  as auction, matching, voting, contract, marketing, and so on.
\item
  Note that the depth of the research question and the relevance of the
  policy/managerial implications are the most important part of the
  research.
\item
  Focusing on the methodology in this class is to minimize the time to
  allocate to less important issues and maximize the attention and time
  to the most valuable part in the future research.
\item
  Given a research question, what kind of data is necessary to answer
  the question?
\item
  Given data, what kind of research questions can you address? Which
  question can be credibly answered? Which question can be an
  over-stretch?
\item
  Given a research question and data, what is the best way to answer the
  question? What type of problem can you avoid using the method? What is
  the limitation of your approach? How will you defend the possible
  referee comments?
\item
  Given a result, what kinds of interpretation can you credibly derive?
  What kinds of interpretation can be contested by potential opponents?
  What kinds of contribution can you claim?
\item
  To address these issues is \textbf{necessary} to publish a paper and
  it is \textbf{necessary} to be familiar with the methodology to do so.
\end{itemize}

\subsection{Historical Remark}\label{historical-remark}

\begin{itemize}
\tightlist
\item
  The words \textbf{reduced-form} and \textbf{structural-form} date back
  to the literature of estimation of simultaneous equations in
  macroeconomics \citep{Hsiao1983}.
\item
  Let \(y\) be the vector of observed endogenous variables, \(x\) be the
  vector of observed exogenous variables, and \(\epsilon\) be the vector
  of unobserved exogenous variables.
\item
  The equilibrium condition for \(y\) on \(x\) and \(\epsilon\) is often
  written as:

  \begin{equation}
  Ay + Bx = \epsilon. \label{eq:structuralform}
  \end{equation}
\item
  These equations \textbf{implicitly} determine the vector of endogenous
  variables \(y\) .
\item
  If \(A\) is invertible, we can solve the equations for \(y\) to
  obtain:

  \begin{equation}
  y = - A^{-1} B x + A^{-1} \epsilon. \label{eq:reducedform}
  \end{equation}
\item
  These equations \textbf{explicitly} determine the vector of endogenous
  variables \(y\).
\item
  Equation \eqref{eq:structuralform} is the \textbf{structural-form} and
  \eqref{eq:reducedform} is the \textbf{reduced-form}.
\item
  If \(y\) and \(x\) are observed and \(x\) is of full column rank, then
  \(A^{-1}B\) and \(A^{-1}\) will be estimated by regression for
  \eqref{eq:reducedform}. But this does not mean that \(A\) and \(B\) are
  separately estimated.
\item
  This was the traditional identification problems.
\item
  Thus, reduced-form does not mean either of:

  \begin{itemize}
  \tightlist
  \item
    Regression analysis;
  \item
    Statistical analysis free from economic assumptions.
  \end{itemize}
\item
  Recent development in this line of literature of identification is
  found in \citet{Matzkin2007}.
\item
  In econometrics, the idea of imposing restrictions from economic
  theories seems to have been formalized by the work of
  \citet{Manski1994a} and \citet{Matzkin1994b}.
\end{itemize}

\section{Setting Up The Environment}\label{setting-up-the-environment}

\begin{itemize}
\tightlist
\item
  Assume that R, RStudio and LaTex are all installed in the local
  computer.
\end{itemize}

\subsection{RStudio Project}\label{rstudio-project}

\begin{itemize}
\tightlist
\item
  The assignments should be conducted inside a project folder for this
  course.
\item
  \texttt{File\ \textgreater{}\ New\ Project...\textgreater{}\ New\ Directory\ \textgreater{}\ New\ Directory\ \textgreater{}\ R\ Package\ using\ RcppEigen}.
\item
  Name the directory \texttt{ECON6120I} and place in your favorite
  location.
\item
  You can open this project from the upper right menu of RStudio or by
  double clicking the \texttt{ECON6120I.Rproj} file in the
  \texttt{ECON6120I} directory.
\item
  This navigates you to the root directory of the project.
\item
  In the root directory, make folders named:

  \begin{itemize}
  \tightlist
  \item
    \texttt{assignment}.
  \item
    \texttt{input}.
  \item
    \texttt{output}.
  \item
    \texttt{figuretable}.
  \end{itemize}
\item
  We will store R functions in \texttt{R} folder, C/C++ functions in
  \texttt{src} folder, and data in \texttt{input} folder, data generated
  from the code in \texttt{output}, and figures and tables in
  \texttt{figurtable} folder.
\item
  Open \texttt{src/Makevars} and erase the content. Then, write:
  \texttt{PKG\_CPPFLAGS\ =\ -w\ -std=c++11\ -O3}
\item
  Open \texttt{src/Makevars.win} and erase the content. Then, write:
  \texttt{PKG\_CPPFLAGS\ =\ -w\ -std=c++11}
\end{itemize}

\subsection{Basic Programming in R}\label{basic-programming-in-r}

\begin{itemize}
\tightlist
\item
  \texttt{File\ \textgreater{}\ New\ File\ \textgreater{}\ R\ Script} to
  open \texttt{Untitled} file.
\item
  \texttt{Ctrl\ (Cmd)\ +\ S} to save it with \texttt{test.R} in
  \texttt{assignment} folder.
\item
  In the console, type and push enter:
\end{itemize}

\begin{Shaded}
\begin{Highlighting}[]
\DecValTok{1} \OperatorTok{+}\StringTok{ }\DecValTok{1}
\end{Highlighting}
\end{Shaded}

\begin{verbatim}
## [1] 2
\end{verbatim}

\begin{Shaded}
\begin{Highlighting}[]
\DecValTok{100}\OperatorTok{:}\DecValTok{130}
\end{Highlighting}
\end{Shaded}

\begin{verbatim}
##  [1] 100 101 102 103 104 105 106 107 108 109 110 111 112 113 114 115 116
## [18] 117 118 119 120 121 122 123 124 125 126 127 128 129 130
\end{verbatim}

\begin{itemize}
\tightlist
\item
  This is the interactive way of using R functionalities.
\item
  In \texttt{test.R}, write:
\end{itemize}

\begin{Shaded}
\begin{Highlighting}[]
\DecValTok{1} \OperatorTok{+}\StringTok{ }\DecValTok{1}
\end{Highlighting}
\end{Shaded}

\begin{itemize}
\item
  Then, save the file and push \texttt{Run}.
\item
  Alternatively, place the mouse over the \texttt{1\ +\ 1} line in
  \texttt{test.R} file.
\item
  Then, \texttt{Ctrl\ (Cmd)\ +\ Enter} to run the line.
\item
  In this way, we can write procedures in the file and send to the
  console to run.
\item
  There are functions to conduct basic calculations:
\end{itemize}

\begin{Shaded}
\begin{Highlighting}[]
\DecValTok{1} \OperatorTok{+}\StringTok{ }\DecValTok{2}
\end{Highlighting}
\end{Shaded}

\begin{verbatim}
## [1] 3
\end{verbatim}

\begin{Shaded}
\begin{Highlighting}[]
\DecValTok{2} \OperatorTok{*}\StringTok{ }\DecValTok{3}
\end{Highlighting}
\end{Shaded}

\begin{verbatim}
## [1] 6
\end{verbatim}

\begin{Shaded}
\begin{Highlighting}[]
\DecValTok{4} \OperatorTok{-}\StringTok{ }\DecValTok{1}
\end{Highlighting}
\end{Shaded}

\begin{verbatim}
## [1] 3
\end{verbatim}

\begin{Shaded}
\begin{Highlighting}[]
\DecValTok{6} \OperatorTok{/}\StringTok{ }\DecValTok{2}
\end{Highlighting}
\end{Shaded}

\begin{verbatim}
## [1] 3
\end{verbatim}

\begin{Shaded}
\begin{Highlighting}[]
\DecValTok{2}\OperatorTok{^}\DecValTok{3}
\end{Highlighting}
\end{Shaded}

\begin{verbatim}
## [1] 8
\end{verbatim}

\begin{itemize}
\tightlist
\item
  We can define objects and assign values to them.
\end{itemize}

\begin{Shaded}
\begin{Highlighting}[]
\NormalTok{a <-}\StringTok{ }\DecValTok{1}
\NormalTok{a}
\end{Highlighting}
\end{Shaded}

\begin{verbatim}
## [1] 1
\end{verbatim}

\begin{Shaded}
\begin{Highlighting}[]
\NormalTok{a }\OperatorTok{+}\StringTok{ }\DecValTok{2}
\end{Highlighting}
\end{Shaded}

\begin{verbatim}
## [1] 3
\end{verbatim}

\begin{itemize}
\tightlist
\item
  In addition to scalar object, we can define a vector by:
\end{itemize}

\begin{Shaded}
\begin{Highlighting}[]
\DecValTok{2}\OperatorTok{:}\DecValTok{10}
\end{Highlighting}
\end{Shaded}

\begin{verbatim}
## [1]  2  3  4  5  6  7  8  9 10
\end{verbatim}

\begin{Shaded}
\begin{Highlighting}[]
\DecValTok{3}\OperatorTok{:}\DecValTok{20}
\end{Highlighting}
\end{Shaded}

\begin{verbatim}
##  [1]  3  4  5  6  7  8  9 10 11 12 13 14 15 16 17 18 19 20
\end{verbatim}

\begin{Shaded}
\begin{Highlighting}[]
\KeywordTok{c}\NormalTok{(}\DecValTok{2}\NormalTok{, }\DecValTok{3}\NormalTok{, }\DecValTok{5}\NormalTok{, }\DecValTok{9}\NormalTok{, }\DecValTok{10}\NormalTok{)}
\end{Highlighting}
\end{Shaded}

\begin{verbatim}
## [1]  2  3  5  9 10
\end{verbatim}

\begin{Shaded}
\begin{Highlighting}[]
\KeywordTok{seq}\NormalTok{(}\DecValTok{1}\NormalTok{, }\DecValTok{10}\NormalTok{, }\DecValTok{2}\NormalTok{)}
\end{Highlighting}
\end{Shaded}

\begin{verbatim}
## [1] 1 3 5 7 9
\end{verbatim}

\begin{itemize}
\tightlist
\item
  \texttt{seq} is a function with initial value, end values, and the
  increment value.
\item
  By typing \texttt{seq} in the \texttt{help}, we can read the manual
  page of the function.
\item
  \texttt{seq\ \{base\}} means that this function is named \texttt{seq}
  and is contained in the library called \texttt{base}.
\item
  Some libraries are automatically called when the R is launched, but
  some are not.
\item
  Some libraries are even not installed.
\item
  We can install a library from a repository called \texttt{CRAN}.
\end{itemize}

\begin{Shaded}
\begin{Highlighting}[]
\KeywordTok{install.packages}\NormalTok{(}\StringTok{"ggplot2"}\NormalTok{)}
\end{Highlighting}
\end{Shaded}

\begin{itemize}
\tightlist
\item
  To use the package, we have to load by:
\end{itemize}

\begin{Shaded}
\begin{Highlighting}[]
\KeywordTok{library}\NormalTok{(ggplot2)}
\end{Highlighting}
\end{Shaded}

\begin{itemize}
\tightlist
\item
  Use \texttt{qplot} function in \texttt{ggplot2} library to draw a
  scatter plot.
\end{itemize}

\begin{Shaded}
\begin{Highlighting}[]
\NormalTok{x <-}\StringTok{ }\KeywordTok{c}\NormalTok{(}\OperatorTok{-}\DecValTok{1}\NormalTok{, }\OperatorTok{-}\FloatTok{0.8}\NormalTok{, }\OperatorTok{-}\FloatTok{0.6}\NormalTok{, }\OperatorTok{-}\FloatTok{0.4}\NormalTok{, }\OperatorTok{-}\FloatTok{0.2}\NormalTok{, }\DecValTok{0}\NormalTok{, }\FloatTok{0.2}\NormalTok{, }\FloatTok{0.4}\NormalTok{, }\FloatTok{0.6}\NormalTok{, }\FloatTok{0.7}\NormalTok{, }\DecValTok{1}\NormalTok{)}
\NormalTok{y <-}\StringTok{ }\NormalTok{x}\OperatorTok{^}\DecValTok{3}
\KeywordTok{qplot}\NormalTok{(x, y)}
\end{Highlighting}
\end{Shaded}

\includegraphics{lecture_files/figure-latex/unnamed-chunk-15-1.pdf}

\begin{itemize}
\tightlist
\item
  We can write own functions.
\end{itemize}

\begin{Shaded}
\begin{Highlighting}[]
\NormalTok{roll <-}\StringTok{ }\ControlFlowTok{function}\NormalTok{(n) \{}
\NormalTok{  die <-}\StringTok{ }\DecValTok{1}\OperatorTok{:}\DecValTok{6}
\NormalTok{  dice <-}\StringTok{ }\KeywordTok{sample}\NormalTok{(die, }\DataTypeTok{size =}\NormalTok{ n, }\DataTypeTok{replace =} \OtherTok{TRUE}\NormalTok{)}
\NormalTok{  y <-}\StringTok{ }\KeywordTok{sum}\NormalTok{(dice)}
  \KeywordTok{return}\NormalTok{(y)}
\NormalTok{\}}
\KeywordTok{roll}\NormalTok{(}\DecValTok{1}\NormalTok{)}
\end{Highlighting}
\end{Shaded}

\begin{verbatim}
## [1] 2
\end{verbatim}

\begin{Shaded}
\begin{Highlighting}[]
\KeywordTok{roll}\NormalTok{(}\DecValTok{2}\NormalTok{)}
\end{Highlighting}
\end{Shaded}

\begin{verbatim}
## [1] 7
\end{verbatim}

\begin{Shaded}
\begin{Highlighting}[]
\KeywordTok{roll}\NormalTok{(}\DecValTok{10}\NormalTok{)}
\end{Highlighting}
\end{Shaded}

\begin{verbatim}
## [1] 42
\end{verbatim}

\begin{Shaded}
\begin{Highlighting}[]
\KeywordTok{roll}\NormalTok{(}\DecValTok{10}\NormalTok{)}
\end{Highlighting}
\end{Shaded}

\begin{verbatim}
## [1] 29
\end{verbatim}

\begin{itemize}
\tightlist
\item
  We can \texttt{set.seed} to obtain the same realization of random
  variables.
\end{itemize}

\begin{Shaded}
\begin{Highlighting}[]
\KeywordTok{set.seed}\NormalTok{(}\DecValTok{1}\NormalTok{)}
\KeywordTok{roll}\NormalTok{(}\DecValTok{10}\NormalTok{)}
\end{Highlighting}
\end{Shaded}

\begin{verbatim}
## [1] 38
\end{verbatim}

\begin{Shaded}
\begin{Highlighting}[]
\KeywordTok{set.seed}\NormalTok{(}\DecValTok{1}\NormalTok{)}
\KeywordTok{roll}\NormalTok{(}\DecValTok{10}\NormalTok{)}
\end{Highlighting}
\end{Shaded}

\begin{verbatim}
## [1] 38
\end{verbatim}

\begin{itemize}
\tightlist
\item
  You can write the functions in the files with executing codes.
\item
  But I recommend you to separate files for writing functions and
  executing codes.
\item
  \texttt{File\ \textgreater{}\ New\ File\ \textgreater{}\ R\ Rcript}
  and name it as \texttt{functions.R} and save to \texttt{R} folder.
\item
  Cut the function you wrote and paste it in \texttt{functions.R}.
\item
  There are two ways of calling a function in \texttt{functions.R} from
  \texttt{test.R}.
\item
  One way is to use \texttt{source} function.
\end{itemize}

\begin{Shaded}
\begin{Highlighting}[]
\KeywordTok{source}\NormalTok{(}\StringTok{"R/functions.R"}\NormalTok{)}
\end{Highlighting}
\end{Shaded}

\begin{itemize}
\tightlist
\item
  When this line is read, the codes in the file are executed.
\item
  The other way is to bundle functions as a package and load it.
\item
  Choose \texttt{Build\ \textgreater{}\ Clean\ and\ Rebuild}.
\item
  This compiles files in \texttt{src} folder and bundle functions in
  \texttt{R} folder and build a package named \texttt{ECON6120I}.
\item
  Now, the functions in \texttt{R} folder and \texttt{src} folder can be
  used by loading the package by:
\end{itemize}

\begin{Shaded}
\begin{Highlighting}[]
\KeywordTok{library}\NormalTok{(ECON6120I)}
\end{Highlighting}
\end{Shaded}

\begin{itemize}
\tightlist
\item
  Best practice:

  \begin{enumerate}
  \def\labelenumi{\arabic{enumi}.}
  \tightlist
  \item
    Write functions in the scratch file.
  \item
    As the functions are tested, move them to \texttt{R/functions.R}.
  \item
    Clean and rebuild and load them as a package.
  \end{enumerate}
\end{itemize}

\subsection{Reproducible Reports using
Rmarkdown}\label{reproducible-reports-using-rmarkdown}

\begin{itemize}
\tightlist
\item
  Reporting in empirical studies involves:
\end{itemize}

\begin{enumerate}
\def\labelenumi{\arabic{enumi}.}
\tightlist
\item
  Writing texts;
\item
  Writing formulas;
\item
  Writing and implementing programs;
\item
  Demonstrating the results with figures and tables.
\end{enumerate}

\begin{itemize}
\item
  Moreover, this has to be done in a \textbf{reproducible} manner:
  Whoever can reproduce the output from the scratch.
\item
  ``Whoever'' includes yourself in the future. Because the revision
  process of structural papers is usually lengthy, you often have to
  remember the content few weeks or few months later. It is inefficient
  if you cannot recall what you have done.
\item
  We use \texttt{Rmarkdown} to achieve this goal.
\item
  This assumes that you have LaTex installed.
\item
  Install package \texttt{Rmarkdown}:
\end{itemize}

\begin{Shaded}
\begin{Highlighting}[]
\KeywordTok{install.packages}\NormalTok{(}\StringTok{"rmarkdown"}\NormalTok{)}
\end{Highlighting}
\end{Shaded}

\begin{itemize}
\tightlist
\item
  \texttt{File\ \textgreater{}\ New\ File\ \textgreater{}\ R\ Markdown...\ \textgreater{}\ HTML}
  with title \texttt{Test}.
\item
  Save it in \texttt{assignment} folder with name \texttt{test.Rmd}.
\item
  From \texttt{Knit} tab, choose \texttt{Knit\ to\ HTML}.
\item
  This outputs the content to html file.
\item
  You can also choose \texttt{Knit\ to\ PDF} from \texttt{Knit} tab to
  obtain output in pdf file.
\item
  Reports should be knit to pdf to submit.
\item
  But you can use html output while writing a report because html is
  lighter to compile.
\item
  Refer to the \href{https://rmarkdown.rstudio.com/lesson-1.html}{help
  page} for further information.
\end{itemize}

\chapter{Production and Cost Function Estimation}\label{production}

\section{Motivations}\label{motivations}

\begin{itemize}
\tightlist
\item
  Estimating \textbf{production and cost functions} of producers is the
  cornerstone of economic analysis.
\item
  Estimating the functions includes to separate the contribution of
  observed inputs and the other factors, which is often referred to as
  the \textbf{productivity}.
\item
  ``What determines productivity?'' \citep{Syverson2011}-type research
  questions naturally follow.
\item
  The methods covered in this chapter are widely used across different
  fields.
\item
  Some of them are variants from the standard methods.
\end{itemize}

\subsection{IO}\label{io}

\begin{itemize}
\tightlist
\item
  \citet{Olley1996}:

  \begin{itemize}
  \tightlist
  \item
    How much did the deregulation in the U.S. telecommunication
    industry, in particular the divestiture of AT\&T in 1984, spurred
    the productivity growth of the incumbent, facilitated entries, and
    increased the aggregate productivity?
  \item
    To do so, the authors estimate the plant-level production functions
    and productivity in the telecommunication industry.
  \end{itemize}
\item
  \citet{Doraszelski2013a}:

  \begin{itemize}
  \tightlist
  \item
    What is the role of R\&D in determining the differences in
    productivity across firms and the evolution of firm-level
    productivity over time?
  \item
    To do so, the authors estimate the firm-level production functions
    and productivity of Spanish manufacturing firms during 1990s in
    which the transition probability of a productivity is a function of
    the R\&D activities.
  \end{itemize}
\end{itemize}

\subsection{Development}\label{development}

\begin{itemize}
\tightlist
\item
  \citet{Hsieh2009}:

  \begin{itemize}
  \tightlist
  \item
    How large is the misallocation of inputs across manufacturing firms
    in China and India compared to the U.S? How will the aggregate
    productivity of China and India change if the degree of
    misallocation is reduced to the U.S. level?
  \item
    To do so, the authors measure the revenue productivity of firms,
    which should be the same across firms within an industry if there
    were no distortion, and the measurement of the revenue productivity
    requires to estimate the production function.
  \end{itemize}
\item
  \citet{Gennaioli2013}:

  \begin{itemize}
  \tightlist
  \item
    What are the determinants of regional growth? Do geographic,
    institutional, cultural, and human capital factors explain the
    difference across regions?
  \item
    To do so, the authors construct the data set that covers 74\% of the
    world's surface and 97\% of its GDP and estimate the production
    function in which the above mentioned factors could affect the
    productivity.
  \end{itemize}
\end{itemize}

\subsection{Trade}\label{trade}

\begin{itemize}
\tightlist
\item
  \citet{Haskel2007}:

  \begin{itemize}
  \tightlist
  \item
    Are there spillovers from FDI to domestic firms?
  \item
    To do so, the authors estimate the plant-level production function
    of the U.K. manufacturing firms during 1973 and 1992 and study how
    the foreign presence in the U.K. affected the productivity.
  \end{itemize}
\item
  \citet{Loecker2011}:

  \begin{itemize}
  \tightlist
  \item
    Does the removal of trade barriers induces efficiency gain for
    producers?
  \item
    To do so, the author estimate the production functions of Belgian
    textile industry during 1994-2002 in which the degree of trade
    protection can affect the productivity level.
  \end{itemize}
\end{itemize}

\subsection{Management}\label{management}

\begin{itemize}
\tightlist
\item
  \citet{Bloom2007}:

  \begin{itemize}
  \tightlist
  \item
    How do management practices affect the firm productivity?
  \item
    To do so, the authors first estimate the production function and
    productivity of manufacturing firms in developed countries, and then
    study how the independently measured management practices of the
    firms affect the estimated productivity.
  \end{itemize}
\item
  \citet{Braguinsky2015}:

  \begin{itemize}
  \tightlist
  \item
    How do changes in ownership affect the productivity and
    profitability of firms?
  \item
    To do so, the authors estimate the production function for various
    outputs including the physical output, return on capital and labor,
    and the utilization rate, price level, using the cotton spinners
    data in Japan during 1896 and 1920.
  \end{itemize}
\end{itemize}

\subsection{Education}\label{education}

\begin{itemize}
\tightlist
\item
  \citet{Cunha2010}:

  \begin{itemize}
  \tightlist
  \item
    How do childhood and schooling interventions ``produce'' the
    cognitive and non-cognitive skills of children?
  \item
    To do so, the authors estimate the mapping from childhood and
    schooling interventions to children's cognitive and non-cognitive
    skills, the ``production function'' of childhood environment and
    education.
  \end{itemize}
\end{itemize}

\section{Analyzing Producer
Behaviors}\label{analyzing-producer-behaviors}

\begin{itemize}
\item
  There are several levels of parameters that govern the behavior of
  firms:
\item
  \textbf{Production function}

  \begin{itemize}
  \tightlist
  \item
    Add factor market structure.
  \item
    Add cost minimization.
  \end{itemize}
\item
  \(\rightarrow\) \textbf{Cost function}

  \begin{itemize}
  \tightlist
  \item
    Add product market structure.
  \item
    Add profit maximization.
  \end{itemize}
\item
  \(\rightarrow\) \textbf{Supply function (Pricing function)}

  \begin{itemize}
  \tightlist
  \item
    Combine cost and supply (pricing) functions.
  \end{itemize}
\item
  \(\rightarrow\) \textbf{Profit function}
\item
  Which parameter to identify?
\item
  Primitive enough to be invariant to relevant policy changes.

  \begin{itemize}
  \tightlist
  \item
    e.g.~If you conduct a policy experiment that changes the factor
    market structure, identifying cost functions is not enough.
  \end{itemize}
\item
  As reduced-form as possible among such specifications.

  \begin{itemize}
  \tightlist
  \item
    A reduced-form parameter usually can be rationalized by a class of
    underlying structural parameters and institutional assumptions.
    Thus, the analysis becomes robust to some misspecifications.
  \item
    e.g.~A non-parametric function \(C(q, w)\) can represent a cost
    function of a producer who is not necessarily minimizing the cost.
    If we derive a cost function from a production function and a factor
    market structure, then the cost function cannot represent such a
    non-optimization behavior.
  \end{itemize}
\end{itemize}

\section{Production Function
Estimation}\label{production-function-estimation}

\subsection{Cobb-Douglas Specification as a
Benchmark}\label{cobb-douglas-specification-as-a-benchmark}

\begin{itemize}
\tightlist
\item
  Most of the following argument carries over to a general model.
\item
  For firm \(j = 1, \cdots, J\) and time \(t = 1, \cdots, T\), we
  observe output \(Y_{jt}\), labor \(L_{jt}\), and capital \(K_{jt}\).
\item
  We consider an asymptotic of \(J \to \infty\) for a fixed \(T\).
\item
  Assume Cobb-Douglas production function:

  \begin{equation}
  Y_{jt} = A_{jt}  L_{jt}^{\beta_l} K_{jt}^{\beta_k},
  \end{equation}

  where \(A_{jt}\) is firm \(j\) and time \(t\) specific unobserved
  heterogeneity in the model.
\item
  Taking the logarithm gives:

  \begin{equation}
  y_{jt} = \beta_0 + \beta_l l_{jt} + \beta_k k_{jt} + \epsilon_{jt},
  \end{equation}

  where lowercase symbols represent natural logs of variables and
  \(\ln(A_{jt}) = \beta_0 + \epsilon_{jt}\).
\item
  This can be regarded as a first-order log-linear approximation of a
  production function.
\item
  Linear regression model! May OLS work?
\end{itemize}

\subsection{Potential Bias I:
Endogeneity}\label{potential-bias-i-endogeneity}

\begin{itemize}
\tightlist
\item
  \(\epsilon_{jt}\) contains everything that cannot be explained by the
  observed inputs: better capital may be employed, a worker may have
  obtained better skills, etc.
\item
  When the manager of a firm makes an input choice, she should have some
  information about the realization of \(\epsilon_{jt}\).
\item
  Thus, the input choice can be correlated with \(\epsilon_{jt}\); for
  example under static optimization of \(L_{jt}\) given \(K_{jt}\):

  \begin{equation}
  L_{jt} = \Bigg[\frac{p_{jt}}{w_{jt}} \beta_l \exp^{\beta_0 + \epsilon_{jt}} K_{jt}^{\beta_k}\Bigg]^{\frac{1}{1 - \beta_l}}.
  \end{equation}
\item
  In this case, OLS estimator for \(\beta_l\) is \textit{positively}
  biased, because when \(\epsilon_{jt}\) is high, \(l_{jt}\) is high and
  thus the increase in output caused by \(\epsilon_{jt}\) is captured as
  if caused by the increase in labor input.
\item
  The endogeneity problem was already recognized by
  \citet{Marschak1944}.
\end{itemize}

\subsection{Potential Bias II:
Selection}\label{potential-bias-ii-selection}

\begin{itemize}
\tightlist
\item
  Firms freely enter and exit market.
\item
  Therefore, a firm that had low \(\epsilon_{jt}\) is likely to exit.
\item
  However, if firms have high capital \(K_{jt}\), it can stay in the
  market even if the realization of \(\epsilon_{jt}\) is very low.
\item
  Therefore, conditional on being in the market, there is a
  \textit{negative} correlation between the capital \(K_{jt}\) and
  \(\epsilon_{jt}\).
\item
  This problem occurs even if the choice of \(K_{jt}\) itself is not a
  function of \(\epsilon_{jt}\).
\end{itemize}

\subsection{How to Resolve Endogeneity
Bias?}\label{how-to-resolve-endogeneity-bias}

\begin{itemize}
\tightlist
\item
  Temporarily abstract away from entry and exit.
\item
  The data is balanced.
\end{itemize}

\begin{enumerate}
\def\labelenumi{\arabic{enumi}.}
\tightlist
\item
  Panel data.
\item
  First-order condition for inputs.
\item
  Instrumental variable.
\item
  Olley-Pakes approach and its followers/critics.
\end{enumerate}

\begin{itemize}
\tightlist
\item
  \citet{Griliches1998} is a good survey of the history up to
  Olley-Pakes approach.
\item
  \citet{Ackerberg2015} also offer a good survey and clarify problems
  and implicit assumptions in Olley-Pakes approach.
\end{itemize}

\subsection{Panel Data}\label{panel-data}

\begin{itemize}
\tightlist
\item
  Assume that \(\epsilon_{jt} = \mu_j + \eta_{jt}\), where \(\eta_{jt}\)
  is uncorrelated with input choices up to period \(t\):

  \begin{equation}
  y_{jt} = \beta_0 + \beta_l l_{jt} + \beta_k k_{jt} + \mu_j + \eta_{jt}.
  \end{equation}
\item
  Then, by differentiating period \(t\) and \(t - 1\) equations, we get:

  \begin{equation}
  y_{jt} - y_{j, t - 1}= \beta_l (l_{jt} - l_{j, t - 1}) + \beta_k (k_{jt} - k_{j, t - 1}) + (\eta_{jt} - \eta_{j, t - 1}).
  \end{equation}
\item
  Then, because \(\eta_{jt} - \eta_{j, t - 1}\) is uncorrelated either
  with \(l_{jt} - l_{j, t - 1}\) or \(k_{jt} - k_{j, t - 1}\), we can
  identify the parameter.
\item
  Problem:

  \begin{itemize}
  \tightlist
  \item
    Restrictive heterogeneity.
  \item
    When there are measurement errors, fixed-effect estimator can
    generate higher biases than OLS estimator, because measurement
    errors more likely to survive first-difference and
    within-transformation.
  \end{itemize}
\end{itemize}

\subsection{First-Order Condition for
Inputs}\label{first-order-condition-for-inputs}

\begin{itemize}
\tightlist
\item
  Use the first-order condition for inputs as the moment condition
  \citep{McElroy1987}.
\item
  Closely related to the cost function estimation literature.
\item
  Need to specify the factor market structure and the nature of the
  optimization problem for a firm.
\item
  Recently being center of attention again as one of the solutions to
  the ``collinearity problem'' discussed below.
\end{itemize}

\subsection{Instrumental Variable}\label{productioniv}

\begin{itemize}
\item
  Borrow the idea from the first-order condition approach that the input
  choices are affected by some exogenous variables.
\item
  If we have instrumental variables that affect inputs but are
  uncorrelated with errors \(\epsilon_{jt}\), then we can identify the
  parameter by an instrumental variable method.
\item
  One candidate for the instrumental variables: \textbf{input prices}.
\item
  Input price affect input decision.
\item
  Input price is not correlated with \(\epsilon_{jt}\) if the factor
  product market is competitive and \(\epsilon_{jt}\) is an
  idiosyncratic shock to a firm.
\item
  Problems:

  \begin{itemize}
  \tightlist
  \item
    Input prices often lack cross-sectional variation.
  \item
    Cross-sectional variation is often due to unobserved input quality.
  \end{itemize}
\item
  Another candidate for the instrumental variables: \textbf{lagged
  inputs}.
\item
  If \(\epsilon_{jt}\) does not have auto-correlation, lagged inputs are
  not correlated with the current shock.
\item
  If there are adjustment costs for inputs, then lagged inputs are
  correlated with the current inputs.
\item
  Problem:

  \begin{itemize}
  \tightlist
  \item
    If \(\epsilon_{jt}\) has auto-correlation, all lagged inputs are
    correlated with the errors: For example, if \(\epsilon_{jt}\) is
    AR(1),
    \(\epsilon_{jt} = \alpha \epsilon_{j, t - 1} + \nu_{j, t - 1} = \cdots \alpha^l \epsilon_{j, t - l} + \nu_{j, t - 1} + \cdots, \alpha^{l - 1} \nu_{j, t - l}\)
    for any \(l\).
  \end{itemize}
\end{itemize}

\subsection{Olley-Pakes Approach}\label{olley-pakes-approach}

\begin{itemize}
\tightlist
\item
  Exploit restrictions from the economic theory \citep{Olley1996}.
\item
  Write \(\epsilon_{jt} = \omega_{jt} + \eta_{jt}\), where
  \(\omega_{jt}\) is an anticipated shock and \(\eta_{jt}\) is an
  ex-post shock.
\item
  Inputs are correlated with \(\omega_{jt}\) but not with \(\eta_{jt}\)
\item
  The model is written as:

  \begin{equation}
  y_{jt} = \beta_0 + \beta_l l_{jt} + \beta_k k_{jt} + \omega_{jt} + \eta_{jt}.
  \end{equation}
\item
  OP use economic theory to derive a valid proxy for the anticipated
  shock \(\omega_{jt}\).
\end{itemize}

\subsection{Assumption I: Information
Set}\label{assumption-i-information-set}

\begin{itemize}
\tightlist
\item
  The firm's information set at \(t\), \(I_{jt}\), includes current and
  past productivity shocks \(\{\omega_{j\tau}\}_{\tau = 0}^t\) but does
  not include future productivity shocks
  \(\{\omega_{j\tau}\}_{\tau = t + 1}^{\infty}\).
\item
  The transitory shocks \(\eta_{jt}\) satisfy
  \(\mathbb{E}\{\eta_{jt}|I_{jt}\} = 0\).
\end{itemize}

\subsection{Assumption II: First Order
Markov}\label{assumption-ii-first-order-markov}

\begin{itemize}
\tightlist
\item
  Productivity shocks evolve according to the distribution:

  \begin{equation}
  p(\omega_{j, t + 1}|I_{jt}) = p(\omega_{j, t + 1}|\omega_{jt}), 
  \end{equation}

  and the distribution is known to firms and stochastically increasing
  in \(\omega_{jt}\).
\item
  Then:

  \begin{equation}
  \omega_{jt} = \mathbb{E}\{\omega_{jt}|\omega_{j, t - 1}\} + \nu_{jt},
  \end{equation}

  and:

  \begin{equation}
  \mathbb{E}\{\nu_{jt}|I_{j, t - 1}\} = 0,
  \end{equation}

  by construction.
\end{itemize}

\subsection{Assumption III: Timing of Input
Choices}\label{assumption-iii-timing-of-input-choices}

\begin{itemize}
\tightlist
\item
  Firms accumulate capital according to:

  \begin{equation}
  k_{jt} = \kappa(k_{j, t - 1}, i_{j, t - 1}),
  \end{equation}

  where investment \(i_{j, t - 1}\) is chosen in period \(t - 1\).
\item
  Labor input \(l_{jt}\) is non-dynamic and chosen at \(t\).
\item
  This assumption characterizes and distinguishes labor and capital.
\item
  Intuitively, it takes a full period for new capital to be ordered,
  delivered, and installed.
\end{itemize}

\subsection{Assumption IV: Scalar
Unobservable}\label{assumption-iv-scalar-unobservable}

\begin{itemize}
\tightlist
\item
  Firms' investment decisions are given by:

  \begin{equation}
  i_{jt} = f_t(k_{jt}, \omega_{jt}).
  \end{equation}
\item
  This assumption places strong implicit restrictions on additional
  firm-specific unobservables.

  \begin{itemize}
  \tightlist
  \item
    No \textbf{across firm} unobserved heterogeneity in adjustment cost
    of capital, in demand and labor market conditions, or in other parts
    of the production function.
  \item
    Okay with \textbf{across time} unobserved heterogeneity.
  \end{itemize}
\end{itemize}

\subsection{Assumption IV: Strict
Monotonicity}\label{assumption-iv-strict-monotonicity}

\begin{itemize}
\tightlist
\item
  The investment policy function \(f_t(k_{jt}, \omega_{jt})\) is
  strictly increasing in \(\omega_{jt}\).
\item
  This holds if the realization of higher \(\omega_{jt}\) implies higher
  expectation for future productivity (Assumption III) and if the
  marginal product of capital is increasing in the expectation for
  future productivity.
\item
  To verify the latter condition in a given game is often not easy.
\end{itemize}

\subsection{Two-step Approach: The First
Step}\label{two-step-approach-the-first-step}

\begin{itemize}
\tightlist
\item
  Insert \(\omega_{jt} = h(k_{jt}, i_{jt})\) to the original equation to
  get:

  \begin{equation}
  \begin{split}
  y_{jt} &= \beta_l l_{jt} + \underbrace{\beta_0 + \beta_k k_{jt} + h(k_{jt}, i_{jt})}_{\text{unknown function of $k_{jt}$ and $i_{jt}$}} + \eta_{jt}\\
  & \equiv \beta_l l_{jt} + \phi(k_{jt}, i_{jt}) + \eta_{jt}.
  \end{split}
  \end{equation}
\item
  This is a \textbf{partially linear model}: see \citet{Ichimura2007}
  for reference.
\item
  Because \(l_{jt}, k_{jt}\) and \(i_{jt}\) are uncorrelated with
  \(\eta_{jt}\), we can identify \(\beta_l\) and \(\phi(\cdot)\) by
  exploiting the moment condition:

  \begin{equation}
  \begin{split}
  & \mathbb{E}\{\eta_{jt}|l_{jt}, k_{jt}, i_{jt}\} = 0\\
  & \Leftrightarrow \mathbb{E}\{y_{jt} - \beta_l l_{jt} - \phi(k_{jt}, i_{jt}) |l_{jt}, k_{jt}, i_{jt}\} = 0.
  \end{split}
  \end{equation}

  \textbf{if there is enough variation} in \(l_{jt}, k_{jt}\) and
  \(i_{jt}\).
\item
  This ``if there is enough variation'' part is actually problematic.
  Discuss later.
\item
  Let \(\beta_l^0\) and \(\phi^0\) be the identified true parameters.
\end{itemize}

\subsection{Two-step Approach: The Second
Step}\label{two-step-approach-the-second-step}

\begin{itemize}
\tightlist
\item
  Note that:

  \begin{equation}
  \omega_{jt} \equiv \phi(k_{jt}, i_{jt}) - \beta_0 - \beta_k k_{jt}.
  \end{equation}
\item
  Therefore, we have:

  \begin{equation}
  \begin{split}
  &y_{jt} - \beta_l^0 l_{jt} \\
  &= \beta_0 + \beta_k k_{jt} + \omega_{jt} + \eta_{jt}\\
  &= \beta_0 + \beta_k k_{jt} + g(\omega_{j, t - 1}) + \nu_{jt} + \eta_{jt}\\
  &= \beta_0 + \beta_k k_{jt} + g[\phi^0(k_{j, t - 1}, i_{j, t - 1}) - (\beta_0 + \beta_k k_{j, t - 1})] + \nu_{jt} + \eta_{jt}.
  \end{split}
  \end{equation}
\item
  \(\nu_{jt}\) and \(\eta_{jt}\) are independent of the covariates.
\item
  This is a \textbf{multiple-index model} with indices
  \(\beta_0 + \beta_1 k_{jt}\) and \(\beta_0 + \beta_1 k_{j, t - 1}\)
  where parameters of two indices are restricted to be the same: see
  \citet{Ichimura2007} for reference.
\item
  We can identify \(\beta_0, \beta_k\) and \(g\) by exploiting the
  moment condition:

  \begin{equation}
  \begin{split}
  & \mathbb{E}\{\nu_{jt} + \eta_{jt}|k_{jt}, k_{j, t - 1}, i_{j, t - 1}\} = 0\\
  & \Leftrightarrow \mathbb{E}\{\nu_{jt} + \eta_{jt}|k_{jt}, k_{j, t - 1}, i_{j, t - 1}\} = 0.
  \end{split}
  \end{equation}
\end{itemize}

\subsection{Identification of the Anticipated
Shocks}\label{identification-of-the-anticipated-shocks}

\begin{itemize}
\tightlist
\item
  If \(\phi, \beta_0, \beta_k\) are identified, then \(\omega_{jt}\) is
  also identified by:

  \begin{equation}
  \omega_{jt} \equiv \phi(k_{jt}, i_{jt}) - \beta_0 - \beta_k k_{jt}.
  \end{equation}
\end{itemize}

\subsection{\texorpdfstring{Two-Step Estimation of
\citet{Olley1996}.}{Two-Step Estimation of @Olley1996.}}\label{two-step-estimation-of-olley1996.}

\begin{itemize}
\tightlist
\item
  \textbf{First step}: Estimate \(\beta_L\) and \(\phi\) in :

  \begin{equation}
  \begin{split}
  y_{jt} = \beta_l l_{jt} + \phi(k_{jt}, i_{jt}) + \eta_{jt}.
  \end{split}
  \end{equation}

  by approximating \(\phi\) with some basis functions, say, polynomials
  or splines:

  \begin{equation}
  \begin{split}
  y_{jt} &= \beta_l l_{jt} +  \sum_{p = 1}^P \gamma_p \phi_p(k_{jt}, i_{jt}) +  \left[\phi(k_{jt}, i_{jt}) - \sum_{n = 1}^N \gamma_n \phi_n(k_{jt}, i_{jt})\right] + \eta_{jt}\\
  & = \beta_l l_{jt} +  \sum_{p = 1}^P \gamma_p \phi_p(k_{jt}, i_{jt}) + \tilde{\eta}_{jt}
  \end{split}
  \end{equation}

  where \(P \to \infty\) when the sample size goes to infinity.
\item
  e.g.~second-order polynomial approximation:

  \begin{equation}
  \begin{split}
  & \phi_1(k_{jt}, i_{jt}) = k_{jt}, \phi_2(k_{jt}, i_{jt}) = i_{jt}\\
  & \phi_3(k_{jt}, i_{jt}) = k_{jt}^2, \phi_4(k_{jt}, i_{jt}) = i_{jt}^2\\
  & \phi_5(k_{jt}, i_{jt}) = k_{jt} i_{jt}.
  \end{split}
  \end{equation}
\item
  Once the basis functions are fixed, estimation is the same as the
  linear model.
\item
  But the inference (the computation of the standard deviation) is
  difference, because of the approximation error.
\item
  See \citet{Chen2007} for reference.
\item
  Let \(\hat{\beta}_l\) and \(\hat{\phi}\) be the estimates from the
  first step.
\item
  \textbf{Second step}: Estimate \(\beta_0\), \(\beta_k\), and \(g\) in:

  \begin{equation}
  \begin{split}
  y_{jt} - \hat{\beta}_l l_{jt}& = \beta_0 + \beta_k k_{jt} + g[\hat{\phi}(k_{j, t - 1}, i_{j, t - 1}) - (\beta_0 + \beta_k k_{j, t - 1})] + \nu_{jt} + \eta_{jt}\\
  &+ [\beta_l - \hat{\beta}_l] l_{jt}\\
  &+ \left\{g[\phi(k_{j, t - 1}, i_{j, t - 1}) - (\beta_0 + \beta_k k_{j, t - 1})] - g[\hat{\phi}(k_{j, t - 1}, i_{j, t - 1}) - (\beta_0 + \beta_k k_{j, t - 1})]\right\}\\
  & = \beta_0 + \beta_k k_{jt} + g[\hat{\phi}(k_{j, t - 1}, i_{j, t - 1}) - (\beta_0 + \beta_k k_{j, t - 1})] + \nu_{jt} + \tilde{\eta}_{jt}
  \end{split}
  \end{equation}

  by approximating \(g\) by some basis functions, say, polynomials or
  splines.
\end{itemize}

\subsection{From An Economic Models to An Econometric
Model}\label{from-an-economic-models-to-an-econometric-model}

\begin{itemize}
\tightlist
\item
  Starting from economic model with some unobserved heterogeneity, we
  reach some reduced-form model.
\item
  If the resulting model belongs to a class of econometric models whose
  identification and estimation are established, we can simply apply the
  existing methods.
\end{itemize}

\subsection{How to Resolve Selection
Bias}\label{how-to-resolve-selection-bias}

\begin{itemize}
\tightlist
\item
  Use propensity score to correct selection bias: \citet{Ahn1993}.
\item
  At the beginning of period \(t\), after observing \(\omega_{jt}\),
  firm \(j\) decides whether to continue the business
  (\(\chi_{jt} = 1\)) or exit (\(\chi_{jt} = 0)\).
\item
  Assume that the difference between continuation and exit values is
  strictly increasing in \(\omega_{jt}\).
\item
  Then, there is a threshold \(\underline{\omega}(k_{jt})\) such that:

  \begin{equation}
  \chi_{jt} = 
  \begin{cases}
  1 &\text{   if   } \omega_{jt} \ge \underline{\omega}(k_{jt})\\
  0 &\text{   otherwise.}
  \end{cases}
  \end{equation}
\item
  We can only observe firms that satisfy \(\chi_{jt} = 1\).
\end{itemize}

\subsection{Correction in the First
Step}\label{correction-in-the-first-step}

\begin{itemize}
\tightlist
\item
  In the first step, we need no correction because:

  \begin{equation}
  \begin{split}
  &\mathbb{E}\{y_{jt}|l_{jt}, k_{jt}, i_{jt}, \chi_{jt} = 1 \}\\
  &=\beta_l l_{jt} + \phi(k_{jt}, i_{jt}) + \mathbb{E}\{\eta_{jt}|\chi_{jt} = 1\}\\
  &= \beta_l l_{jt} + \phi(k_{jt}, i_{jt}).
  \end{split}
  \end{equation}
\item
  Ex-post shock \(\eta_{jt}\) is independent of continuation/exit
  decision. Therefore, we can identify \(\beta_l\) and \(\phi(\cdot)\)
  as in the previous case.
\end{itemize}

\subsection{Correction in the Second Step I: The Source of
Bias}\label{correction-in-the-second-step-i-the-source-of-bias}

\begin{itemize}
\tightlist
\item
  One the other hand, we need correction in the second step, because:

  \begin{equation}
  \begin{split}
  &\mathbb{E}\{y_{jt} - \beta_l^0 l_{jt}|k_{jt}, i_{jt}, k_{j, t - 1}, l_{j, t - 1}, \chi_{jt} = 1\} \\
  &= \beta_0 + \beta_k k_{jt} + g[\phi^0(k_{jt}, i_{jt}) - (\beta_0 + \beta_k k_{jt})]\\
  & + \mathbb{E}\{\nu_{jt} + \eta_{jt}| k_{jt}, i_{jt}, k_{j, t - 1}, l_{j, t - 1}, \chi_{jt} = 1\}\\
  &= \beta_0 + \beta_k k_{jt} + g[\phi^0(k_{j, t - 1}, i_{j, t - 1}) - (\beta_0 + \beta_k k_{j, t - 1})]\\
  & + \mathbb{E}\{\nu_{jt}| k_{jt}, i_{jt}, k_{j, t - 1}, l_{j, t - 1} , \chi_{jt} = 1\}.
  \end{split}
  \end{equation}

  and

  \begin{equation}
  \mathbb{E}\{\nu_{jt}| k_{jt}, i_{jt}, k_{j, t - 1}, l_{j, t - 1}, \chi_{jt} = 1 \} \neq 0,
  \end{equation}

  since anticipated shock matters continuation/exit decision in period
  \(t\).
\end{itemize}

\subsection{Correction in the Second Step II: Conditional Exit
Probability}\label{correction-in-the-second-step-ii-conditional-exit-probability}

\begin{itemize}
\tightlist
\item
  Let's see that the conditional expectation:

  \begin{equation}
  \begin{split}
  &\mathbb{E}\{\omega_{jt}| k_{jt}, i_{jt}, k_{j, t - 1}, l_{j, t - 1}, \chi_{jt} = 1 \}\\
  &=\mathbb{E}\{\omega_{jt}| k_{jt}, i_{jt}, k_{j, t - 1}, l_{j, t - 1}, \omega_{jt} \ge \underline{\omega}(k_{jt}) \}\\
  &=\int_{\underline{\omega}(k_{jt})} \omega_{jt} \frac{p(\omega_{jt}|\omega_{j, t - 1})}{\int_{\underline{\omega}(k_{jt})} p(\omega|\omega_{j, t - 1}) d\omega } d \omega_{jt}\\
  &\equiv \tilde{g}(\omega_{j, t - 1}, \underline{\omega}(k_{jt})),
  \end{split}
  \end{equation}

  is a function of \(\omega_{j, t - 1}\) and
  \(\underline{\omega}(k_{jt})\).
\end{itemize}

\subsection{Correction in the Second Step III: Invertibility in
Threshold}\label{correction-in-the-second-step-iii-invertibility-in-threshold}

\begin{itemize}
\tightlist
\item
  The propensity of continuation conditional on observed information up
  to period \(t - 1\):

  \begin{equation}
  \begin{split}
  P_{jt} &\equiv \mathbb{P}\{\chi_{jt} = 1|\mathcal{I}_{j, t - 1}\}\\
  &= \mathbb{P}\{\omega_{jt} \ge \underline{\omega}(k_{jt}) |\mathcal{I}_{j, t - 1}\}\\
  &= \mathbb{P}\{g(\omega_{j, t - 1}) + \nu_{jt} \ge \underline{\omega}[(1 - \delta) k_{j, t - 1} + i_{j, t - 1}]|\mathcal{I}_{j, t - 1} \}\\
  &= \mathbb{P}\{ \chi_{jt} = 1| i_{j, t - 1}, k_{j, t - 1}\}.
  \end{split}
  \end{equation}
\item
  \(\rightarrow\) It suffices to condition on
  \(i_{j, t - 1}, k_{j, t - 1}\).
\item
  We also have:

  \begin{equation}
  P_{jt} = \mathbb{P}\{\chi_{jt} = 1| \omega_{j, t - 1}, \underline{\omega}(k_{jt})\},
  \end{equation}

  and it is invertible in \(\underline{\omega}(k_{jt})\), that is,

  \begin{equation}
  \underline{\omega}(k_{jt}) \equiv \psi(P_{jt}, \omega_{j, t - 1}).
  \end{equation}
\end{itemize}

\subsection{Correction in the Second Step IV: Controlling the
Threshold}\label{correction-in-the-second-step-iv-controlling-the-threshold}

\begin{itemize}
\tightlist
\item
  Now, he have:

  \begin{equation}
  \begin{split}
  &\mathbb{E}\{y_{jt} - \beta_l^0 l_{jt}|k_{jt}, i_{jt}, k_{j, t - 1}, l_{j, t - 1}, \chi_{jt} = 1\} \\
  &= \beta_0 + \beta_k k_{jt} + \mathbb{E}\{\omega_{jt}| k_{jt}, i_{jt}, k_{j, t - 1}, l_{j, t - 1} , \chi_{jt} = 1\}\\
  &= \beta_0 + \beta_k k_{jt} + \tilde{g}(\omega_{j, t - 1}, \underline{\omega}(k_{jt}))\\
  &= \beta_0 + \beta_k k_{jt} + \tilde{g}(\omega_{j, t - 1}, \psi(P_{jt}, \omega_{j, t - 1}))\\
  &\equiv \beta_0 + \beta_k k_{jt} + \tilde{\tilde{g}}(\omega_{j, t - 1}, P_{jt})\\
  &= \beta_0 + \beta_k k_{jt} + \tilde{\tilde{g}}[\phi^0(k_{j, t - 1}, i_{j, t - 1}) - (\beta_0 + \beta_k k_{j, t - 1}), P_{jt}].
  \end{split}
  \end{equation}
\item
  At the end, the only difference is to include \(P_{jt}\) as a
  covariate.
\item
  \(P_{jt}\) is a \textbf{known} function of \(i_{j, t - 1}\) and
  \(k_{j, t - 1}\).
\item
  Even if we condition on \(P_{jt} = p\), there are still many
  combinations of \(i_{j, t - 1}\) and \(k_{j, t - 1}\) that gives
  \(P_{jt} = p\).
\item
  With this remaining variation, we can identify \(\beta_0\),
  \(\beta_k\), and \(\tilde{\tilde{g}}\) by the same argument as the
  case without selection, for each \(P_{jt} = p\).
\end{itemize}

\subsection{\texorpdfstring{Three Step Estimation of
\citet{Olley1996}}{Three Step Estimation of @Olley1996}}\label{three-step-estimation-of-olley1996}

\begin{itemize}
\tightlist
\item
  \textbf{Zero step}: Estimate the propensity score:

  \begin{equation}
  P_{jt} = 1\{\chi_{jt} = 1| i_{j, t - 1}, k_{j, t - 1}\},
  \end{equation}

  by a kernel estimator.
\item
  Insert the resulting estimates \(\widehat{P}_{jt}\) into the first and
  second steps.
\end{itemize}

\subsection{Zero Investment Problem}\label{zero-investment-problem}

\begin{itemize}
\tightlist
\item
  One of the key assumptions in OP method was invertibility between
  anticipated shock and investment:

  \begin{equation}
  \omega_{jt} = i^{-1}(k_{jt}, i_{jt}) \equiv h(k_{jt}, i_{jt}).
  \end{equation}
\item
  However, in micro data, zero investment is a rule rather than
  exceptions.
\item
  Then, the invertibility does not hold globally: there are some region
  of the anticipated shock in which the investment takes value zero.
\end{itemize}

\subsection{Tackle Zero Investment Problem I: Discard Some
Data}\label{tackle-zero-investment-problem-i-discard-some-data}

\begin{itemize}
\tightlist
\item
  Discard a data \((j, t)\) such that \(i_{j, t - 1} = 0\).
\item
  Use a data \((j, t)\) such that \(i_{j, t - 1} > 0\).
\item
  Then, invertibility recovers on this selected sample.
\item
  This \textbf{does not} cause bias in the estimator because
  \(\nu_{jt}\) in :

  \begin{equation}
  \beta_0 + \beta_l k_{jt} + g[\phi^0(k_{j, t - 1}, i_{j, t - 1}) - (\beta_0 + \beta_k k_{j, t - 1})] + \nu_{jt} + \eta_{jt},
  \end{equation}

  is independent of the event up to \(t - 1\), including
  \(i_{j, t - 1}\).
\item
  However, this \textbf{does} cause information loss. The loss is high
  if the proportion of the sample such that \(i_{j, t - 1} = 0\) is
  high.
\end{itemize}

\subsection{Tackle Zero Investment Problem II: Use Another
Proxy}\label{tackle-zero-investment-problem-ii-use-another-proxy}

\begin{itemize}
\tightlist
\item
  Investment is just a possible proxy for the anticipated shock.
\item
  Intermediate inputs can be used as proxies as well
  \citep{Levinsohn2003}.
\item
  The problem is that these intermediate inputs are included in the
  gross production function, whereas investment is excluded.
\item
  Let \(m_{jt}\) be the log material input, and assume that the
  production function takes the form of:

  \begin{equation}
  y_{jt} = \beta_0 + \beta_l l_{jt} + \beta_k k_{jt} + \beta_m m_{jt} + \omega_{jt} + \eta_{jt}.
  \end{equation}
\item
  In addition, assume that the \textbf{optimal policy function} for
  \(m_{jt}\) is strictly monotonic in the ex-ante shock, and hence is
  invertible:

  \begin{equation}
  m_{jt} = m(k_{jt}, \omega_{jt}) \Leftrightarrow \omega_{jt} = m^{-1}(m_{jt}, k_{jt}) \equiv h(m_{jt}, k_{jt}). \label{eq:material}
  \end{equation}
\item
  \textbf{First step}:

  \begin{equation}
  \begin{split}
  y_{jt} &= \beta_0 + \beta_l l_{jt} + \beta_k k_{jt} + \beta_m m_{jt} + h(m_{jt}, k_{jt}) + \eta_{jt}\\
  &= \beta_l l_{jt} + \phi(m_{jt}, k_{jt}) + \eta_{jt}.
  \end{split}
  \end{equation}
\item
  We can identify \(\beta_l\) and \(\phi\) by exploiting the moment
  condition:

  \begin{equation}
  \begin{split}
  & \mathbb{E}\{\eta_{jt}|l_{jt}, m_{jt}, k_{jt}, i_{jt}\} = 0\\
  & \Leftrightarrow \mathbb{E}\{y_{jt} - \beta_0 - \beta_l l_{jt} - \phi(m_{jt}, k_{jt}) |l_{jt}, m_{jt}, k_{jt}\} = 0,
  \end{split}
  \end{equation}

  if \textbf{there is enough variation} in \(l_{jt}, m_{jt}, k_{jt}\).
\item
  \textbf{Second step}:

  \begin{equation}
  \begin{split}
  &y_{jt} - \beta_l^0 l_{jt}\\
  & = \beta_k k_{jt} + \beta_m m_{jt} + g[\phi^0(m_{j, t - 1}, k_{j, t - 1}) - \beta_k k_{j, t - 1} - \beta_m m_{j, t - 1}]\\
  & + \nu_{jt} + \eta_{jt}.
  \end{split}
  \end{equation}
\item
  We can identify \(\beta_k\), \(\beta_m\), and \(g\) by exploiting the
  moment condition:

  \begin{equation}
  \begin{split}
  \mathbb{E}\{\nu_{jt} + \eta_{jt} | k_{jt}, m_{j, t - 1}, k_{j,t - 1}\} = 0.
  \end{split}
  \end{equation}
\item
  Because \(m_{jt}\) is correlated with \(\nu_{jt}\), the moment should
  not condition on \(m_{jt}\).
\item
  The identification of \(\beta_{m}\) comes from
  \(\beta_m m_{j, t - 1}\).
\end{itemize}

\subsection{\texorpdfstring{One-step Estimation of \citet{Olley1996} and
\citet{Levinsohn2003}}{One-step Estimation of @Olley1996 and @Levinsohn2003}}\label{one-step-estimation-of-olley1996-and-levinsohn2003}

\begin{itemize}
\tightlist
\item
  \citet{Levinsohn2003} can be estimated in the similar two-step method.
\item
  We can jointly estimate the parameters in first and second steps to
  improve the efficiency \citep{Wooldridge2009}.
\item
  We estimate under the assumptions of \citet{Olley1996}:

  \begin{equation}
  y_{jt} = \beta_0 + \beta_1 l_{jt} + \beta_k k_{jt} + \omega_{jt} + \eta_{jt}.
  \end{equation}
\item
  The first step exploits the following moment:

  \begin{equation}
  \mathbb{E}\{\eta_{jt}|l_{jt}, k_{jt}, i_{jt}\} = 0,
  \end{equation}

  that is:

  \begin{equation}
  \mathbb{E}\{y_{jt} - \beta_1 l_{jt} - \beta_0 - \beta_k k_{jt} - \omega(k_{jt}, i_{jt})|l_{jt}, k_{jt}, i_{jt}\} = 0. \label{eq:opfirst}
  \end{equation}
\item
  We can reinforce the moment condition as:

  \begin{equation}
  \mathbb{E}\{\eta_{jt}|l_{jt}, k_{jt}, i_{jt}, \cdots, l_{j1}, k_{j1}, i_{j1}\} = 0
  \end{equation}

  if we assume that lagged inputs are correlated with the current inputs
  and \(\eta_{jt}\) is independent.
\item
  The second step exploits the following moment:

  \begin{equation}
  \mathbb{E}\{\nu_{jt}|k_{jt}, i_{j, t - 1}, l_{j, t - 1}\} = 0,
  \end{equation}

  that is:

  \begin{equation}
  \mathbb{E}\{y_{jt} - \beta_0 - \beta_1 l_{jt} - \beta_k k_{jt} - g[\omega(k_{j,t - 1}, i_{j, t - 1})]|k_{jt}, i_{j, t - 1}, l_{j, t - 1}\} = 0. \label{eq:opsecond}
  \end{equation}
\item
  We can reinforce the moment condition as:

  \begin{equation}
  \mathbb{E}\{\nu_{jt}|k_{jt}, i_{j, t - 1}, l_{j, t - 1}, \cdots, k_{j1}, i_{j1}, l_{j1}\} = 0,
  \end{equation}

  if we assume that lagged input are correlated with the current inputs
  and \(\nu_{jt} + \eta_{jt}\) are independent.
\item
  We can construct a GMM estimator based on equations \eqref{eq:opfirst}
  and \eqref{eq:opsecond}.
\item
  The one-step estimator can be more efficient but can be
  computationally heavier than the two-step estimator.
\end{itemize}

\subsection{Scalar Unobservable Problem: Finite-order Markov
Process}\label{scalar-unobservable-problem-finite-order-markov-process}

\begin{itemize}
\tightlist
\item
  Borrow the idea of using the first-order condition to resolve the
  collinearity problem \citep{Gandhi2017a}.
\item
  We have assumed that anticipated shocks follow a first-order Markov
  process:

  \begin{equation}
  \omega_{jt} = g(\omega_{j, t - 1}) + \nu_{jt}.
  \end{equation}
\item
  However, it may be true that it has more than one lags, for example:

  \begin{equation}
  \omega_{jt} = g(\omega_{j, t - 1}, \omega_{j, t - 2}) + \nu_{jt}.
  \end{equation}
\item
  Then, we need proxies as many as the number of unobservables:

  \begin{equation}
  \begin{pmatrix}
  i_{jt} \\ m_{jt} 
  \end{pmatrix}
  = \Gamma(k_{jt}, \omega_{jt}, \omega_{j, t - 1}),
  \end{equation}

  such that the policy function for the proxies is a bijection in
  \((\omega_{jt}, \omega_{j, t - 1})\).
\item
  Then, we can have:

  \begin{equation}
  \omega_{jt} = \Gamma_1^{-1}(k_{jt}, i_{jt}, m_{jt}).
  \end{equation}
\item
  The reminder goes as in the standard OP method.
\end{itemize}

\subsection{Scalar Unobservable Problem: Demand and Productivity
Shocks}\label{scalar-unobservable-problem-demand-and-productivity-shocks}

\begin{itemize}
\tightlist
\item
  There may be a demand shock \(\mu_{jt}\) that also follows first-order
  Markov process.
\item
  Then, the policy function depend both on \(\mu_{jt}\) and
  \(\omega_{jt}\).
\item
  We again need proxies as many as the number of unobservable.
\item
  Suppose that we can observe the price of the firm \(p_{jt}\).
\item
  Inverting the policy function:

  \begin{equation}
  \begin{pmatrix}
  i_{jt}\\ p_{jt}
  \end{pmatrix}
  = \Gamma(k_{jt}, \omega_{jt}, \mu_{jt}).
  \end{equation}

  yields:

  \begin{equation}
  \omega_{jt} = \Gamma_1^{- 1}(k_{jt}, i_{jt}, p_{jt}).
  \end{equation}
\item
  If \(\omega_{jt}\) only depends on \(\omega_{j, t - 1}\) but not on
  \(\mu_{j, t - 1}\), then the second step of the modified OP method is
  to estimate:

  \begin{equation}
  \begin{split}
  y_{jt} - \hat{\beta}_l l_{jt} 
  &= \beta_0 + \beta_k k_{jt}\\
  & + g(\omega_{j, t - 1}) + \nu_{jt} + \eta_{jt}\\
  &= \beta_0 + \beta_k k_{jt}\\
  & + g(\hat{\phi}_{j, t - 1} - \beta_0 - \beta_k k_{j, t - 1}) + \nu_{jt} + \eta_{jt}.
  \end{split}
  \end{equation}
\item
  It goes as in the standard OP method.
\item
  If \(\omega_{jt}\) depends both on \(\omega_{j, t - 1}\) and
  \(\mu_{j, t - 1}\), the second step regression equation will be:

  \begin{equation}
  \begin{split}
  y_{jt} - \hat{\beta}_l l_{jt} 
  &= \beta_0 + \beta_k k_{jt}\\
  & + g(\omega_{j, t - 1}, \mu_{j, t - 1}) + \nu_{jt} + \eta_{jt}\\
  &= \beta_0 + \beta_k k_{jt}\\
  & + g(\hat{\phi}_{j, t - 1} - \beta_0 - \beta_k k_{j, t - 1}, \mu_{j, t - 1}) + \nu_{jt} + \eta_{jt}.
  \end{split}
  \end{equation}
\item
  We still have to control \(\mu_{j, t - 1}\) in the second step.
\item
  Invert the policy function for \(\mu_{j, t - 1}\) to get:

  \begin{equation}
  \mu_{j, t - 1} = \Gamma_2^{- 1}(k_{j, t - 1}, i_{j, t - 1}, p_{j, t - 1}),
  \end{equation}

  and plug it into the second step regression equation to get:

  \begin{equation}
  \begin{split}
  &y_{jt} - \hat{\beta}_l l_{jt}\\
  &= \beta_0 + \beta_k k_{jt}\\
  &+g(\hat{\phi}_{j, t - 1} - \beta_0 - \beta_k k_{j, t - 1}, \Gamma_2^{- 1}(k_{j, t - 1}, i_{j, t - 1}, p_{j, t - 1})) + \nu_{jt} + \eta_{jt}.
  \end{split}
  \end{equation}
\item
  The parameters \(\beta_0\) and \(\beta_k\) \textbf{cannot} be
  identified only with this observation, because \(\Gamma_2^{-1}\) is
  \textbf{unknown non-parametric} function: it can mean any function of
  \((k_{j, t - 1}, i_{j, t - 1}, p_{j, t - 1})\).
\item
  To estimate such a model, we jointly estimate the demand function
  along with the production function.
\item
  At this point, we do not investigate it further because we have not
  yet learned how to estimate the demand function.
\item
  For now just keep in mind that:

  \begin{itemize}
  \tightlist
  \item
    There has to be as many proxies as the dimension of the unobservable
    state variables.
  \item
    It is okay that the unobservable state variable includes a demand
    shock.
  \item
    It can be problematic when the unobservable demand shock affect the
    evolution of the anticipated productivity shock.
  \end{itemize}
\end{itemize}

\subsection{Collinearity Problem}\label{collinearity-problem}

\begin{itemize}
\tightlist
\item
  The collinearity problem is formally pointed out by
  \citet{Ackerberg2015}.
\item
  This paper is finally published in 2015, but has been circulated since
  2005.
\item
  We assumed that \(k_{jt}\) and \(\omega_{jt}\) are state variables.
\item
  Then the policy function for labor input should take the form of:

  \begin{equation}
  l_{jt} = l(k_{jt}, \omega_{jt}).
  \end{equation}
\item
  However, because \(\omega_{jt} = h(i_{jt}, k_{jt})\), we have:

  \begin{equation}
  l_{jt} = l[k_{jt}, h(i_{jt}, k_{jt})] = \tilde{l}(i_{jt}, k_{jt}).
  \end{equation}
\item
  Therefore, in the first stage, we encounter a multicollinearity
  problem:

  \begin{equation}
  \begin{split}
  y_{jt} &= \beta_0 + \beta_l \tilde{l}(i_{jt}, k_{jt}) + \phi(i_{jt}, k_{jt}) + \eta_{jt}\\
  &\equiv \tilde{\phi}(i_{jt}, k_{jt}).
  \end{split}
  \end{equation}
\item
  Thus, \(\beta_l\) cannot be identified in the first step.
\item
  The second step becomes:

  \begin{equation}
  y_{jt} = \beta_0 + \beta_l l_{jt} + \beta_k k_{jt} + g[\tilde{\phi}(i_{j, t - 1}, k_{j, t - 1}) - \beta_0 - \beta_l l_{j, t - 1} - \beta_k k_{jt}] + \nu_{jt} + \eta_{jt}
  \end{equation}
\item
  Because \(l_{jt}\) is correlated with \(\nu_{jt}\), moment can only
  condition on \(l_{j, t - 1}\).
\item
  However, conditioning on \(k_{j, t - 1}\) and \(i_{j, t - 1}\), again
  there is no remaining variation in \(l_{j, t - 1}\).
\item
  Therefore, \(\beta_l\) cannot be identified either in the second step.
\item
  \textbf{\(\beta_l\) cannot be identified!}
\end{itemize}

\subsection{Tackle Collinearity Problem: Peculiar
Assumptions}\label{tackle-collinearity-problem-peculiar-assumptions}

\begin{itemize}
\tightlist
\item
  To make Olley-Pakes/Levinsohn-Petrin approach workable, we need
  peculiar data generating process for \(l_{jt}\).
\item
  Consider Levinsohn-Petrin framework.
\end{itemize}

\begin{enumerate}
\def\labelenumi{\arabic{enumi}.}
\tightlist
\item
  There is an optimization error in \(l_{jt}\).

  \begin{itemize}
  \tightlist
  \item
    If it is not i.i.d over time, it becomes a state variable and enters
    to the policy for \(m_{jt}\), violating the scalar unobserved
    heterogeneity assumption of \(m_{jt}\).
  \item
    If there is an optimization error for \(m_{jt}\), this again
    violates the scalar unobserved heterogeneity assumption.
  \end{itemize}
\item
  \(k_{jt}\) is realized, \(\omega_{jt}\) is observed, \(m_{jt}\) and
  \(i_{jt}\) are determined, a new i.i.d. unexpected shock is observed,
  \(l_{jt}\) is determined, and \(\eta_{jt}\) is observed.

  \begin{itemize}
  \tightlist
  \item
    If it is not i.i.d over time, it becomes a state variable and enters
    to the policy for \(m_{jt}\), violating the scalar unobserved
    heterogeneity assumption.
  \end{itemize}
\item
  \(k_{jt}\) is realized, an unexpected shock is observed, \(l_{jt}\) is
  determined, \(\omega_{jt}\) is observed, \(m_{jt}\) and \(i_{jt}\) are
  determined, and \(\eta_{jt}\) is observed (\citet{Ackerberg2016}
  recommends this assumption).

  \begin{itemize}
  \tightlist
  \item
    In this case, the unexpected shock can be serially correlated,
    because it suffices to know \(k_{jt}\), \(i_{jt}\), \(l_{jt}\) to
    decide \(m_{jt}\). It does not have to predict the future unexpected
    shock based on the realization of the current shock because
    \(m_{jt}\) is a static decision.
  \item
    This changes the optimal policy function of \(m_{jt}\)
    \eqref{eq:material} to:

    \begin{equation}
    m_{jt} = m(k_{jt}, \omega_{jt}, l_{jt}).
    \end{equation}
  \item
    The first step:

    \begin{equation}
    \begin{split}
    y_{jt} &= \beta_0 + \beta_l l_{jt} + \beta_k k_{jt} + h(k_{jt}, m_{jt}, l_{jt}) + \eta_{jt}\\
    &= \psi(k_{jt}, m_{jt}, l_{jt}) + \eta_{jt}.\\
    \Rightarrow & \mathbb{E}\{y_{jt} - \psi(k_{jt}, m_{jt}, l_{jt})|k_{jt}, m_{jt}, l_{jt}\} = 0.
    \end{split}
    \end{equation}
  \item
    The second step:

    \begin{equation}
    \begin{split}
    y_{jt} &= \beta_0 + \beta_l l_{jt} + \beta_k k_{jt} + g[\psi(k_{j, t - 1}, m_{j, t - 1}, l_{j, t - 1}) - \beta_0 - \beta_l l_{j, t - 1} - \beta_k k_{j, t - 1}] + \nu_{jt} + \eta_{jt}\\
    \Rightarrow & \mathbb{E}\{y_{jt} - \beta_0 - \beta_l l_{jt} - \beta_k k_{jt} - g[\psi(k_{j, t - 1}, m_{j, t - 1}, l_{j, t - 1}) - \beta_0 - \beta_l l_{j, t - 1} - \beta_k k_{j, t - 1}]|k_{j, t - 1}, i_{j, t - 1}, l_{j, t - 1}, m_{j, t - 1}\}
    \end{split}
    \end{equation}
  \item
    \(m_{jt}\) has to be excluded from the production function, i.e., it
    has to be a value-added production function. Otherwise,
    \(\beta_m m_{jt}\) and \(\beta_m m_{j, t - 1}\) appear in the second
    step. Because \(m_{jt}\) is correlated with \(\nu_{jt}\), the only
    hope is to vary \(m_{j, t - 1}\). But there is no additional
    variation in \(m_{j, t - 1}\) conditional on \(k_{j, t - 1}\),
    \(i_{j, t - 1}\), and \(l_{j, t - 1}\).
  \end{itemize}
\end{enumerate}

\subsection{Tackle Collinearity Problem: Share
Regression}\label{tackle-collinearity-problem-share-regression}

\begin{itemize}
\item
  How to avoid the peculiar assumptions on shocks and timing of
  decisions?
\item
  How to identify gross production function avoiding the third
  assumption by \citet{Ackerberg2015}?
\item
  Return to the old literature using the first-order condition.
\item
  Let \(w_t\) be wage and \(p_t\) be the product price.
\item
  Assume that the factor market is competitive.
\item
  Then, the first-order condition for profit maximization with respect
  to \(L_{jt}\) is:

  \begin{equation}
  \begin{split}
  &P_t F_L(L_{jt}, K_{jt})e^{\omega_{jt}} \mathbb{E} e^{\eta_{jt}} = w_t\\
  &\Leftrightarrow \frac{P_t F_L(L_{jt}, K_{jt})e^{\omega_{jt}} \mathbb{E} e^{\eta_{jt}}}{F(L_{jt}, K_{jt}) } = \frac{w_t}{F(L_{jt}, K_{jt}) }\\
  &\Leftrightarrow \frac{F_L(L_{jt}, K_{jt}) L_{jt}}{F(L_{jt}, K_{jt})  e^{\eta_{jt}} } = \frac{w_t L_{jt}}{P_t \underbrace{F(L_{jt}, K_{jt}) e^{\omega_{jt}} e^{\eta_{jt}}}_{Y_{jt}} },
  \end{split}
  \end{equation}

  where the right hand side is expenditure share to the labor, which is
  observed.
\item
  Furthermore, on the left hand side, we only have \(\eta_{jt}\), which
  is independent of inputs.
\item
  Let \(s_{jt}\) be the log of expenditure share to the labor, and take
  a log of the previous equation gives:

  \begin{equation}
  \begin{split}
  s_{jt} &= \log [F_L(L_{jt}, K_{jt}) L_{jt} \mathbb{E} e^{\eta_{jt}} / F(L_{jt}, K_{jt})] - \eta_{jt}\\
  & = \log(\beta_l) + \ln \mathbb{E} e^{\eta_{jt}} - \eta_{jt}.
  \end{split}
  \end{equation}
\item
  Remember that the coefficient in the Cobb-Douglas function is equal to
  the expenditure share.
\item
  In general, share regression provides additional variation to identify
  the elasticity of anticipated production with respect to the labor.
  Then we can follow the standard OP method to recover other parameters.
\end{itemize}

\section{Cost Function Estimation}\label{cost-function-estimation}

\subsection{Cost Function: Duality}\label{cost-function-duality}

\begin{itemize}
\item
  Given a function \(y = F(x)\) such that:

  \begin{itemize}
  \tightlist
  \item
    Add factor market structure.
  \item
    Add cost minimization.
  \end{itemize}
\item
  \(\rightarrow\) There exists a unique \textbf{cost function}
  \(c = C(y, p)\):

  \begin{itemize}
  \tightlist
  \item
    \textbf{Positivity}: positive for positive input prices and a
    positive.
  \item
    \textbf{Homogeneity}: homogeneous of degree one in the input prices.
  \item
    \textbf{Monotonicity}: increasing in the input prices and in the
    level of output.
  \item
    \textbf{Concavity}: concave in the input prices.
  \end{itemize}
\item
  Given a function \(c = C(y, p)\) such that:

  \begin{itemize}
  \tightlist
  \item
    \textbf{Positivity}: positive for positive input prices and a
    positive.
  \item
    \textbf{Homogeneity}: homogeneous of degree one in the input prices.
  \item
    \textbf{Monotonicity}: increasing in the input prices and in the
    level of output.
  \item
    \textbf{Concavity}: concave in the input prices.
  \end{itemize}
\item
  \(\rightarrow\) There exists a unique production function \(F(x)\)
  that yields \(C(y, p)\) as a solution to the cost minimization
  problem:

  \begin{equation}
  C(y, p) = \min_{x} p'x \text{   s.t.   } F(x) \ge y.
  \end{equation}
\item
  If the latter condition holds, the function \(C\) is said to be
  \textbf{integrable}.
\item
  It is rare that you can find a closed-form cost function of a
  production function.
\item
  It makes sense to start from cost function.
\item
  The duality ensures that there is a one-to-one mapping between a class
  of cost function and a class of production function.
\item
  If you accept competitive factor markets and cost minimization,
  identifying a cost function is equivalent to identifying a production
  function.
\item
  We used this idea in the last slides to identify the parameters
  regarding static decision variables.
\item
  See \citet{Jorgenson1986} for the literature in this topic up to the
  mid 80s.
\end{itemize}

\subsection{Translog Cost Function}\label{translog-cost-function}

\begin{itemize}
\tightlist
\item
  One of the popular specifications:

  \begin{equation}
  \begin{split}
  \ln c &= \alpha_0 + \alpha_p' \ln p + \alpha_y \ln y + \frac{1}{2} \ln p' B_{pp} \ln p\\
  & + \ln p' \beta_{py} \ln y + \frac{1}{2}\beta_{yy}(\ln y)^2.
  \end{split}
  \end{equation}
\item
  It assumes that the first and second order elasticities are constant.
\item
  A second-order (log) Taylor approximation of a general cost function.
\end{itemize}

\subsection{Translog Cost Function:
Integrability}\label{translog-cost-function-integrability}

\begin{itemize}
\tightlist
\item
  Translog cost function is known to be integrable if the following
  conditions hold:
\item
  \textbf{Homogeneity}: the cost shares and the cost flexibility are
  homogeneity of degree zero: \(B_{pp}1 = 0\), \(\beta_{py}'1 = 0\).
\item
  \textbf{Cost exhaustion}: the sum of cost shares is equal to unity:
  \(\alpha_p'1 = 1\), \(B_{pp}'1 = 0\), \(\beta_{py}'1 = 0\).
\item
  \textbf{Symmetry}: the matrix of share elasticities, biases of scale,
  and the cost flexibility elasticity is symmetric:

  \begin{equation}
  \begin{pmatrix}
  B_{pp} & \beta_{py}\\
  \beta_{py}' & \beta_{yy}
  \end{pmatrix}
  =
  \begin{pmatrix}
  B_{pp} & \beta_{py}\\
  \beta_{py}' & \beta_{yy}
  \end{pmatrix}'.
  \end{equation}
\item
  \textbf{Monotonicity}: The matrix of share elasticities
  \(B_{pp} + vv' - diag(v)\) is positive semi-definite.
\end{itemize}

\subsection{Two Approaches}\label{two-approaches}

\begin{enumerate}
\def\labelenumi{\arabic{enumi}.}
\tightlist
\item
  Cost data approach.

  \begin{itemize}
  \tightlist
  \item
    Use accounting cost data.
  \item
    It does not depend on behavioral assumption.
  \item
    One can impose restrictions of assuming cost minimization.
  \item
    The accounting cost data may not represent economic cost.
  \end{itemize}
\item
  Revealed preference approach.

  \begin{itemize}
  \tightlist
  \item
    Assume decision problem for firms.
  \item
    Assume profit maximization.
  \item
    Reveal the costs from firm's equilibrium strategy.
  \item
    It depends on structural assumptions.
  \item
    It reveals the cost as perceived by firms.
  \end{itemize}
\end{enumerate}

\subsection{Cost Data Approach}\label{cost-data-approach}

\begin{itemize}
\tightlist
\item
  Estimating a cost function using cost data from accounting data.
\item
  \citet{McElroy1987} is one of the most flexible and robust frameworks.
\item
  The approach is somewhat getting less popular in IO researchers.
\item
  Recently, the approach is not popular among IO researchers.
\item
  I \textit{conjecture} one of the reasons for this is that IO
  researchers believe cost data taken from accounting information does
  not capture all the costs firms face.
\item
  However, it is good to know the classical literature because it
  sometimes gives a new insight.
\item
  cf. \citet{Byrne2015} : Propose a novel method to combine accounting
  cost data to estimate demand and cost function jointly without using
  instrumental variable approach.
\end{itemize}

\subsection{Revealed Preference
Approach}\label{revealed-preference-approach}

\begin{itemize}
\tightlist
\item
  Another approach is to \textbf{reveal} the marginal cost from firm's
  price/quantity setting behavior assuming it is maximizing profit.

  \begin{itemize}
  \tightlist
  \item
    A parameter affects economic agent's action.
  \item
    Therefore, economic agent's action \textbf{reveals} the information
    about the parameter.
  \item
    See \citet{Bresnahan1981} and \citet{Bresnahan1989} for reference.
  \end{itemize}
\item
  We have shown that the assumption on the factor market and cost
  function minimization gives restriction on the cost parameters.
\item
  We may further assume the product market structure and profit
  maximization to identify cost parameters.
\item
  Example: In a competitive market, the equilibrium price is equal to
  the marginal cost. Therefore, the marginal cost is identified from
  prices.
\item
  What if the competition is imperfect?
\end{itemize}

\subsection{Single-product Monopolist}\label{single-product-monopolist}

\begin{itemize}
\item
  This approach requires researcher to specify the decision problem of a
  firm.
\item
  Assume that the firm is a single-product monopolist.
\item
  Let \(D(p)\) be the demand function.
\item
  Let \(C(q)\) be the cost function.
\item
  Temporarily, assume that we \textbf{know} the demand function.
\item
  We learn how to estimate demand functions in coming weeks.
\item
  The only unknown parameter is the cost function.
\item
  The monopolist solves:

  \begin{equation}
  \max_{p} D(p)p - C(D(p)).
  \end{equation}
\item
  The first-order condition w.r.t. \(p\) for profit maximization is:

  \begin{equation}
  \begin{split}
  &D(p) + pD'(p) - C'(D(p)) D'(p) = 0.\\
  &\Leftrightarrow C'(D(p)) = \underbrace{\frac{D(p) + pD'(p)}{D'(p)}}_{\text{$p$ is observed and $D(p)$ is known.}}
  \end{split}
  \end{equation}
\item
  This identifies the marginal cost
  \textit{at the equilibrium quantity}.
\item
  To trace out the entire marginal cost function, you need a demand
  shifter \(Z\) that changes the equilibrium: \(D(p, Z)\).

  \begin{equation}
  C'(D(p, z)) = \frac{D(p, z) + pD'(p, z)}{D'(p, z)}
  \end{equation}
\item
  This identifies the marginal cost function
  \textit{at the equilibrium quantity when $Z = z$}.
\item
  If the equilibrium quantities cover the domain of the marginal cost
  function when the demand shifter \(Z\) moves around, then it
  identifies the entire marginal cost function.
\end{itemize}

\subsection{Unobserved Heterogeneity in the Cost
Function}\label{unobserved-heterogeneity-in-the-cost-function}

\begin{itemize}
\item
  Previously we did not consider any unobserved heterogeneity in the
  cost function.
\item
  Now suppose that the cost function is given by:

  \begin{equation}
  C(q) = \tilde{C}(q) + q \epsilon + \mu,
  \end{equation}

  and \(\epsilon\) and \(\mu\) are not observed.
\item
  Moreover, because it includes anticipated shocks, it is likely to be
  correlated with input decisions and hence the output.
\item
  The first-order condition w.r.t. \(p\) for profit maximization is:

  \begin{equation}
  \begin{split}
  &D(p, z) + pD'(p, z) - [\tilde{C}'(D(p, z)) + \epsilon]D'(p, z) = 0.\\
  &\Leftrightarrow \tilde{C}'(D(p, z))  = \frac{D(p, z) + pD'(p, z)}{D'(p,z)} - \epsilon.
  \end{split}
  \end{equation}
\item
  Take the expectation conditional on \(Z = z\):

  \begin{equation}
  \tilde{C}'(D(p, z)) = \frac{D(p, z) + pD'(p, z)}{D'(p, z)} - \mathbb{E}\{\epsilon|Z = z\}.
  \end{equation}
\item
  If \(Z\) and \(\epsilon\) is independent, then the last term becomes
  zero and we can follow the same argument as before to trace out the
  marginal cost function.
\end{itemize}

\subsection{Multi-product Monopolist
Case}\label{multi-product-monopolist-case}

\begin{itemize}
\item
  Demand for good \(j\) is \(D_j(p)\) given a price vector \(p\).
\item
  Cost for producing a vector of good \(q\) is \(C(q)\).
\item
  Demand function is \textbf{known} but cost function is not known.
\item
  The monopolist solves:

  \begin{equation}
  \max_{p} \sum_{j = 1}^J p_j D_j(p) - C(D_1(p), \cdots, D_J(p)).
  \end{equation}
\item
  The first-order condition w.r.t. \(p_i\) for profit maximization is:

  \begin{equation}
  \begin{split}
  &D_i(p) + p_i \sum_{j = 1}^J \frac{\partial D_j(p)}{\partial p_i} = \sum_{j = 1}^J \frac{\partial C(D_1(p), \cdots, D_J(p))}{\partial q_j} \frac{\partial D_j(p)}{\partial p_i}.\\
  &= 
  \begin{pmatrix}
  \frac{\partial D_1(p)}{\partial p_i} & \cdots & \frac{\partial D_J(p)}{\partial p_i}
  \end{pmatrix}
  \begin{pmatrix}
  \frac{\partial C(D_1(p), \cdots, D_J(p))}{\partial q_1}\\
  \vdots\\
  \frac{\partial C(D_1(p), \cdots, D_J(p))}{\partial q_J}
  \end{pmatrix}
  \end{split}
  \end{equation}
\item
  Summing up, the first-order condition w.r.t. \(p\) is summarized as:

  \begin{equation}
  \begin{split}
  &\begin{pmatrix}
   D_1(p) + p_1 \sum_{j = 1}^J \frac{\partial D_j(p)}{\partial p_1}\\
   \vdots\\
   D_J(p) + p_J \sum_{j = 1}^J \frac{\partial D_j(p)}{\partial p_J}
  \end{pmatrix} 
  =
  \begin{pmatrix}
  \frac{\partial D_1(p)}{\partial p_1} & \cdots & \frac{\partial D_J(p)}{\partial p_1}\\
  \vdots\\
  \frac{\partial D_1(p)}{\partial p_J} & \cdots & \frac{\partial D_J(p)}{\partial p_J}
  \end{pmatrix}
  \begin{pmatrix}
  \frac{\partial C(D_1(p), \cdots, D_J(p))}{\partial q_1}\\
  \vdots\\
  \frac{\partial C(D_1(p), \cdots, D_J(p))}{\partial q_J}
  \end{pmatrix}\\
  &\Leftrightarrow
  \begin{pmatrix}
  \frac{\partial C(D_1(p), \cdots, D_J(p))}{\partial q_1}\\
  \vdots\\
  \frac{\partial C(D_1(p), \cdots, D_J(p))}{\partial q_J}
  \end{pmatrix} = 
  \underbrace{\begin{pmatrix}
  \frac{\partial D_1(p)}{\partial p_1} & \cdots & \frac{\partial D_J(p)}{\partial p_1}\\
  \vdots\\
  \frac{\partial D_1(p)}{\partial p_J} & \cdots & \frac{\partial D_J(p)}{\partial p_J}
  \end{pmatrix}^{-1}  
  \begin{pmatrix}
   D_1(p) + p_1 \sum_{j = 1}^J \frac{\partial D_j(p)}{\partial p_1}\\
   \vdots\\
   D_J(p) + p_J \sum_{j = 1}^J \frac{\partial D_j(p)}{\partial p_J}
  \end{pmatrix}.}_{\text{$p$ is observed and $D(p)$s are known.}}
  \end{split}
  \end{equation}
\item
  Hence, the cost function is identified.
\item
  Including unobserved heterogeneity in the cost function causes the
  same problem as in the previous case.
\end{itemize}

\subsection{Oligopoly}\label{oligopoly}

\begin{itemize}
\item
  There are firm \(j = 1, \cdots, J\) and they sell product
  \(j = 1, \cdots, J\), that is, firm = product (for simplicity).
\item
  Consider a price setting game. When the price vector is \(p\), demand
  for product \(j\) is given by \(D_j(p)\).
\item
  The cost function for firm \(j\) is \(C_j(q_j)\).
\item
  Given other firms' price \(p_{-j}\), firm \(j\) solves:

  \begin{equation}
  \max_{p_j} D_j(p) p_j - C_j(D_j(p)).
  \end{equation}
\item
  The first-order condition w.r.t. \(p_j\) for profit maximization is:

  \begin{equation}
  \begin{split}
  &D_j(p) + \frac{\partial D_j(p)}{\partial p_j} p_j = \frac{\partial C_j(D_j(p))}{\partial q_j} \frac{\partial D_j(p)}{\partial p_j}.\\
  &\frac{\partial C_j(D_j(p))}{\partial q_j} = \underbrace{\frac{\partial D_j(p)}{\partial p_j}^{-1}[D_j(p) + \frac{\partial D_j(p)}{\partial p_j} p_j ]}_{\text{$p$ is observed and $D_j(p)$ is known}}.
  \end{split}
  \end{equation}
\item
  In Nash equilibrium, these equations jointly hold for all firms
  \(j = 1, \cdots, J\).{]}
\item
  Including unobserved heterogeneity in the cost function causes the
  same problem as in the previous case.
\end{itemize}

\bibliography{library.bib,packages.bib}


\end{document}
