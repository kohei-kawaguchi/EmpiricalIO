\documentclass[]{book}
\usepackage{lmodern}
\usepackage{amssymb,amsmath}
\usepackage{ifxetex,ifluatex}
\usepackage{fixltx2e} % provides \textsubscript
\ifnum 0\ifxetex 1\fi\ifluatex 1\fi=0 % if pdftex
  \usepackage[T1]{fontenc}
  \usepackage[utf8]{inputenc}
\else % if luatex or xelatex
  \ifxetex
    \usepackage{mathspec}
  \else
    \usepackage{fontspec}
  \fi
  \defaultfontfeatures{Ligatures=TeX,Scale=MatchLowercase}
\fi
% use upquote if available, for straight quotes in verbatim environments
\IfFileExists{upquote.sty}{\usepackage{upquote}}{}
% use microtype if available
\IfFileExists{microtype.sty}{%
\usepackage{microtype}
\UseMicrotypeSet[protrusion]{basicmath} % disable protrusion for tt fonts
}{}
\usepackage[margin=1in]{geometry}
\usepackage{hyperref}
\hypersetup{unicode=true,
            pdftitle={ECON 6120I Topics in Empirical Industrial Organization},
            pdfauthor={Kohei Kawaguchi},
            pdfborder={0 0 0},
            breaklinks=true}
\urlstyle{same}  % don't use monospace font for urls
\usepackage{natbib}
\bibliographystyle{plainnat}
\usepackage{color}
\usepackage{fancyvrb}
\newcommand{\VerbBar}{|}
\newcommand{\VERB}{\Verb[commandchars=\\\{\}]}
\DefineVerbatimEnvironment{Highlighting}{Verbatim}{commandchars=\\\{\}}
% Add ',fontsize=\small' for more characters per line
\usepackage{framed}
\definecolor{shadecolor}{RGB}{248,248,248}
\newenvironment{Shaded}{\begin{snugshade}}{\end{snugshade}}
\newcommand{\KeywordTok}[1]{\textcolor[rgb]{0.13,0.29,0.53}{\textbf{#1}}}
\newcommand{\DataTypeTok}[1]{\textcolor[rgb]{0.13,0.29,0.53}{#1}}
\newcommand{\DecValTok}[1]{\textcolor[rgb]{0.00,0.00,0.81}{#1}}
\newcommand{\BaseNTok}[1]{\textcolor[rgb]{0.00,0.00,0.81}{#1}}
\newcommand{\FloatTok}[1]{\textcolor[rgb]{0.00,0.00,0.81}{#1}}
\newcommand{\ConstantTok}[1]{\textcolor[rgb]{0.00,0.00,0.00}{#1}}
\newcommand{\CharTok}[1]{\textcolor[rgb]{0.31,0.60,0.02}{#1}}
\newcommand{\SpecialCharTok}[1]{\textcolor[rgb]{0.00,0.00,0.00}{#1}}
\newcommand{\StringTok}[1]{\textcolor[rgb]{0.31,0.60,0.02}{#1}}
\newcommand{\VerbatimStringTok}[1]{\textcolor[rgb]{0.31,0.60,0.02}{#1}}
\newcommand{\SpecialStringTok}[1]{\textcolor[rgb]{0.31,0.60,0.02}{#1}}
\newcommand{\ImportTok}[1]{#1}
\newcommand{\CommentTok}[1]{\textcolor[rgb]{0.56,0.35,0.01}{\textit{#1}}}
\newcommand{\DocumentationTok}[1]{\textcolor[rgb]{0.56,0.35,0.01}{\textbf{\textit{#1}}}}
\newcommand{\AnnotationTok}[1]{\textcolor[rgb]{0.56,0.35,0.01}{\textbf{\textit{#1}}}}
\newcommand{\CommentVarTok}[1]{\textcolor[rgb]{0.56,0.35,0.01}{\textbf{\textit{#1}}}}
\newcommand{\OtherTok}[1]{\textcolor[rgb]{0.56,0.35,0.01}{#1}}
\newcommand{\FunctionTok}[1]{\textcolor[rgb]{0.00,0.00,0.00}{#1}}
\newcommand{\VariableTok}[1]{\textcolor[rgb]{0.00,0.00,0.00}{#1}}
\newcommand{\ControlFlowTok}[1]{\textcolor[rgb]{0.13,0.29,0.53}{\textbf{#1}}}
\newcommand{\OperatorTok}[1]{\textcolor[rgb]{0.81,0.36,0.00}{\textbf{#1}}}
\newcommand{\BuiltInTok}[1]{#1}
\newcommand{\ExtensionTok}[1]{#1}
\newcommand{\PreprocessorTok}[1]{\textcolor[rgb]{0.56,0.35,0.01}{\textit{#1}}}
\newcommand{\AttributeTok}[1]{\textcolor[rgb]{0.77,0.63,0.00}{#1}}
\newcommand{\RegionMarkerTok}[1]{#1}
\newcommand{\InformationTok}[1]{\textcolor[rgb]{0.56,0.35,0.01}{\textbf{\textit{#1}}}}
\newcommand{\WarningTok}[1]{\textcolor[rgb]{0.56,0.35,0.01}{\textbf{\textit{#1}}}}
\newcommand{\AlertTok}[1]{\textcolor[rgb]{0.94,0.16,0.16}{#1}}
\newcommand{\ErrorTok}[1]{\textcolor[rgb]{0.64,0.00,0.00}{\textbf{#1}}}
\newcommand{\NormalTok}[1]{#1}
\usepackage{longtable,booktabs}
\usepackage{graphicx,grffile}
\makeatletter
\def\maxwidth{\ifdim\Gin@nat@width>\linewidth\linewidth\else\Gin@nat@width\fi}
\def\maxheight{\ifdim\Gin@nat@height>\textheight\textheight\else\Gin@nat@height\fi}
\makeatother
% Scale images if necessary, so that they will not overflow the page
% margins by default, and it is still possible to overwrite the defaults
% using explicit options in \includegraphics[width, height, ...]{}
\setkeys{Gin}{width=\maxwidth,height=\maxheight,keepaspectratio}
\IfFileExists{parskip.sty}{%
\usepackage{parskip}
}{% else
\setlength{\parindent}{0pt}
\setlength{\parskip}{6pt plus 2pt minus 1pt}
}
\setlength{\emergencystretch}{3em}  % prevent overfull lines
\providecommand{\tightlist}{%
  \setlength{\itemsep}{0pt}\setlength{\parskip}{0pt}}
\setcounter{secnumdepth}{5}
% Redefines (sub)paragraphs to behave more like sections
\ifx\paragraph\undefined\else
\let\oldparagraph\paragraph
\renewcommand{\paragraph}[1]{\oldparagraph{#1}\mbox{}}
\fi
\ifx\subparagraph\undefined\else
\let\oldsubparagraph\subparagraph
\renewcommand{\subparagraph}[1]{\oldsubparagraph{#1}\mbox{}}
\fi

%%% Use protect on footnotes to avoid problems with footnotes in titles
\let\rmarkdownfootnote\footnote%
\def\footnote{\protect\rmarkdownfootnote}

%%% Change title format to be more compact
\usepackage{titling}

% Create subtitle command for use in maketitle
\newcommand{\subtitle}[1]{
  \posttitle{
    \begin{center}\large#1\end{center}
    }
}

\setlength{\droptitle}{-2em}

  \title{ECON 6120I Topics in Empirical Industrial Organization}
    \pretitle{\vspace{\droptitle}\centering\huge}
  \posttitle{\par}
    \author{Kohei Kawaguchi}
    \preauthor{\centering\large\emph}
  \postauthor{\par}
      \predate{\centering\large\emph}
  \postdate{\par}
    \date{Last updated: 2019-02-25}

\usepackage{booktabs}

\begin{document}
\maketitle

{
\setcounter{tocdepth}{1}
\tableofcontents
}
\chapter{Syllabus}\label{syllabus}

\section{Instructor Information}\label{instructor-information}

\begin{itemize}
\tightlist
\item
  Instructor:

  \begin{itemize}
  \tightlist
  \item
    Name: Kohei Kawaguchi.
  \item
    Office: LSK6070, Monday 11:00-12:00.
  \end{itemize}
\item
  All questions related to the class have to be publicly asked on the
  discussion board of canvas rather than being privately asked in
  e-mail. The instructor usually does not reply in the evening,
  weekends, and holidays.
\end{itemize}

\section{General Information}\label{general-information}

\subsection{Class Time}\label{class-time}

\begin{itemize}
\tightlist
\item
  Date: Monday.
\item
  Time: 13:30-17:20.
\item
  Venue: CYTG001.
\end{itemize}

\subsection{Description}\label{description}

\begin{itemize}
\item
  This is a PhD-level course for empirical industrial organization. This
  course covers various econometric methods used in industrial
  organization that is often referred to as the structural estimation
  approach. These methods have been gradually developed since 1980s in
  parallel with the modernization of industrial organization based on
  the game theory and now widely applied in antitrust policy, business
  strategy, and neighboring fields such as labor economics and
  international economics.
\item
  This course presumes a good understanding of PhD-level microeconomics
  and microeconometrics. Participants are expected to understand at
  least UG-level industrial organization. This course requires
  participants to write programs mostly in R and sometimes in C++ to
  implement various econometric methods. In particular, all assignments
  will involve such a non-trivial programming task. Even though the
  understanding of these programming languages is not a prerequisite, a
  sharp learning curve will be required. Some experience in other
  programming languages will help. Audit without a credit is not
  admitted for students.
\end{itemize}

\subsection{Expectation and Goals}\label{expectation-and-goals}

\begin{itemize}
\tightlist
\item
  The goal of this course is to learn and practice econometric methods
  for empirical industrial organization. The lecture covers the
  econometric methods that have been developed between 80s and 00s to
  estimate primitive parameters governing imperfect competition among
  firms, such as production and cost function estimation, demand
  function estimation, merger simulation, entry and exit analysis, and
  dynamic decision models. The lecture also covers various new methods
  to recover model primitives in certain mechanisms such as auction,
  matching, network, and bargaining. The emphasis is put on the former
  group of methods, because they are the basis for other new methods.
  Participants will not only understand the theoretical background of
  the methods but also become able to implement these methods from
  scratches by writing their own programs. I will briefly discuss the
  latter class of new methods through reading recent papers. The
  participants will become able to understand and use these new methods.
\end{itemize}

\section{Required Environment}\label{required-environment}

\begin{itemize}
\tightlist
\item
  Participants should bring their laptop to the class. We have enough
  extension codes for students. The laptop should have sufficient RAM
  (at least \(\ge\) 8GB, \(\ge\) 16GB is recommended) and CPU power (at
  least Core i5 grade, Core i7 grade is recommended). Participants are
  fully responsible for their hardware issues. Operating System can be
  arbitrary. The instructor mainly uses OSX High Siera with iMac (Retina
  5K, 27-inch, Late 2015) and Macbook Pro (Retina, 15-inch, Early 2017).
\item
  Please install the following software by the first lecture. Technical
  issues related to the installment should be resolved by yourself,
  because it depends on your local environment. If you had an error,
  copy and paste the error message on a search engine, and find a
  solution. This solves 99.9\% of the problems.

  \begin{itemize}
  \tightlist
  \item
    R: \url{https://www.r-project.org/}
  \item
    RStudio: \url{https://www.rstudio.com/}
  \item
    LaTeX:

    \begin{itemize}
    \tightlist
    \item
      MixTex \url{https://miktex.org/}
    \item
      TeXLive \url{https://www.tug.org/texlive/}
    \item
      MacTeX \url{http://www.tug.org/mactex/}
    \end{itemize}
  \end{itemize}
\end{itemize}

\section{Evaluation}\label{evaluation}

\begin{itemize}
\tightlist
\item
  Assignments (80): In total 8 homework are assigned. Each homework
  involves the implementation of the methods covered in the class. Each
  homework has 10 points. The working hour for each homework will be
  around 10-20 hours.
\item
  Participation (10): Every time a participant asks a question in the
  class, after the class, during the office hour, or in the canvas. the
  participant gets one point, up to 10 points. The participant who asked
  the question writes the name, ID number, his/her question, and my
  answer in a discussion board on the course website to claim a point.
\item
  Referee report (10): Toward the end of the semester, a paper in
  industrial organization is randomly assigned to each participant. Each
  participant writes a critical referee report of the assigned paper in
  A4 2 pages that consists of the summary, contribution, strong and weak
  points of the paper.
\item
  Grading is based on the absolute scores: A+ with more than 80 points,
  A with more than 70 points, A- with more than 60 points, B+ with more
  than 50 points, B with more than 40 points, B- with more than 30
  points and C otherwise.
\end{itemize}

\section{Academic Integrity}\label{academic-integrity}

Without academic integrity, there is no serious learning. Thus you are
expected to hold the highest standard of academic integrity in the
course. You are encouraged to study and do homework in groups. However,
no cheating, plagiarism will be tolerated. Anyone caught cheating,
plagiarism will fail the course. Please make sure adhere to the HKUST
Academic Honor Code at all time (see
\url{http://www.ust.hk/vpaao/integrity/}).

\section{Schedule}\label{schedule}

\begin{itemize}
\tightlist
\item
  Introduction to structural estimation, R and RStudio
\item
  Production function estimation I
\item
  Production function estimation II
\item
  Demand function estimation I
\item
  Demand function estimation II
\item
  Merger Analysis
\item
  Entry and Exit
\item
  Single-Agent Dynamics I
\item
  Single-Agent Dynamics II, I change date due to my business trip
\item
  Dynamic Game I
\item
  Dynamic Game II
\item
  Auction
\item
  Other Mechanisms
\end{itemize}

\section{Course Materials}\label{course-materials}

\subsection{R and RStudio}\label{r-and-rstudio}

\begin{itemize}
\tightlist
\item
  Grolemund, G., 2014, Hands-On Programming with R, O'Reilly.

  \begin{itemize}
  \tightlist
  \item
    Free online version is available:
    \url{https://rstudio-education.github.io/hopr/}.
  \end{itemize}
\item
  Wickham, H., \& Grolemund, G., 2017, R for Data Science, O'Reilly.

  \begin{itemize}
  \tightlist
  \item
    Free online version is available: \url{https://r4ds.had.co.nz/}.
  \end{itemize}
\item
  Boswell, D., \& Foucher, T., 2011, The Art of Readable Code: Simple
  and Practical Techniques for Writing Better Code, O'Reilly.
\end{itemize}

\subsection{Handbook Chapters}\label{handbook-chapters}

\begin{itemize}
\tightlist
\item
  Ackerberg, D., Benkard, C., Berry, S., \& Pakes, A. (2007).
  ``Econometric tools for analyzing market outcomes''. Handbook of
  econometrics, 6, 4171-4276.
\item
  Athey, S., \& Haile, P. A. (2007). ``Nonparametric approaches to
  auctions''. Handbook of Econometrics, 6, 3847-3965.
\item
  Berry, S., \& Reiss, P. (2007). ``Empirical models of entry and market
  structure''. Handbook of Industrial Organization, 3, 1845-1886.
\item
  Bresnahan, T. F. (1989). ``Empirical studies of industries with market
  power''. Handbook of Industrial Organization, 2, 1011-1057.
\item
  Hendricks, K., \& Porter, R. H. (2007). ``An empirical perspective on
  auctions''. Handbook of Industrial Organization, 3, 2073-2143.
\item
  Matzkin, R. L. (2007). ``Nonparametric identification''. Handbook of
  Econometrics, 6, 5307-5368.
\item
  Newey, W. K., \& McFadden, D. (1994). ``Large sample estimation and
  hypothesis testing''. Handbook of Econometrics, 4, 2111-2245.
\item
  Reiss, P. C., \& Wolak, F. A. (2007). ``Structural econometric
  modeling: Rationales and examples from industrial organization''.
  Handbook of Econometrics, 6, 4277-4415.
\end{itemize}

\subsection{Books}\label{books}

\begin{itemize}
\tightlist
\item
  Train, K. E. (2009). Discrete Choice Methods with Simulation,
  Cambridge university press.
\item
  Davis, P., \& Garces, E. (2010). Quantitative Techniques for
  Competition and Antitrust Analysis, Princeton University Press.
\item
  Tirole, J. (1988). The Theory of Industrial Organization, The MIT
  Press.
\end{itemize}

\subsection{Papers}\label{papers}

\begin{itemize}
\tightlist
\item
  The list of important papers are occasionally given during the course.
\end{itemize}

\chapter{Introduction}\label{intro}

\section{Structural Estimation and Counterfactual
Analysis}\label{structural-estimation-and-counterfactual-analysis}

\subsection{Example}\label{example}

\begin{itemize}
\item
  \citet{Igami2017} ``Estimating the Innovator's Dilemma: Structural
  Analysis of Creative Destruction in the Hard Disk Drive Industry,
  1981-1998''.
\item
  \textbf{Question}:
\item
  Does ``Innovator's Dilemma'' \citep{Christensen1997} or the delay of
  innovation among incumbents exist?
\item
  Christensen argued that old winners tend to lag behind entrants even
  when introducing a new technology is not too difficult, with a case
  study of the HDD industry.

  \begin{itemize}
  \tightlist
  \item
    Apple's smartphone vs.~Nokia's feature phones.
  \item
    Amazon vs.~Borders.
  \item
    Kodak's digital camera.
  \end{itemize}
\item
  If it exists, what is the reason for that?
\item
  How do we empirically answer this question?
\end{itemize}

\begin{figure}

{\centering \includegraphics[width=0.8\linewidth]{figuretable/Igam2017Fig1} 

}

\caption{Figure 1 of Igam (2017)}\label{fig:unnamed-chunk-2}
\end{figure}

\begin{itemize}
\item
  \textbf{Hypotheses}:
\item
  Identify potentially competing hypotheses to explain the phenomenon.

  \begin{enumerate}
  \def\labelenumi{\arabic{enumi}.}
  \tightlist
  \item
    Cannibalization: Because of cannibalization, the benefits of
    introducing a new product are smaller for incumbents than for
    entrants.
  \item
    Different costs: The incumbents may have higher costs for innovation
    due to the organizational inertia, but at the same time they may
    have some cost advantage due to accumulated R\&D and better
    financial access.
  \item
    Preemption: The incumbents have additional incentive for innovation
    to preempt potential rivals.
  \item
    Institutional environment: The impacts of the three components
    differ across different institutional contexts such as the rules
    governing patents and market size.
  \end{enumerate}
\item
  Casual empiricists pick up their favorite factors to make up a story.
\item
  Serious empiricists should try to separate the contributions of each
  factor from data.
\item
  To do so, the author develops an economic model that explicitly
  incorporates the above mentioned factors, while keeping the model
  parameters flexible enough to let the data tell the sign and size of
  the effects of each factor on innovation.
\item
  \textbf{Economic model}:
\item
  The time is discrete with finite horizon \(t = 1, \cdots, T\).
\item
  In each year, there is a finite number of firms indexed by \(i\).
\item
  Each firm is in one of the technological states:

  \begin{equation}
  s_{it} \in \{\text{old only, both, new only, potential entrant}\},
  \end{equation}

  where the first two states are for incumbents (stick to the old
  technology or start using the new technology) and the last two states
  are for actual and potential entrants (enter with the new technology
  or stay outside the market).
\item
  In each year:

  \begin{itemize}
  \tightlist
  \item
    Pre-innovation incumbent (\(s_{it} =\) old): exit or innovate by
    paying a sunk cost \(\kappa^{inc}\) (to be \(s_{i, t + 1} =\) both).
  \item
    Post-innovation incumbent (\(s_{it} =\) both): exit or stay to be
    both.
  \item
    Potential entrant (\(s_{it} =\) potential entrant): give up entry or
    enter with the new technology by paying a sunk cost \(\kappa^{net}\)
    (to be \(s_{i, t + 1} =\) new).
  \item
    Actual entrant (\(s_{it} =\) new): exit or stay to be new.
  \end{itemize}
\item
  Given the industry state \(s_t = \{s_{it}\}_i\), the product market
  competition opens and the profit of firm \(i\),
  \(\pi_t(s_{it}, s_{-it})\), is realized for each active firm.
\item
  As the product market competition closes:

  \begin{itemize}
  \tightlist
  \item
    Pre-innovation incumbents draw private cost shocks and make
    decisions: \(a_t^{pre}\).
  \item
    Observing this, post-innovation incumbents draw private cost shocks
    and make decisions: \(a_t^{post}\).
  \item
    Observing this, actual entrants draw private cost shocks and make
    decisions: \(a_t^{act}\).
  \item
    Observing this, potential entrants draw private cost shocks and make
    decisions: \(a_t^{pot}\).
  \end{itemize}
\item
  This is a dynamic game. The equilibrium is defined by the concept of
  \textbf{Markov-perfect equilibrium} \citep{Maskin1988}.
\item
  The representation of the competing theories in the model:

  \begin{itemize}
  \tightlist
  \item
    The existence of cannibalization is represented by the assumption
    that an incumbent maximizes the joint profits of old and new
    technology products.
  \item
    The size of cannibalization is captured by the shape of profit
    function.
  \item
    The difference in the cost of innovation is captured by the
    difference in the sunk costs of innovation.
  \item
    The preemptive incentive for incumbents are embodied in the dynamic
    optimization problem for each incumbent.
  \end{itemize}
\item
  \textbf{Econometric model}:
\item
  The author then turns the economic model into an econometric model.
\item
  This amounts to specify which part of the economic model is
  observed/known and which part is unobserved/unknown.
\item
  The author collects the data set of the HDD industry during 1977-99.
\item
  Based on the data, the author specify the identities of active firms
  and their products and the technologies embodied in the products in
  each year to code their \textbf{state variables}.
\item
  Moreover, by tracking the change in the state, the author code their
  \textbf{action variables}.
\item
  Thus, the state and action variables, \(s_t\) and \(a_t\). These are
  the \textbf{observables}.
\item
  The author does not observe:

  \begin{itemize}
  \tightlist
  \item
    Profit function \(\pi_t(\cdot)\).
  \item
    Sunk cost of innovation for pre-innovation incumbents
    \(\kappa^{inc}\).
  \item
    Sunk cost of entry for potential entrants \(\kappa^{net}\).
  \item
    Private cost shocks.
  \end{itemize}
\item
  These are the \textbf{unobservables}.
\item
  Among the unobservables, the profit function and sunk costs are the
  \textbf{parameter of interets} and the private cost shocks are
  \textbf{nuissance parameters} in the sense only the knowledge about
  the distribution of the latter is demanded.
\item
  \textbf{Identification}:
\item
  Can we infer the unobservables from the observables and the
  restrictions on the distribution of observable by the economic theory?
\item
  The profit function is identified from estimating the demand function
  for each firm's product, and estimating the cost function for each
  firm from using their price setting behavior.
\item
  The sunk costs of innovation are identified from the conditional
  probability of innovation across various states. If the cost is low,
  the probability should be high.
\item
  \textbf{Estimation}:
\item
  The identification established that in principle we can uncover the
  parameters of interests from observables under the restrictions of
  economic theory.
\item
  Finally, we apply a statistical method to the econometric model and
  infer the parameters of interest.
\item
  \textbf{Counterfactual analysis}:
\item
  If we can uncover the parameters of interest, we can conduct
  \textbf{comparative statics}: study the change in the endogenous
  variables when the exogenous variables including the model parameters
  are set different. In the current framework, this exercise is often
  called the \textbf{counterfactual analysis}.
\item
  What if there was no cannibalization?:

  \begin{itemize}
  \tightlist
  \item
    An incumbents separately maximizes the profit from old technology
    and new technology instead of jointly maximizing the profits. Solve
    the model under this new assumption everything else being equal.
  \item
    Free of cannibalization concerns, 8.95 incumbents start producing
    new HDDs in the first 10 years, compared with 6.30 in the baseline.
  \item
    The cumulative numbers of innovators among incumbents and entrants
    differ only by 2.8 compared with 6.45 in the baseline.
  \item
    Thus cannibalization can explain a significant part of the
    incumbent-entrant innovation gap.
  \end{itemize}
\item
  What if there was no preemption?:

  \begin{itemize}
  \tightlist
  \item
    A potential entrant ignores the incumbents' innovations upon making
    entry decisions.
  \item
    Without the preemptive motives, only 6.02 incumbents would innovate
    in the first 10 7ears, compared with 6.30 in the baseline.
  \item
    The cumulative incumbent-entrant innovation gap widen to 8.91
    compared with 6.45 in the baseline.
  \end{itemize}
\item
  The sunk cost of entry is smaller for incumbents than for entrants in
  the baseline.
\item
  \textbf{Interpretations and policy/managerial implication}:
\item
  Despite the cost advantage and the preemptive motives, the speed of
  innovation is slower among incumbents due to the strong
  cannibalization effect.
\item
  Incumbents that attempt to avoid the ``innovator's dilemma'' should
  separate the decision makings between old and new sections inside the
  organization so that it can avoid the concern for cannibalization.
\end{itemize}

\subsection{Recap}\label{recap}

\begin{itemize}
\tightlist
\item
  The structural approach in empirical industrial organization consists
  of the following components:
\end{itemize}

\begin{enumerate}
\def\labelenumi{\arabic{enumi}.}
\tightlist
\item
  Research question.
\item
  Competing hypotheses.
\item
  Economic model.
\item
  Econometric model
\item
  Identification.
\item
  Data collection.
\item
  Data cleaning.
\item
  Estimation.
\item
  Counterfactual analysis.
\item
  Coding.
\item
  Interpretations and policy/managerial implications.
\end{enumerate}

\begin{itemize}
\tightlist
\item
  The goal of this course is to be familiar with the standard
  methodology to complete this process.
\item
  The methodology covered in this class is mostly developed to analyze
  the standard framework to dynamic or oligopoly competition.
\item
  The policy implications are centered around competition policies.
\item
  But the basic idea can be extend to different class of situations such
  as auction, matching, voting, contract, marketing, and so on.
\item
  Note that the depth of the research question and the relevance of the
  policy/managerial implications are the most important part of the
  research.
\item
  Focusing on the methodology in this class is to minimize the time to
  allocate to less important issues and maximize the attention and time
  to the most valuable part in the future research.
\item
  Given a research question, what kind of data is necessary to answer
  the question?
\item
  Given data, what kind of research questions can you address? Which
  question can be credibly answered? Which question can be an
  over-stretch?
\item
  Given a research question and data, what is the best way to answer the
  question? What type of problem can you avoid using the method? What is
  the limitation of your approach? How will you defend the possible
  referee comments?
\item
  Given a result, what kinds of interpretation can you credibly derive?
  What kinds of interpretation can be contested by potential opponents?
  What kinds of contribution can you claim?
\item
  To address these issues is \textbf{necessary} to publish a paper and
  it is \textbf{necessary} to be familiar with the methodology to do so.
\end{itemize}

\subsection{Historical Remark}\label{historical-remark}

\begin{itemize}
\tightlist
\item
  The words \textbf{reduced-form} and \textbf{structural-form} date back
  to the literature of estimation of simultaneous equations in
  macroeconomics \citep{Hsiao1983}.
\item
  Let \(y\) be the vector of observed endogenous variables, \(x\) be the
  vector of observed exogenous variables, and \(\epsilon\) be the vector
  of unobserved exogenous variables.
\item
  The equilibrium condition for \(y\) on \(x\) and \(\epsilon\) is often
  written as:

  \begin{equation}
  Ay + Bx = \Sigma \epsilon. \label{eq:structuralform}
  \end{equation}
\item
  These equations \textbf{implicitly} determine the vector of endogenous
  variables \(y\) .
\item
  If \(A\) is invertible, we can solve the equations for \(y\) to
  obtain:

  \begin{equation}
  y = - A^{-1} B x + A^{-1} \Sigma \epsilon. \label{eq:reducedform}
  \end{equation}
\item
  These equations \textbf{explicitly} determine the vector of endogenous
  variables \(y\).
\item
  Equation \eqref{eq:structuralform} is the \textbf{structural-form} and
  \eqref{eq:reducedform} is the \textbf{reduced-form}.
\item
  If \(y\) and \(x\) are observed and \(x\) is of full column rank, then
  \(A^{-1}B\) and \(A^{-1} \Sigma A^{-1}\) will be estimated by
  regression for \eqref{eq:reducedform}. But this does not mean that
  \(A, B\) and \(\Sigma\) are separately estimated.
\item
  This was the traditional identification problems.
\item
  Thus, reduced-form does not mean either of:

  \begin{itemize}
  \tightlist
  \item
    Regression analysis;
  \item
    Statistical analysis free from economic assumptions.
  \end{itemize}
\item
  Recent development in this line of literature of identification is
  found in \citet{Matzkin2007}.
\item
  In econometrics, the idea of imposing restrictions from economic
  theories seems to have been formalized by the work of
  \citet{Manski1994a} and \citet{Matzkin1994b}.
\end{itemize}

\section{Setting Up The Environment}\label{setting-up-the-environment}

\begin{itemize}
\tightlist
\item
  Assume that R, RStudio and LaTex are all installed in the local
  computer.
\end{itemize}

\subsection{RStudio Project}\label{rstudio-project}

\begin{itemize}
\tightlist
\item
  The assignments should be conducted inside a project folder for this
  course.
\item
  \texttt{File\ \textgreater{}\ New\ Project...\textgreater{}\ New\ Directory\ \textgreater{}\ New\ Directory\ \textgreater{}\ R\ Package\ using\ RcppEigen}.
\item
  Name the directory \texttt{ECON6120I} and place in your favorite
  location.
\item
  You can open this project from the upper right menu of RStudio or by
  double clicking the \texttt{ECON6120I.Rproj} file in the
  \texttt{ECON6120I} directory.
\item
  This navigates you to the root directory of the project.
\item
  In the root directory, make folders named:

  \begin{itemize}
  \tightlist
  \item
    \texttt{assignment}.
  \item
    \texttt{input}.
  \item
    \texttt{output}.
  \item
    \texttt{figuretable}.
  \end{itemize}
\item
  We will store R functions in \texttt{R} folder, C/C++ functions in
  \texttt{src} folder, and data in \texttt{input} folder, data generated
  from the code in \texttt{output}, and figures and tables in
  \texttt{figurtable} folder.
\item
  Open \texttt{src/Makevars} and erase the content. Then, write:
  \texttt{PKG\_CPPFLAGS\ =\ -w\ -std=c++11\ -O3}
\item
  Open \texttt{src/Makevars.win} and erase the content. Then, write:
  \texttt{PKG\_CPPFLAGS\ =\ -w\ -std=c++11}
\end{itemize}

\subsection{Basic Programming in R}\label{basic-programming-in-r}

\begin{itemize}
\tightlist
\item
  \texttt{File\ \textgreater{}\ New\ File\ \textgreater{}\ R\ Script} to
  open \texttt{Untitled} file.
\item
  \texttt{Ctrl\ (Cmd)\ +\ S} to save it with \texttt{test.R} in
  \texttt{assignment} folder.
\item
  In the console, type and push enter:
\end{itemize}

\begin{Shaded}
\begin{Highlighting}[]
\DecValTok{1} \OperatorTok{+}\StringTok{ }\DecValTok{1}
\end{Highlighting}
\end{Shaded}

\begin{verbatim}
## [1] 2
\end{verbatim}

\begin{Shaded}
\begin{Highlighting}[]
\DecValTok{100}\OperatorTok{:}\DecValTok{130}
\end{Highlighting}
\end{Shaded}

\begin{verbatim}
##  [1] 100 101 102 103 104 105 106 107 108 109 110 111 112 113 114 115 116
## [18] 117 118 119 120 121 122 123 124 125 126 127 128 129 130
\end{verbatim}

\begin{itemize}
\tightlist
\item
  This is the interactive way of using R functionalities.
\item
  In \texttt{test.R}, write:
\end{itemize}

\begin{Shaded}
\begin{Highlighting}[]
\DecValTok{1} \OperatorTok{+}\StringTok{ }\DecValTok{1}
\end{Highlighting}
\end{Shaded}

\begin{itemize}
\item
  Then, save the file and push \texttt{Run}.
\item
  Alternatively, place the mouse over the \texttt{1\ +\ 1} line in
  \texttt{test.R} file.
\item
  Then, \texttt{Ctrl\ (Cmd)\ +\ Enter} to run the line.
\item
  In this way, we can write procedures in the file and send to the
  console to run.
\item
  There are functions to conduct basic calculations:
\end{itemize}

\begin{Shaded}
\begin{Highlighting}[]
\DecValTok{1} \OperatorTok{+}\StringTok{ }\DecValTok{2}
\end{Highlighting}
\end{Shaded}

\begin{verbatim}
## [1] 3
\end{verbatim}

\begin{Shaded}
\begin{Highlighting}[]
\DecValTok{2} \OperatorTok{*}\StringTok{ }\DecValTok{3}
\end{Highlighting}
\end{Shaded}

\begin{verbatim}
## [1] 6
\end{verbatim}

\begin{Shaded}
\begin{Highlighting}[]
\DecValTok{4} \OperatorTok{-}\StringTok{ }\DecValTok{1}
\end{Highlighting}
\end{Shaded}

\begin{verbatim}
## [1] 3
\end{verbatim}

\begin{Shaded}
\begin{Highlighting}[]
\DecValTok{6} \OperatorTok{/}\StringTok{ }\DecValTok{2}
\end{Highlighting}
\end{Shaded}

\begin{verbatim}
## [1] 3
\end{verbatim}

\begin{Shaded}
\begin{Highlighting}[]
\DecValTok{2}\OperatorTok{^}\DecValTok{3}
\end{Highlighting}
\end{Shaded}

\begin{verbatim}
## [1] 8
\end{verbatim}

\begin{itemize}
\tightlist
\item
  We can define objects and assign values to them.
\end{itemize}

\begin{Shaded}
\begin{Highlighting}[]
\NormalTok{a <-}\StringTok{ }\DecValTok{1}
\NormalTok{a}
\end{Highlighting}
\end{Shaded}

\begin{verbatim}
## [1] 1
\end{verbatim}

\begin{Shaded}
\begin{Highlighting}[]
\NormalTok{a }\OperatorTok{+}\StringTok{ }\DecValTok{2}
\end{Highlighting}
\end{Shaded}

\begin{verbatim}
## [1] 3
\end{verbatim}

\begin{itemize}
\tightlist
\item
  In addition to scalar object, we can define a vector by:
\end{itemize}

\begin{Shaded}
\begin{Highlighting}[]
\DecValTok{2}\OperatorTok{:}\DecValTok{10}
\end{Highlighting}
\end{Shaded}

\begin{verbatim}
## [1]  2  3  4  5  6  7  8  9 10
\end{verbatim}

\begin{Shaded}
\begin{Highlighting}[]
\DecValTok{3}\OperatorTok{:}\DecValTok{20}
\end{Highlighting}
\end{Shaded}

\begin{verbatim}
##  [1]  3  4  5  6  7  8  9 10 11 12 13 14 15 16 17 18 19 20
\end{verbatim}

\begin{Shaded}
\begin{Highlighting}[]
\KeywordTok{c}\NormalTok{(}\DecValTok{2}\NormalTok{, }\DecValTok{3}\NormalTok{, }\DecValTok{5}\NormalTok{, }\DecValTok{9}\NormalTok{, }\DecValTok{10}\NormalTok{)}
\end{Highlighting}
\end{Shaded}

\begin{verbatim}
## [1]  2  3  5  9 10
\end{verbatim}

\begin{Shaded}
\begin{Highlighting}[]
\KeywordTok{seq}\NormalTok{(}\DecValTok{1}\NormalTok{, }\DecValTok{10}\NormalTok{, }\DecValTok{2}\NormalTok{)}
\end{Highlighting}
\end{Shaded}

\begin{verbatim}
## [1] 1 3 5 7 9
\end{verbatim}

\begin{itemize}
\tightlist
\item
  \texttt{seq} is a function with initial value, end values, and the
  increment value.
\item
  By typing \texttt{seq} in the \texttt{help}, we can read the manual
  page of the function.
\item
  \texttt{seq\ \{base\}} means that this function is named \texttt{seq}
  and is contained in the library called \texttt{base}.
\item
  Some libraries are automatically called when the R is launched, but
  some are not.
\item
  Some libraries are even not installed.
\item
  We can install a library from a repository called \texttt{CRAN}.
\end{itemize}

\begin{Shaded}
\begin{Highlighting}[]
\KeywordTok{install.packages}\NormalTok{(}\StringTok{"ggplot2"}\NormalTok{)}
\end{Highlighting}
\end{Shaded}

\begin{itemize}
\tightlist
\item
  To use the package, we have to load by:
\end{itemize}

\begin{Shaded}
\begin{Highlighting}[]
\KeywordTok{library}\NormalTok{(ggplot2)}
\end{Highlighting}
\end{Shaded}

\begin{itemize}
\tightlist
\item
  Use \texttt{qplot} function in \texttt{ggplot2} library to draw a
  scatter plot.
\end{itemize}

\begin{Shaded}
\begin{Highlighting}[]
\NormalTok{x <-}\StringTok{ }\KeywordTok{c}\NormalTok{(}\OperatorTok{-}\DecValTok{1}\NormalTok{, }\OperatorTok{-}\FloatTok{0.8}\NormalTok{, }\OperatorTok{-}\FloatTok{0.6}\NormalTok{, }\OperatorTok{-}\FloatTok{0.4}\NormalTok{, }\OperatorTok{-}\FloatTok{0.2}\NormalTok{, }\DecValTok{0}\NormalTok{, }\FloatTok{0.2}\NormalTok{, }\FloatTok{0.4}\NormalTok{, }\FloatTok{0.6}\NormalTok{, }\FloatTok{0.7}\NormalTok{, }\DecValTok{1}\NormalTok{)}
\NormalTok{y <-}\StringTok{ }\NormalTok{x}\OperatorTok{^}\DecValTok{3}
\KeywordTok{qplot}\NormalTok{(x, y)}
\end{Highlighting}
\end{Shaded}

\includegraphics{lecture_files/figure-latex/unnamed-chunk-15-1.pdf} -
Instead of loading a package by \texttt{library}, you can directly call
it as:

\begin{Shaded}
\begin{Highlighting}[]
\NormalTok{ggplot2}\OperatorTok{::}\KeywordTok{qplot}\NormalTok{(x, y)}
\end{Highlighting}
\end{Shaded}

\includegraphics{lecture_files/figure-latex/unnamed-chunk-16-1.pdf}

\begin{itemize}
\tightlist
\item
  We can write own functions.
\end{itemize}

\begin{Shaded}
\begin{Highlighting}[]
\NormalTok{roll <-}\StringTok{ }\ControlFlowTok{function}\NormalTok{(n) \{}
\NormalTok{  die <-}\StringTok{ }\DecValTok{1}\OperatorTok{:}\DecValTok{6}
\NormalTok{  dice <-}\StringTok{ }\KeywordTok{sample}\NormalTok{(die, }\DataTypeTok{size =}\NormalTok{ n, }\DataTypeTok{replace =} \OtherTok{TRUE}\NormalTok{)}
\NormalTok{  y <-}\StringTok{ }\KeywordTok{sum}\NormalTok{(dice)}
  \KeywordTok{return}\NormalTok{(y)}
\NormalTok{\}}
\KeywordTok{roll}\NormalTok{(}\DecValTok{1}\NormalTok{)}
\end{Highlighting}
\end{Shaded}

\begin{verbatim}
## [1] 5
\end{verbatim}

\begin{Shaded}
\begin{Highlighting}[]
\KeywordTok{roll}\NormalTok{(}\DecValTok{2}\NormalTok{)}
\end{Highlighting}
\end{Shaded}

\begin{verbatim}
## [1] 9
\end{verbatim}

\begin{Shaded}
\begin{Highlighting}[]
\KeywordTok{roll}\NormalTok{(}\DecValTok{10}\NormalTok{)}
\end{Highlighting}
\end{Shaded}

\begin{verbatim}
## [1] 27
\end{verbatim}

\begin{Shaded}
\begin{Highlighting}[]
\KeywordTok{roll}\NormalTok{(}\DecValTok{10}\NormalTok{)}
\end{Highlighting}
\end{Shaded}

\begin{verbatim}
## [1] 41
\end{verbatim}

\begin{itemize}
\tightlist
\item
  We can \texttt{set.seed} to obtain the same realization of random
  variables.
\end{itemize}

\begin{Shaded}
\begin{Highlighting}[]
\KeywordTok{set.seed}\NormalTok{(}\DecValTok{1}\NormalTok{)}
\KeywordTok{roll}\NormalTok{(}\DecValTok{10}\NormalTok{)}
\end{Highlighting}
\end{Shaded}

\begin{verbatim}
## [1] 38
\end{verbatim}

\begin{Shaded}
\begin{Highlighting}[]
\KeywordTok{set.seed}\NormalTok{(}\DecValTok{1}\NormalTok{)}
\KeywordTok{roll}\NormalTok{(}\DecValTok{10}\NormalTok{)}
\end{Highlighting}
\end{Shaded}

\begin{verbatim}
## [1] 38
\end{verbatim}

\begin{itemize}
\tightlist
\item
  When a variable used in a function is not given as its argument, the
  function calls the variable in the global environment:
\end{itemize}

\begin{Shaded}
\begin{Highlighting}[]
\NormalTok{y <-}\StringTok{ }\DecValTok{1}
\NormalTok{plus_}\DecValTok{1}\NormalTok{ <-}\StringTok{ }\ControlFlowTok{function}\NormalTok{(x) \{}
  \KeywordTok{return}\NormalTok{(x }\OperatorTok{+}\StringTok{ }\NormalTok{y)}
\NormalTok{\}}
\KeywordTok{plus_1}\NormalTok{(}\DecValTok{1}\NormalTok{)}
\end{Highlighting}
\end{Shaded}

\begin{verbatim}
## [1] 2
\end{verbatim}

\begin{Shaded}
\begin{Highlighting}[]
\KeywordTok{plus_1}\NormalTok{(}\DecValTok{2}\NormalTok{)}
\end{Highlighting}
\end{Shaded}

\begin{verbatim}
## [1] 3
\end{verbatim}

\begin{itemize}
\tightlist
\item
  However, you should NOT do this. All variables used in a function
  should be given as its arguments:
\end{itemize}

\begin{Shaded}
\begin{Highlighting}[]
\NormalTok{y <-}\StringTok{ }\DecValTok{1}
\NormalTok{plus_}\DecValTok{2}\NormalTok{ <-}\StringTok{ }\ControlFlowTok{function}\NormalTok{(x, y) \{}
  \KeywordTok{return}\NormalTok{(x }\OperatorTok{+}\StringTok{ }\NormalTok{y)}
\NormalTok{\}}
\KeywordTok{plus_2}\NormalTok{(}\DecValTok{2}\NormalTok{, }\DecValTok{3}\NormalTok{)}
\end{Highlighting}
\end{Shaded}

\begin{verbatim}
## [1] 5
\end{verbatim}

\begin{Shaded}
\begin{Highlighting}[]
\KeywordTok{plus_2}\NormalTok{(}\DecValTok{2}\NormalTok{, }\DecValTok{4}\NormalTok{)}
\end{Highlighting}
\end{Shaded}

\begin{verbatim}
## [1] 6
\end{verbatim}

\begin{itemize}
\tightlist
\item
  The best practice is to use \texttt{findGlobals} function in
  \texttt{codetools} to check global variables in a funciton:
\end{itemize}

\begin{Shaded}
\begin{Highlighting}[]
\KeywordTok{library}\NormalTok{(codetools)}
\KeywordTok{findGlobals}\NormalTok{(plus_}\DecValTok{1}\NormalTok{)}
\end{Highlighting}
\end{Shaded}

\begin{verbatim}
## [1] "{"      "+"      "return" "y"
\end{verbatim}

\begin{Shaded}
\begin{Highlighting}[]
\KeywordTok{findGlobals}\NormalTok{(plus_}\DecValTok{2}\NormalTok{)}
\end{Highlighting}
\end{Shaded}

\begin{verbatim}
## [1] "{"      "+"      "return"
\end{verbatim}

\begin{itemize}
\item
  This function returns the list of global variables used in a function.
  If this returns a global variable other than the system global
  gariables, you should include it as the argument of the function.
\item
  You can write the functions in the files with executing codes.
\item
  But I recommend you to separate files for writing functions and
  executing codes.
\item
  \texttt{File\ \textgreater{}\ New\ File\ \textgreater{}\ R\ Rcript}
  and name it as \texttt{functions.R} and save to \texttt{R} folder.
\item
  Cut the function you wrote and paste it in \texttt{functions.R}.
\item
  There are two ways of calling a function in \texttt{functions.R} from
  \texttt{test.R}.
\item
  One way is to use \texttt{source} function.
\end{itemize}

\begin{Shaded}
\begin{Highlighting}[]
\KeywordTok{source}\NormalTok{(}\StringTok{"R/functions.R"}\NormalTok{)}
\end{Highlighting}
\end{Shaded}

\begin{itemize}
\tightlist
\item
  When this line is read, the codes in the file are executed.
\item
  The other way is to bundle functions as a package and load it.
\item
  Choose \texttt{Build\ \textgreater{}\ Clean\ and\ Rebuild}.
\item
  This compiles files in \texttt{src} folder and bundle functions in
  \texttt{R} folder and build a package named \texttt{ECON6120I}.
\item
  Now, the functions in \texttt{R} folder and \texttt{src} folder can be
  used by loading the package by:
\end{itemize}

\begin{Shaded}
\begin{Highlighting}[]
\KeywordTok{library}\NormalTok{(ECON6120I)}
\end{Highlighting}
\end{Shaded}

\begin{itemize}
\tightlist
\item
  Best practice:

  \begin{enumerate}
  \def\labelenumi{\arabic{enumi}.}
  \tightlist
  \item
    Write functions in the scratch file.
  \item
    As the functions are tested, move them to \texttt{R/functions.R}.
  \item
    Clean and rebuild and load them as a package.
  \end{enumerate}
\end{itemize}

\subsection{Reproducible Reports using
Rmarkdown}\label{reproducible-reports-using-rmarkdown}

\begin{itemize}
\tightlist
\item
  Reporting in empirical studies involves:
\end{itemize}

\begin{enumerate}
\def\labelenumi{\arabic{enumi}.}
\tightlist
\item
  Writing texts;
\item
  Writing formulas;
\item
  Writing and implementing programs;
\item
  Demonstrating the results with figures and tables.
\end{enumerate}

\begin{itemize}
\item
  Moreover, this has to be done in a \textbf{reproducible} manner:
  Whoever can reproduce the output from the scratch.
\item
  ``Whoever'' includes yourself in the future. Because the revision
  process of structural papers is usually lengthy, you often have to
  remember the content few weeks or few months later. It is inefficient
  if you cannot recall what you have done.
\item
  We use \texttt{Rmarkdown} to achieve this goal.
\item
  This assumes that you have LaTex installed.
\item
  Install package \texttt{Rmarkdown}:
\end{itemize}

\begin{Shaded}
\begin{Highlighting}[]
\KeywordTok{install.packages}\NormalTok{(}\StringTok{"rmarkdown"}\NormalTok{)}
\end{Highlighting}
\end{Shaded}

\begin{itemize}
\tightlist
\item
  \texttt{File\ \textgreater{}\ New\ File\ \textgreater{}\ R\ Markdown...\ \textgreater{}\ HTML}
  with title \texttt{Test}.
\item
  Save it in \texttt{assignment} folder with name \texttt{test.Rmd}.
\item
  From \texttt{Knit} tab, choose \texttt{Knit\ to\ HTML}.
\item
  This outputs the content to html file.
\item
  You can also choose \texttt{Knit\ to\ PDF} from \texttt{Knit} tab to
  obtain output in pdf file.
\item
  Reports should be knit to pdf to submit.
\item
  But you can use html output while writing a report because html is
  lighter to compile.
\item
  Refer to the \href{https://rmarkdown.rstudio.com/lesson-1.html}{help
  page} for further information.
\end{itemize}

\chapter{Production and Cost Function Estimation}\label{production}

\section{Motivations}\label{motivations}

\begin{itemize}
\tightlist
\item
  Estimating \textbf{production and cost functions} of producers is the
  cornerstone of economic analysis.
\item
  Estimating the functions includes to separate the contribution of
  observed inputs and the other factors, which is often referred to as
  the \textbf{productivity}.
\item
  ``What determines productivity?'' \citep{Syverson2011}-type research
  questions naturally follow.
\item
  The methods covered in this chapter are widely used across different
  fields.
\item
  Some of them are variants from the standard methods.
\end{itemize}

\subsection{IO}\label{io}

\begin{itemize}
\tightlist
\item
  \citet{Olley1996}:

  \begin{itemize}
  \tightlist
  \item
    How much did the deregulation in the U.S. telecommunication
    industry, in particular the divestiture of AT\&T in 1984, spurred
    the productivity growth of the incumbent, facilitated entries, and
    increased the aggregate productivity?
  \item
    To do so, the authors estimate the plant-level production functions
    and productivity in the telecommunication industry.
  \end{itemize}
\item
  \citet{Doraszelski2013a}:

  \begin{itemize}
  \tightlist
  \item
    What is the role of R\&D in determining the differences in
    productivity across firms and the evolution of firm-level
    productivity over time?
  \item
    To do so, the authors estimate the firm-level production functions
    and productivity of Spanish manufacturing firms during 1990s in
    which the transition probability of a productivity is a function of
    the R\&D activities.
  \end{itemize}
\end{itemize}

\subsection{Development}\label{development}

\begin{itemize}
\tightlist
\item
  \citet{Hsieh2009}:

  \begin{itemize}
  \tightlist
  \item
    How large is the misallocation of inputs across manufacturing firms
    in China and India compared to the U.S? How will the aggregate
    productivity of China and India change if the degree of
    misallocation is reduced to the U.S. level?
  \item
    To do so, the authors measure the revenue productivity of firms,
    which should be the same across firms within an industry if there
    were no distortion, and the measurement of the revenue productivity
    requires to estimate the production function.
  \end{itemize}
\item
  \citet{Gennaioli2013}:

  \begin{itemize}
  \tightlist
  \item
    What are the determinants of regional growth? Do geographic,
    institutional, cultural, and human capital factors explain the
    difference across regions?
  \item
    To do so, the authors construct the data set that covers 74\% of the
    world's surface and 97\% of its GDP and estimate the production
    function in which the above mentioned factors could affect the
    productivity.
  \end{itemize}
\end{itemize}

\subsection{Trade}\label{trade}

\begin{itemize}
\tightlist
\item
  \citet{Haskel2007}:

  \begin{itemize}
  \tightlist
  \item
    Are there spillovers from FDI to domestic firms?
  \item
    To do so, the authors estimate the plant-level production function
    of the U.K. manufacturing firms during 1973 and 1992 and study how
    the foreign presence in the U.K. affected the productivity.
  \end{itemize}
\item
  \citet{Loecker2011}:

  \begin{itemize}
  \tightlist
  \item
    Does the removal of trade barriers induces efficiency gain for
    producers?
  \item
    To do so, the author estimate the production functions of Belgian
    textile industry during 1994-2002 in which the degree of trade
    protection can affect the productivity level.
  \end{itemize}
\end{itemize}

\subsection{Management}\label{management}

\begin{itemize}
\tightlist
\item
  \citet{Bloom2007}:

  \begin{itemize}
  \tightlist
  \item
    How do management practices affect the firm productivity?
  \item
    To do so, the authors first estimate the production function and
    productivity of manufacturing firms in developed countries, and then
    study how the independently measured management practices of the
    firms affect the estimated productivity.
  \end{itemize}
\item
  \citet{Braguinsky2015}:

  \begin{itemize}
  \tightlist
  \item
    How do changes in ownership affect the productivity and
    profitability of firms?
  \item
    To do so, the authors estimate the production function for various
    outputs including the physical output, return on capital and labor,
    and the utilization rate, price level, using the cotton spinners
    data in Japan during 1896 and 1920.
  \end{itemize}
\end{itemize}

\subsection{Education}\label{education}

\begin{itemize}
\tightlist
\item
  \citet{Cunha2010}:

  \begin{itemize}
  \tightlist
  \item
    How do childhood and schooling interventions ``produce'' the
    cognitive and non-cognitive skills of children?
  \item
    To do so, the authors estimate the mapping from childhood and
    schooling interventions to children's cognitive and non-cognitive
    skills, the ``production function'' of childhood environment and
    education.
  \end{itemize}
\end{itemize}

\section{Analyzing Producer
Behaviors}\label{analyzing-producer-behaviors}

\begin{itemize}
\item
  There are several levels of parameters that govern the behavior of
  firms:
\item
  \textbf{Production function}

  \begin{itemize}
  \tightlist
  \item
    Add factor market structure.
  \item
    Add cost minimization.
  \end{itemize}
\item
  \(\rightarrow\) \textbf{Cost function}

  \begin{itemize}
  \tightlist
  \item
    Add product market structure.
  \item
    Add profit maximization.
  \end{itemize}
\item
  \(\rightarrow\) \textbf{Supply function (Pricing function)}

  \begin{itemize}
  \tightlist
  \item
    Combine cost and supply (pricing) functions.
  \end{itemize}
\item
  \(\rightarrow\) \textbf{Profit function}
\item
  Which parameter to identify?
\item
  Primitive enough to be invariant to relevant policy changes.

  \begin{itemize}
  \tightlist
  \item
    e.g.~If you conduct a policy experiment that changes the factor
    market structure, identifying cost functions is not enough.
  \end{itemize}
\item
  As reduced-form as possible among such specifications.

  \begin{itemize}
  \tightlist
  \item
    A reduced-form parameter usually can be rationalized by a class of
    underlying structural parameters and institutional assumptions.
    Thus, the analysis becomes robust to some misspecifications.
  \item
    e.g.~A non-parametric function \(C(q, w)\) can represent a cost
    function of a producer who is not necessarily minimizing the cost.
    If we derive a cost function from a production function and a factor
    market structure, then the cost function cannot represent such a
    non-optimization behavior.
  \end{itemize}
\end{itemize}

\section{Production Function
Estimation}\label{production-function-estimation}

\subsection{Cobb-Douglas Specification as a
Benchmark}\label{cobb-douglas-specification-as-a-benchmark}

\begin{itemize}
\tightlist
\item
  Most of the following argument carries over to a general model.
\item
  For firm \(j = 1, \cdots, J\) and time \(t = 1, \cdots, T\), we
  observe output \(Y_{jt}\), labor \(L_{jt}\), and capital \(K_{jt}\).
\item
  We consider an asymptotic of \(J \to \infty\) for a fixed \(T\).
\item
  Assume Cobb-Douglas production function:

  \begin{equation}
  Y_{jt} = A_{jt}  L_{jt}^{\beta_l} K_{jt}^{\beta_k},
  \end{equation}

  where \(A_{jt}\) is firm \(j\) and time \(t\) specific unobserved
  heterogeneity in the model.
\item
  Taking the logarithm gives:

  \begin{equation}
  y_{jt} = \beta_0 + \beta_l l_{jt} + \beta_k k_{jt} + \epsilon_{jt},
  \end{equation}

  where lowercase symbols represent natural logs of variables and
  \(\ln(A_{jt}) = \beta_0 + \epsilon_{jt}\).
\item
  This can be regarded as a first-order log-linear approximation of a
  production function.
\item
  Linear regression model! May OLS work?
\end{itemize}

\subsection{Potential Bias I:
Endogeneity}\label{potential-bias-i-endogeneity}

\begin{itemize}
\tightlist
\item
  \(\epsilon_{jt}\) contains everything that cannot be explained by the
  observed inputs: better capital may be employed, a worker may have
  obtained better skills, etc.
\item
  When the manager of a firm makes an input choice, she should have some
  information about the realization of \(\epsilon_{jt}\).
\item
  Thus, the input choice can be correlated with \(\epsilon_{jt}\); for
  example under static optimization of \(L_{jt}\) given \(K_{jt}\):

  \begin{equation}
  L_{jt} = \Bigg[\frac{p_{jt}}{w_{jt}} \beta_l \exp^{\beta_0 + \epsilon_{jt}} K_{jt}^{\beta_k}\Bigg]^{\frac{1}{1 - \beta_l}}.
  \end{equation}
\item
  In this case, OLS estimator for \(\beta_l\) is \textit{positively}
  biased, because when \(\epsilon_{jt}\) is high, \(l_{jt}\) is high and
  thus the increase in output caused by \(\epsilon_{jt}\) is captured as
  if caused by the increase in labor input.
\item
  The endogeneity problem was already recognized by
  \citet{Marschak1944}.
\end{itemize}

\subsection{Potential Bias II:
Selection}\label{potential-bias-ii-selection}

\begin{itemize}
\tightlist
\item
  Firms freely enter and exit market.
\item
  Therefore, a firm that had low \(\epsilon_{jt}\) is likely to exit.
\item
  However, if firms have high capital \(K_{jt}\), it can stay in the
  market even if the realization of \(\epsilon_{jt}\) is very low.
\item
  Therefore, conditional on being in the market, there is a
  \textit{negative} correlation between the capital \(K_{jt}\) and
  \(\epsilon_{jt}\).
\item
  This problem occurs even if the choice of \(K_{jt}\) itself is not a
  function of \(\epsilon_{jt}\).
\end{itemize}

\subsection{How to Resolve Endogeneity
Bias?}\label{how-to-resolve-endogeneity-bias}

\begin{itemize}
\tightlist
\item
  Temporarily abstract away from entry and exit.
\item
  The data is balanced.
\end{itemize}

\begin{enumerate}
\def\labelenumi{\arabic{enumi}.}
\tightlist
\item
  Panel data.
\item
  First-order condition for inputs.
\item
  Instrumental variable.
\item
  Olley-Pakes approach and its followers/critics.
\end{enumerate}

\begin{itemize}
\tightlist
\item
  \citet{Griliches1998} is a good survey of the history up to
  Olley-Pakes approach.
\item
  \citet{Ackerberg2015} also offer a good survey and clarify problems
  and implicit assumptions in Olley-Pakes approach.
\end{itemize}

\subsection{Panel Data}\label{panel-data}

\begin{itemize}
\tightlist
\item
  Assume that \(\epsilon_{jt} = \mu_j + \eta_{jt}\), where \(\eta_{jt}\)
  is uncorrelated with input choices up to period \(t\):

  \begin{equation}
  y_{jt} = \beta_0 + \beta_l l_{jt} + \beta_k k_{jt} + \mu_j + \eta_{jt}.
  \end{equation}
\item
  Then, by differentiating period \(t\) and \(t - 1\) equations, we get:

  \begin{equation}
  y_{jt} - y_{j, t - 1}= \beta_l (l_{jt} - l_{j, t - 1}) + \beta_k (k_{jt} - k_{j, t - 1}) + (\eta_{jt} - \eta_{j, t - 1}).
  \end{equation}
\item
  Then, because \(\eta_{jt} - \eta_{j, t - 1}\) is uncorrelated either
  with \(l_{jt} - l_{j, t - 1}\) or \(k_{jt} - k_{j, t - 1}\), we can
  identify the parameter.
\item
  Problem:

  \begin{itemize}
  \tightlist
  \item
    Restrictive heterogeneity.
  \item
    When there are measurement errors, fixed-effect estimator can
    generate higher biases than OLS estimator, because measurement
    errors more likely to survive first-difference and
    within-transformation.
  \end{itemize}
\end{itemize}

\subsection{First-Order Condition for
Inputs}\label{first-order-condition-for-inputs}

\begin{itemize}
\tightlist
\item
  Use the first-order condition for inputs as the moment condition
  \citep{McElroy1987}.
\item
  Closely related to the cost function estimation literature.
\item
  Need to specify the factor market structure and the nature of the
  optimization problem for a firm.
\item
  Recently being center of attention again as one of the solutions to
  the ``collinearity problem'' discussed below.
\end{itemize}

\subsection{Instrumental Variable}\label{productioniv}

\begin{itemize}
\item
  Borrow the idea from the first-order condition approach that the input
  choices are affected by some exogenous variables.
\item
  If we have instrumental variables that affect inputs but are
  uncorrelated with errors \(\epsilon_{jt}\), then we can identify the
  parameter by an instrumental variable method.
\item
  One candidate for the instrumental variables: \textbf{input prices}.
\item
  Input price affect input decision.
\item
  Input price is not correlated with \(\epsilon_{jt}\) if the factor
  product market is competitive and \(\epsilon_{jt}\) is an
  idiosyncratic shock to a firm.
\item
  Problems:

  \begin{itemize}
  \tightlist
  \item
    Input prices often lack cross-sectional variation.
  \item
    Cross-sectional variation is often due to unobserved input quality.
  \end{itemize}
\item
  Another candidate for the instrumental variables: \textbf{lagged
  inputs}.
\item
  If \(\epsilon_{jt}\) does not have auto-correlation, lagged inputs are
  not correlated with the current shock.
\item
  If there are adjustment costs for inputs, then lagged inputs are
  correlated with the current inputs.
\item
  Problem:

  \begin{itemize}
  \tightlist
  \item
    If \(\epsilon_{jt}\) has auto-correlation, all lagged inputs are
    correlated with the errors: For example, if \(\epsilon_{jt}\) is
    AR(1),
    \(\epsilon_{jt} = \alpha \epsilon_{j, t - 1} + \nu_{j, t - 1} = \cdots \alpha^l \epsilon_{j, t - l} + \nu_{j, t - 1} + \cdots, \alpha^{l - 1} \nu_{j, t - l}\)
    for any \(l\).
  \end{itemize}
\end{itemize}

\subsection{Olley-Pakes Approach}\label{olley-pakes-approach}

\begin{itemize}
\tightlist
\item
  Exploit restrictions from the economic theory \citep{Olley1996}.
\item
  Write \(\epsilon_{jt} = \omega_{jt} + \eta_{jt}\), where
  \(\omega_{jt}\) is an anticipated shock and \(\eta_{jt}\) is an
  ex-post shock.
\item
  Inputs are correlated with \(\omega_{jt}\) but not with \(\eta_{jt}\)
\item
  The model is written as:

  \begin{equation}
  y_{jt} = \beta_0 + \beta_l l_{jt} + \beta_k k_{jt} + \omega_{jt} + \eta_{jt}.
  \end{equation}
\item
  OP use economic theory to derive a valid proxy for the anticipated
  shock \(\omega_{jt}\).
\end{itemize}

\subsection{Assumption I: Information
Set}\label{assumption-i-information-set}

\begin{itemize}
\tightlist
\item
  The firm's information set at \(t\), \(I_{jt}\), includes current and
  past productivity shocks \(\{\omega_{j\tau}\}_{\tau = 0}^t\) but does
  not include future productivity shocks
  \(\{\omega_{j\tau}\}_{\tau = t + 1}^{\infty}\).
\item
  The transitory shocks \(\eta_{jt}\) satisfy
  \(\mathbb{E}\{\eta_{jt}|I_{jt}\} = 0\).
\end{itemize}

\subsection{Assumption II: First Order
Markov}\label{assumption-ii-first-order-markov}

\begin{itemize}
\tightlist
\item
  Productivity shocks evolve according to the distribution:

  \begin{equation}
  p(\omega_{j, t + 1}|I_{jt}) = p(\omega_{j, t + 1}|\omega_{jt}), 
  \end{equation}

  and the distribution is known to firms and stochastically increasing
  in \(\omega_{jt}\).
\item
  Then:

  \begin{equation}
  \omega_{jt} = \mathbb{E}\{\omega_{jt}|\omega_{j, t - 1}\} + \nu_{jt},
  \end{equation}

  and:

  \begin{equation}
  \mathbb{E}\{\nu_{jt}|I_{j, t - 1}\} = 0,
  \end{equation}

  by construction.
\end{itemize}

\subsection{Assumption III: Timing of Input
Choices}\label{assumption-iii-timing-of-input-choices}

\begin{itemize}
\tightlist
\item
  Firms accumulate capital according to:

  \begin{equation}
  k_{jt} = \kappa(k_{j, t - 1}, i_{j, t - 1}),
  \end{equation}

  where investment \(i_{j, t - 1}\) is chosen in period \(t - 1\).
\item
  Labor input \(l_{jt}\) is non-dynamic and chosen at \(t\).
\item
  This assumption characterizes and distinguishes labor and capital.
\item
  Intuitively, it takes a full period for new capital to be ordered,
  delivered, and installed.
\end{itemize}

\subsection{Assumption IV: Scalar
Unobservable}\label{assumption-iv-scalar-unobservable}

\begin{itemize}
\tightlist
\item
  Firms' investment decisions are given by:

  \begin{equation}
  i_{jt} = f_t(k_{jt}, \omega_{jt}).
  \end{equation}
\item
  This assumption places strong implicit restrictions on additional
  firm-specific unobservables.

  \begin{itemize}
  \tightlist
  \item
    No \textbf{across firm} unobserved heterogeneity in adjustment cost
    of capital, in demand and labor market conditions, or in other parts
    of the production function.
  \item
    Okay with \textbf{across time} unobserved heterogeneity.
  \end{itemize}
\end{itemize}

\subsection{Assumption IV: Strict
Monotonicity}\label{assumption-iv-strict-monotonicity}

\begin{itemize}
\tightlist
\item
  The investment policy function \(f_t(k_{jt}, \omega_{jt})\) is
  strictly increasing in \(\omega_{jt}\).
\item
  This holds if the realization of higher \(\omega_{jt}\) implies higher
  expectation for future productivity (Assumption III) and if the
  marginal product of capital is increasing in the expectation for
  future productivity.
\item
  To verify the latter condition in a given game is often not easy.
\end{itemize}

\subsection{Two-step Approach: The First
Step}\label{two-step-approach-the-first-step}

\begin{itemize}
\tightlist
\item
  Insert \(\omega_{jt} = h(k_{jt}, i_{jt})\) to the original equation to
  get:

  \begin{equation}
  \begin{split}
  y_{jt} &= \beta_l l_{jt} + \underbrace{\beta_0 + \beta_k k_{jt} + h(k_{jt}, i_{jt})}_{\text{unknown function of $k_{jt}$ and $i_{jt}$}} + \eta_{jt}\\
  & \equiv \beta_l l_{jt} + \phi(k_{jt}, i_{jt}) + \eta_{jt}.
  \end{split}
  \end{equation}
\item
  This is a \textbf{partially linear model}: see \citet{Ichimura2007}
  for reference.
\item
  Because \(l_{jt}, k_{jt}\) and \(i_{jt}\) are uncorrelated with
  \(\eta_{jt}\), we can identify \(\beta_l\) and \(\phi(\cdot)\) by
  exploiting the moment condition:

  \begin{equation}
  \begin{split}
  & \mathbb{E}\{\eta_{jt}|l_{jt}, k_{jt}, i_{jt}\} = 0\\
  & \Leftrightarrow \mathbb{E}\{y_{jt} - \beta_l l_{jt} - \phi(k_{jt}, i_{jt}) |l_{jt}, k_{jt}, i_{jt}\} = 0.
  \end{split}
  \end{equation}

  \textbf{if there is enough variation} in \(l_{jt}, k_{jt}\) and
  \(i_{jt}\).
\item
  This ``if there is enough variation'' part is actually problematic.
  Discuss later.
\item
  Let \(\beta_l^0\) and \(\phi^0\) be the identified true parameters.
\end{itemize}

\subsection{Two-step Approach: The Second
Step}\label{two-step-approach-the-second-step}

\begin{itemize}
\tightlist
\item
  Note that:

  \begin{equation}
  \omega_{jt} \equiv \phi(k_{jt}, i_{jt}) - \beta_0 - \beta_k k_{jt}.
  \end{equation}
\item
  Therefore, we have:

  \begin{equation}
  \begin{split}
  &y_{jt} - \beta_l^0 l_{jt} \\
  &= \beta_0 + \beta_k k_{jt} + \omega_{jt} + \eta_{jt}\\
  &= \beta_0 + \beta_k k_{jt} + g(\omega_{j, t - 1}) + \nu_{jt} + \eta_{jt}\\
  &= \beta_0 + \beta_k k_{jt} + g[\phi^0(k_{j, t - 1}, i_{j, t - 1}) - (\beta_0 + \beta_k k_{j, t - 1})] + \nu_{jt} + \eta_{jt}.
  \end{split}
  \end{equation}
\item
  \(\nu_{jt}\) and \(\eta_{jt}\) are independent of the covariates.
\item
  This is a \textbf{multiple-index model} with indices
  \(\beta_0 + \beta_1 k_{jt}\) and \(\beta_0 + \beta_1 k_{j, t - 1}\)
  where parameters of two indices are restricted to be the same: see
  \citet{Ichimura2007} for reference.
\item
  We can identify \(\beta_0, \beta_k\) and \(g\) by exploiting the
  moment condition:

  \begin{equation}
  \begin{split}
  & \mathbb{E}\{\nu_{jt} + \eta_{jt}|k_{jt}, k_{j, t - 1}, i_{j, t - 1}\} = 0\\
  & \Leftrightarrow \mathbb{E}\{\nu_{jt} + \eta_{jt}|k_{jt}, k_{j, t - 1}, i_{j, t - 1}\} = 0.
  \end{split}
  \end{equation}
\end{itemize}

\subsection{Identification of the Anticipated
Shocks}\label{identification-of-the-anticipated-shocks}

\begin{itemize}
\tightlist
\item
  If \(\phi, \beta_0, \beta_k\) are identified, then \(\omega_{jt}\) is
  also identified by:

  \begin{equation}
  \omega_{jt} \equiv \phi(k_{jt}, i_{jt}) - \beta_0 - \beta_k k_{jt}.
  \end{equation}
\end{itemize}

\subsection{\texorpdfstring{Two-Step Estimation of
\citet{Olley1996}.}{Two-Step Estimation of @Olley1996.}}\label{two-step-estimation-of-olley1996.}

\begin{itemize}
\tightlist
\item
  \textbf{First step}: Estimate \(\beta_L\) and \(\phi\) in :

  \begin{equation}
  \begin{split}
  y_{jt} = \beta_l l_{jt} + \phi(k_{jt}, i_{jt}) + \eta_{jt}.
  \end{split}
  \end{equation}

  by approximating \(\phi\) with some basis functions, say, polynomials
  or splines:

  \begin{equation}
  \begin{split}
  y_{jt} &= \beta_l l_{jt} +  \sum_{p = 1}^P \gamma_p \phi_p(k_{jt}, i_{jt}) +  \left[\phi(k_{jt}, i_{jt}) - \sum_{n = 1}^N \gamma_n \phi_n(k_{jt}, i_{jt})\right] + \eta_{jt}\\
  & = \beta_l l_{jt} +  \sum_{p = 1}^P \gamma_p \phi_p(k_{jt}, i_{jt}) + \tilde{\eta}_{jt}
  \end{split}
  \end{equation}

  where \(P \to \infty\) when the sample size goes to infinity.
\item
  e.g.~second-order polynomial approximation:

  \begin{equation}
  \begin{split}
  & \phi_1(k_{jt}, i_{jt}) = k_{jt}, \phi_2(k_{jt}, i_{jt}) = i_{jt}\\
  & \phi_3(k_{jt}, i_{jt}) = k_{jt}^2, \phi_4(k_{jt}, i_{jt}) = i_{jt}^2\\
  & \phi_5(k_{jt}, i_{jt}) = k_{jt} i_{jt}.
  \end{split}
  \end{equation}
\item
  Once the basis functions are fixed, estimation is the same as the
  linear model.
\item
  But the inference (the computation of the standard deviation) is
  difference, because of the approximation error.
\item
  See \citet{Chen2007} for reference.
\item
  Let \(\hat{\beta}_l\) and \(\hat{\phi}\) be the estimates from the
  first step.
\item
  \textbf{Second step}: Estimate \(\beta_0\), \(\beta_k\), and \(g\) in:

  \begin{equation}
  \begin{split}
  y_{jt} - \hat{\beta}_l l_{jt}& = \beta_0 + \beta_k k_{jt} + g[\hat{\phi}(k_{j, t - 1}, i_{j, t - 1}) - (\beta_0 + \beta_k k_{j, t - 1})] + \nu_{jt} + \eta_{jt}\\
  &+ [\beta_l - \hat{\beta}_l] l_{jt}\\
  &+ \left\{g[\phi(k_{j, t - 1}, i_{j, t - 1}) - (\beta_0 + \beta_k k_{j, t - 1})] - g[\hat{\phi}(k_{j, t - 1}, i_{j, t - 1}) - (\beta_0 + \beta_k k_{j, t - 1})]\right\}\\
  & = \beta_0 + \beta_k k_{jt} + g[\hat{\phi}(k_{j, t - 1}, i_{j, t - 1}) - (\beta_0 + \beta_k k_{j, t - 1})] + \nu_{jt} + \tilde{\eta}_{jt}
  \end{split}
  \end{equation}

  by approximating \(g\) by some basis functions, say, polynomials or
  splines.
\end{itemize}

\subsection{From An Economic Models to An Econometric
Model}\label{from-an-economic-models-to-an-econometric-model}

\begin{itemize}
\tightlist
\item
  Starting from economic model with some unobserved heterogeneity, we
  reach some reduced-form model.
\item
  If the resulting model belongs to a class of econometric models whose
  identification and estimation are established, we can simply apply the
  existing methods.
\end{itemize}

\subsection{How to Resolve Selection
Bias}\label{how-to-resolve-selection-bias}

\begin{itemize}
\tightlist
\item
  Use propensity score to correct selection bias: \citet{Ahn1993}.
\item
  At the beginning of period \(t\), after observing \(\omega_{jt}\),
  firm \(j\) decides whether to continue the business
  (\(\chi_{jt} = 1\)) or exit (\(\chi_{jt} = 0)\).
\item
  Assume that the difference between continuation and exit values is
  strictly increasing in \(\omega_{jt}\).
\item
  Then, there is a threshold \(\underline{\omega}(k_{jt})\) such that:

  \begin{equation}
  \chi_{jt} = 
  \begin{cases}
  1 &\text{   if   } \omega_{jt} \ge \underline{\omega}(k_{jt})\\
  0 &\text{   otherwise.}
  \end{cases}
  \end{equation}
\item
  We can only observe firms that satisfy \(\chi_{jt} = 1\).
\end{itemize}

\subsection{Correction in the First
Step}\label{correction-in-the-first-step}

\begin{itemize}
\tightlist
\item
  In the first step, we need no correction because:

  \begin{equation}
  \begin{split}
  &\mathbb{E}\{y_{jt}|l_{jt}, k_{jt}, i_{jt}, \chi_{jt} = 1 \}\\
  &=\beta_l l_{jt} + \phi(k_{jt}, i_{jt}) + \mathbb{E}\{\eta_{jt}|\chi_{jt} = 1\}\\
  &= \beta_l l_{jt} + \phi(k_{jt}, i_{jt}).
  \end{split}
  \end{equation}
\item
  Ex-post shock \(\eta_{jt}\) is independent of continuation/exit
  decision. Therefore, we can identify \(\beta_l\) and \(\phi(\cdot)\)
  as in the previous case.
\end{itemize}

\subsection{Correction in the Second Step I: The Source of
Bias}\label{correction-in-the-second-step-i-the-source-of-bias}

\begin{itemize}
\tightlist
\item
  One the other hand, we need correction in the second step, because:

  \begin{equation}
  \begin{split}
  &\mathbb{E}\{y_{jt} - \beta_l^0 l_{jt}|k_{jt}, i_{jt}, k_{j, t - 1}, l_{j, t - 1}, \chi_{jt} = 1\} \\
  &= \beta_0 + \beta_k k_{jt} + g[\phi^0(k_{jt}, i_{jt}) - (\beta_0 + \beta_k k_{jt})]\\
  & + \mathbb{E}\{\nu_{jt} + \eta_{jt}| k_{jt}, i_{jt}, k_{j, t - 1}, l_{j, t - 1}, \chi_{jt} = 1\}\\
  &= \beta_0 + \beta_k k_{jt} + g[\phi^0(k_{j, t - 1}, i_{j, t - 1}) - (\beta_0 + \beta_k k_{j, t - 1})]\\
  & + \mathbb{E}\{\nu_{jt}| k_{jt}, i_{jt}, k_{j, t - 1}, l_{j, t - 1} , \chi_{jt} = 1\}.
  \end{split}
  \end{equation}

  and

  \begin{equation}
  \mathbb{E}\{\nu_{jt}| k_{jt}, i_{jt}, k_{j, t - 1}, l_{j, t - 1}, \chi_{jt} = 1 \} \neq 0,
  \end{equation}

  since anticipated shock matters continuation/exit decision in period
  \(t\).
\end{itemize}

\subsection{Correction in the Second Step II: Conditional Exit
Probability}\label{correction-in-the-second-step-ii-conditional-exit-probability}

\begin{itemize}
\tightlist
\item
  Let's see that the conditional expectation:

  \begin{equation}
  \begin{split}
  &\mathbb{E}\{\omega_{jt}| k_{jt}, i_{jt}, k_{j, t - 1}, l_{j, t - 1}, \chi_{jt} = 1 \}\\
  &=\mathbb{E}\{\omega_{jt}| k_{jt}, i_{jt}, k_{j, t - 1}, l_{j, t - 1}, \omega_{jt} \ge \underline{\omega}(k_{jt}) \}\\
  &=\int_{\underline{\omega}(k_{jt})} \omega_{jt} \frac{p(\omega_{jt}|\omega_{j, t - 1})}{\int_{\underline{\omega}(k_{jt})} p(\omega|\omega_{j, t - 1}) d\omega } d \omega_{jt}\\
  &\equiv \tilde{g}(\omega_{j, t - 1}, \underline{\omega}(k_{jt})),
  \end{split}
  \end{equation}

  is a function of \(\omega_{j, t - 1}\) and
  \(\underline{\omega}(k_{jt})\).
\end{itemize}

\subsection{Correction in the Second Step III: Invertibility in
Threshold}\label{correction-in-the-second-step-iii-invertibility-in-threshold}

\begin{itemize}
\tightlist
\item
  The propensity of continuation conditional on observed information up
  to period \(t - 1\):

  \begin{equation}
  \begin{split}
  P_{jt} &\equiv \mathbb{P}\{\chi_{jt} = 1|\mathcal{I}_{j, t - 1}\}\\
  &= \mathbb{P}\{\omega_{jt} \ge \underline{\omega}(k_{jt}) |\mathcal{I}_{j, t - 1}\}\\
  &= \mathbb{P}\{g(\omega_{j, t - 1}) + \nu_{jt} \ge \underline{\omega}[(1 - \delta) k_{j, t - 1} + i_{j, t - 1}]|\mathcal{I}_{j, t - 1} \}\\
  &= \mathbb{P}\{ \chi_{jt} = 1| i_{j, t - 1}, k_{j, t - 1}\}.
  \end{split}
  \end{equation}
\item
  \(\rightarrow\) It suffices to condition on
  \(i_{j, t - 1}, k_{j, t - 1}\).
\item
  We also have:

  \begin{equation}
  P_{jt} = \mathbb{P}\{\chi_{jt} = 1| \omega_{j, t - 1}, \underline{\omega}(k_{jt})\},
  \end{equation}

  and it is invertible in \(\underline{\omega}(k_{jt})\), that is,

  \begin{equation}
  \underline{\omega}(k_{jt}) \equiv \psi(P_{jt}, \omega_{j, t - 1}).
  \end{equation}
\end{itemize}

\subsection{Correction in the Second Step IV: Controlling the
Threshold}\label{correction-in-the-second-step-iv-controlling-the-threshold}

\begin{itemize}
\tightlist
\item
  Now, he have:

  \begin{equation}
  \begin{split}
  &\mathbb{E}\{y_{jt} - \beta_l^0 l_{jt}|k_{jt}, i_{jt}, k_{j, t - 1}, l_{j, t - 1}, \chi_{jt} = 1\} \\
  &= \beta_0 + \beta_k k_{jt} + \mathbb{E}\{\omega_{jt}| k_{jt}, i_{jt}, k_{j, t - 1}, l_{j, t - 1} , \chi_{jt} = 1\}\\
  &= \beta_0 + \beta_k k_{jt} + \tilde{g}(\omega_{j, t - 1}, \underline{\omega}(k_{jt}))\\
  &= \beta_0 + \beta_k k_{jt} + \tilde{g}(\omega_{j, t - 1}, \psi(P_{jt}, \omega_{j, t - 1}))\\
  &\equiv \beta_0 + \beta_k k_{jt} + \tilde{\tilde{g}}(\omega_{j, t - 1}, P_{jt})\\
  &= \beta_0 + \beta_k k_{jt} + \tilde{\tilde{g}}[\phi^0(k_{j, t - 1}, i_{j, t - 1}) - (\beta_0 + \beta_k k_{j, t - 1}), P_{jt}].
  \end{split}
  \end{equation}
\item
  At the end, the only difference is to include \(P_{jt}\) as a
  covariate.
\item
  \(P_{jt}\) is a \textbf{known} function of \(i_{j, t - 1}\) and
  \(k_{j, t - 1}\).
\item
  Even if we condition on \(P_{jt} = p\), there are still many
  combinations of \(i_{j, t - 1}\) and \(k_{j, t - 1}\) that gives
  \(P_{jt} = p\).
\item
  With this remaining variation, we can identify \(\beta_0\),
  \(\beta_k\), and \(\tilde{\tilde{g}}\) by the same argument as the
  case without selection, for each \(P_{jt} = p\).
\end{itemize}

\subsection{\texorpdfstring{Three Step Estimation of
\citet{Olley1996}}{Three Step Estimation of @Olley1996}}\label{three-step-estimation-of-olley1996}

\begin{itemize}
\tightlist
\item
  \textbf{Zero step}: Estimate the propensity score:

  \begin{equation}
  P_{jt} = 1\{\chi_{jt} = 1| i_{j, t - 1}, k_{j, t - 1}\},
  \end{equation}

  by a kernel estimator.
\item
  Insert the resulting estimates \(\widehat{P}_{jt}\) into the first and
  second steps.
\end{itemize}

\subsection{Zero Investment Problem}\label{zero-investment-problem}

\begin{itemize}
\tightlist
\item
  One of the key assumptions in OP method was invertibility between
  anticipated shock and investment:

  \begin{equation}
  \omega_{jt} = i^{-1}(k_{jt}, i_{jt}) \equiv h(k_{jt}, i_{jt}).
  \end{equation}
\item
  However, in micro data, zero investment is a rule rather than
  exceptions.
\item
  Then, the invertibility does not hold globally: there are some region
  of the anticipated shock in which the investment takes value zero.
\end{itemize}

\subsection{Tackle Zero Investment Problem I: Discard Some
Data}\label{tackle-zero-investment-problem-i-discard-some-data}

\begin{itemize}
\tightlist
\item
  Discard a data \((j, t)\) such that \(i_{j, t - 1} = 0\).
\item
  Use a data \((j, t)\) such that \(i_{j, t - 1} > 0\).
\item
  Then, invertibility recovers on this selected sample.
\item
  This \textbf{does not} cause bias in the estimator because
  \(\nu_{jt}\) in :

  \begin{equation}
  \beta_0 + \beta_l k_{jt} + g[\phi^0(k_{j, t - 1}, i_{j, t - 1}) - (\beta_0 + \beta_k k_{j, t - 1})] + \nu_{jt} + \eta_{jt},
  \end{equation}

  is independent of the event up to \(t - 1\), including
  \(i_{j, t - 1}\).
\item
  However, this \textbf{does} cause information loss. The loss is high
  if the proportion of the sample such that \(i_{j, t - 1} = 0\) is
  high.
\end{itemize}

\subsection{Tackle Zero Investment Problem II: Use Another
Proxy}\label{tackle-zero-investment-problem-ii-use-another-proxy}

\begin{itemize}
\tightlist
\item
  Investment is just a possible proxy for the anticipated shock.
\item
  Intermediate inputs can be used as proxies as well
  \citep{Levinsohn2003}.
\item
  The problem is that these intermediate inputs are included in the
  gross production function, whereas investment is excluded.
\item
  Let \(m_{jt}\) be the log material input, and assume that the
  production function takes the form of:

  \begin{equation}
  y_{jt} = \beta_0 + \beta_l l_{jt} + \beta_k k_{jt} + \beta_m m_{jt} + \omega_{jt} + \eta_{jt}.
  \end{equation}
\item
  In addition, assume that the \textbf{optimal policy function} for
  \(m_{jt}\) is strictly monotonic in the ex-ante shock, and hence is
  invertible:

  \begin{equation}
  m_{jt} = m(k_{jt}, \omega_{jt}) \Leftrightarrow \omega_{jt} = m^{-1}(m_{jt}, k_{jt}) \equiv h(m_{jt}, k_{jt}). \label{eq:material}
  \end{equation}
\item
  \textbf{First step}:

  \begin{equation}
  \begin{split}
  y_{jt} &= \beta_0 + \beta_l l_{jt} + \beta_k k_{jt} + \beta_m m_{jt} + h(m_{jt}, k_{jt}) + \eta_{jt}\\
  &= \beta_l l_{jt} + \phi(m_{jt}, k_{jt}) + \eta_{jt}.
  \end{split}
  \end{equation}
\item
  We can identify \(\beta_l\) and \(\phi\) by exploiting the moment
  condition:

  \begin{equation}
  \begin{split}
  & \mathbb{E}\{\eta_{jt}|l_{jt}, m_{jt}, k_{jt}, i_{jt}\} = 0\\
  & \Leftrightarrow \mathbb{E}\{y_{jt} - \beta_0 - \beta_l l_{jt} - \phi(m_{jt}, k_{jt}) |l_{jt}, m_{jt}, k_{jt}\} = 0,
  \end{split}
  \end{equation}

  if \textbf{there is enough variation} in \(l_{jt}, m_{jt}, k_{jt}\).
\item
  \textbf{Second step}:

  \begin{equation}
  \begin{split}
  &y_{jt} - \beta_l^0 l_{jt}\\
  & = \beta_k k_{jt} + \beta_m m_{jt} + g[\phi^0(m_{j, t - 1}, k_{j, t - 1}) - \beta_k k_{j, t - 1} - \beta_m m_{j, t - 1}]\\
  & + \nu_{jt} + \eta_{jt}.
  \end{split}
  \end{equation}
\item
  We can identify \(\beta_k\), \(\beta_m\), and \(g\) by exploiting the
  moment condition:

  \begin{equation}
  \begin{split}
  \mathbb{E}\{\nu_{jt} + \eta_{jt} | k_{jt}, m_{j, t - 1}, k_{j,t - 1}\} = 0.
  \end{split}
  \end{equation}
\item
  Because \(m_{jt}\) is correlated with \(\nu_{jt}\), the moment should
  not condition on \(m_{jt}\).
\item
  The identification of \(\beta_{m}\) comes from
  \(\beta_m m_{j, t - 1}\).
\end{itemize}

\subsection{\texorpdfstring{One-step Estimation of \citet{Olley1996} and
\citet{Levinsohn2003}}{One-step Estimation of @Olley1996 and @Levinsohn2003}}\label{one-step-estimation-of-olley1996-and-levinsohn2003}

\begin{itemize}
\tightlist
\item
  \citet{Levinsohn2003} can be estimated in the similar two-step method.
\item
  We can jointly estimate the parameters in first and second steps to
  improve the efficiency \citep{Wooldridge2009}.
\item
  We estimate under the assumptions of \citet{Olley1996}:

  \begin{equation}
  y_{jt} = \beta_0 + \beta_1 l_{jt} + \beta_k k_{jt} + \omega_{jt} + \eta_{jt}.
  \end{equation}
\item
  The first step exploits the following moment:

  \begin{equation}
  \mathbb{E}\{\eta_{jt}|l_{jt}, k_{jt}, i_{jt}\} = 0,
  \end{equation}

  that is:

  \begin{equation}
  \mathbb{E}\{y_{jt} - \beta_1 l_{jt} - \beta_0 - \beta_k k_{jt} - \omega(k_{jt}, i_{jt})|l_{jt}, k_{jt}, i_{jt}\} = 0. \label{eq:opfirst}
  \end{equation}
\item
  We can reinforce the moment condition as:

  \begin{equation}
  \mathbb{E}\{\eta_{jt}|l_{jt}, k_{jt}, i_{jt}, \cdots, l_{j1}, k_{j1}, i_{j1}\} = 0
  \end{equation}

  if we assume that lagged inputs are correlated with the current inputs
  and \(\eta_{jt}\) is independent.
\item
  The second step exploits the following moment:

  \begin{equation}
  \mathbb{E}\{\nu_{jt}|k_{jt}, i_{j, t - 1}, l_{j, t - 1}\} = 0,
  \end{equation}

  that is:

  \begin{equation}
  \mathbb{E}\{y_{jt} - \beta_0 - \beta_1 l_{jt} - \beta_k k_{jt} - g[\omega(k_{j,t - 1}, i_{j, t - 1})]|k_{jt}, i_{j, t - 1}, l_{j, t - 1}\} = 0. \label{eq:opsecond}
  \end{equation}
\item
  We can reinforce the moment condition as:

  \begin{equation}
  \mathbb{E}\{\nu_{jt}|k_{jt}, i_{j, t - 1}, l_{j, t - 1}, \cdots, k_{j1}, i_{j1}, l_{j1}\} = 0,
  \end{equation}

  if we assume that lagged input are correlated with the current inputs
  and \(\nu_{jt} + \eta_{jt}\) are independent.
\item
  We can construct a GMM estimator based on equations \eqref{eq:opfirst}
  and \eqref{eq:opsecond}.
\item
  The one-step estimator can be more efficient but can be
  computationally heavier than the two-step estimator.
\end{itemize}

\subsection{Scalar Unobservable Problem: Finite-order Markov
Process}\label{scalar-unobservable-problem-finite-order-markov-process}

\begin{itemize}
\tightlist
\item
  Borrow the idea of using the first-order condition to resolve the
  collinearity problem \citep{Gandhi2017a}.
\item
  We have assumed that anticipated shocks follow a first-order Markov
  process:

  \begin{equation}
  \omega_{jt} = g(\omega_{j, t - 1}) + \nu_{jt}.
  \end{equation}
\item
  However, it may be true that it has more than one lags, for example:

  \begin{equation}
  \omega_{jt} = g(\omega_{j, t - 1}, \omega_{j, t - 2}) + \nu_{jt}.
  \end{equation}
\item
  Then, we need proxies as many as the number of unobservables:

  \begin{equation}
  \begin{pmatrix}
  i_{jt} \\ m_{jt} 
  \end{pmatrix}
  = \Gamma(k_{jt}, \omega_{jt}, \omega_{j, t - 1}),
  \end{equation}

  such that the policy function for the proxies is a bijection in
  \((\omega_{jt}, \omega_{j, t - 1})\).
\item
  Then, we can have:

  \begin{equation}
  \omega_{jt} = \Gamma_1^{-1}(k_{jt}, i_{jt}, m_{jt}).
  \end{equation}
\item
  The reminder goes as in the standard OP method.
\end{itemize}

\subsection{Scalar Unobservable Problem: Demand and Productivity
Shocks}\label{scalar-unobservable-problem-demand-and-productivity-shocks}

\begin{itemize}
\tightlist
\item
  There may be a demand shock \(\mu_{jt}\) that also follows first-order
  Markov process.
\item
  Then, the policy function depend both on \(\mu_{jt}\) and
  \(\omega_{jt}\).
\item
  We again need proxies as many as the number of unobservable.
\item
  Suppose that we can observe the price of the firm \(p_{jt}\).
\item
  Inverting the policy function:

  \begin{equation}
  \begin{pmatrix}
  i_{jt}\\ p_{jt}
  \end{pmatrix}
  = \Gamma(k_{jt}, \omega_{jt}, \mu_{jt}).
  \end{equation}

  yields:

  \begin{equation}
  \omega_{jt} = \Gamma_1^{- 1}(k_{jt}, i_{jt}, p_{jt}).
  \end{equation}
\item
  If \(\omega_{jt}\) only depends on \(\omega_{j, t - 1}\) but not on
  \(\mu_{j, t - 1}\), then the second step of the modified OP method is
  to estimate:

  \begin{equation}
  \begin{split}
  y_{jt} - \hat{\beta}_l l_{jt} 
  &= \beta_0 + \beta_k k_{jt}\\
  & + g(\omega_{j, t - 1}) + \nu_{jt} + \eta_{jt}\\
  &= \beta_0 + \beta_k k_{jt}\\
  & + g(\hat{\phi}_{j, t - 1} - \beta_0 - \beta_k k_{j, t - 1}) + \nu_{jt} + \eta_{jt}.
  \end{split}
  \end{equation}
\item
  It goes as in the standard OP method.
\item
  If \(\omega_{jt}\) depends both on \(\omega_{j, t - 1}\) and
  \(\mu_{j, t - 1}\), the second step regression equation will be:

  \begin{equation}
  \begin{split}
  y_{jt} - \hat{\beta}_l l_{jt} 
  &= \beta_0 + \beta_k k_{jt}\\
  & + g(\omega_{j, t - 1}, \mu_{j, t - 1}) + \nu_{jt} + \eta_{jt}\\
  &= \beta_0 + \beta_k k_{jt}\\
  & + g(\hat{\phi}_{j, t - 1} - \beta_0 - \beta_k k_{j, t - 1}, \mu_{j, t - 1}) + \nu_{jt} + \eta_{jt}.
  \end{split}
  \end{equation}
\item
  We still have to control \(\mu_{j, t - 1}\) in the second step.
\item
  Invert the policy function for \(\mu_{j, t - 1}\) to get:

  \begin{equation}
  \mu_{j, t - 1} = \Gamma_2^{- 1}(k_{j, t - 1}, i_{j, t - 1}, p_{j, t - 1}),
  \end{equation}

  and plug it into the second step regression equation to get:

  \begin{equation}
  \begin{split}
  &y_{jt} - \hat{\beta}_l l_{jt}\\
  &= \beta_0 + \beta_k k_{jt}\\
  &+g(\hat{\phi}_{j, t - 1} - \beta_0 - \beta_k k_{j, t - 1}, \Gamma_2^{- 1}(k_{j, t - 1}, i_{j, t - 1}, p_{j, t - 1})) + \nu_{jt} + \eta_{jt}.
  \end{split}
  \end{equation}
\item
  The parameters \(\beta_0\) and \(\beta_k\) \textbf{cannot} be
  identified only with this observation, because \(\Gamma_2^{-1}\) is
  \textbf{unknown non-parametric} function: it can mean any function of
  \((k_{j, t - 1}, i_{j, t - 1}, p_{j, t - 1})\).
\item
  To estimate such a model, we jointly estimate the demand function
  along with the production function.
\item
  At this point, we do not investigate it further because we have not
  yet learned how to estimate the demand function.
\item
  For now just keep in mind that:

  \begin{itemize}
  \tightlist
  \item
    There has to be as many proxies as the dimension of the unobservable
    state variables.
  \item
    It is okay that the unobservable state variable includes a demand
    shock.
  \item
    It can be problematic when the unobservable demand shock affect the
    evolution of the anticipated productivity shock.
  \end{itemize}
\end{itemize}

\subsection{Collinearity Problem}\label{collinearity-problem}

\begin{itemize}
\tightlist
\item
  The collinearity problem is formally pointed out by
  \citet{Ackerberg2015}.
\item
  This paper is finally published in 2015, but has been circulated since
  2005.
\item
  We assumed that \(k_{jt}\) and \(\omega_{jt}\) are state variables.
\item
  Then the policy function for labor input should take the form of:

  \begin{equation}
  l_{jt} = l(k_{jt}, \omega_{jt}).
  \end{equation}
\item
  However, because \(\omega_{jt} = h(i_{jt}, k_{jt})\), we have:

  \begin{equation}
  l_{jt} = l[k_{jt}, h(i_{jt}, k_{jt})] = \tilde{l}(i_{jt}, k_{jt}).
  \end{equation}
\item
  Therefore, in the first stage, we encounter a multicollinearity
  problem:

  \begin{equation}
  \begin{split}
  y_{jt} &= \beta_0 + \beta_l \tilde{l}(i_{jt}, k_{jt}) + \phi(i_{jt}, k_{jt}) + \eta_{jt}\\
  &\equiv \tilde{\phi}(i_{jt}, k_{jt}).
  \end{split}
  \end{equation}
\item
  Thus, \(\beta_l\) cannot be identified in the first step.
\item
  The second step becomes:

  \begin{equation}
  y_{jt} = \beta_0 + \beta_l l_{jt} + \beta_k k_{jt} + g[\tilde{\phi}(i_{j, t - 1}, k_{j, t - 1}) - \beta_0 - \beta_l l_{j, t - 1} - \beta_k k_{jt}] + \nu_{jt} + \eta_{jt}
  \end{equation}
\item
  Because \(l_{jt}\) is correlated with \(\nu_{jt}\), moment can only
  condition on \(l_{j, t - 1}\).
\item
  However, conditioning on \(k_{j, t - 1}\) and \(i_{j, t - 1}\), again
  there is no remaining variation in \(l_{j, t - 1}\).
\item
  Therefore, \(\beta_l\) cannot be identified either in the second step.
\item
  \textbf{\(\beta_l\) cannot be identified!}
\end{itemize}

\subsection{Tackle Collinearity Problem: Peculiar
Assumptions}\label{tackle-collinearity-problem-peculiar-assumptions}

\begin{itemize}
\tightlist
\item
  To make Olley-Pakes/Levinsohn-Petrin approach workable, we need
  peculiar data generating process for \(l_{jt}\).
\item
  Consider Levinsohn-Petrin framework.
\end{itemize}

\begin{enumerate}
\def\labelenumi{\arabic{enumi}.}
\tightlist
\item
  There is an optimization error in \(l_{jt}\).

  \begin{itemize}
  \tightlist
  \item
    If it is not i.i.d over time, it becomes a state variable and enters
    to the policy for \(m_{jt}\), violating the scalar unobserved
    heterogeneity assumption of \(m_{jt}\).
  \item
    If there is an optimization error for \(m_{jt}\), this again
    violates the scalar unobserved heterogeneity assumption.
  \end{itemize}
\item
  \(k_{jt}\) is realized, \(\omega_{jt}\) is observed, \(m_{jt}\) and
  \(i_{jt}\) are determined, a new i.i.d. unexpected shock is observed,
  \(l_{jt}\) is determined, and \(\eta_{jt}\) is observed.

  \begin{itemize}
  \tightlist
  \item
    If it is not i.i.d over time, it becomes a state variable and enters
    to the policy for \(m_{jt}\), violating the scalar unobserved
    heterogeneity assumption.
  \end{itemize}
\item
  \(k_{jt}\) is realized, an unexpected shock is observed, \(l_{jt}\) is
  determined, \(\omega_{jt}\) is observed, \(m_{jt}\) and \(i_{jt}\) are
  determined, and \(\eta_{jt}\) is observed (\citet{Ackerberg2016}
  recommends this assumption).

  \begin{itemize}
  \tightlist
  \item
    In this case, the unexpected shock can be serially correlated,
    because it suffices to know \(k_{jt}\), \(i_{jt}\), \(l_{jt}\) to
    decide \(m_{jt}\). It does not have to predict the future unexpected
    shock based on the realization of the current shock because
    \(m_{jt}\) is a static decision.
  \item
    This changes the optimal policy function of \(m_{jt}\)
    \eqref{eq:material} to:

    \begin{equation}
    m_{jt} = m(k_{jt}, \omega_{jt}, l_{jt}).
    \end{equation}
  \item
    The first step:

    \begin{equation}
    \begin{split}
    y_{jt} &= \beta_0 + \beta_l l_{jt} + \beta_k k_{jt} + h(k_{jt}, m_{jt}, l_{jt}) + \eta_{jt}\\
    &= \psi(k_{jt}, m_{jt}, l_{jt}) + \eta_{jt}.\\
    \Rightarrow & \mathbb{E}\{y_{jt} - \psi(k_{jt}, m_{jt}, l_{jt})|k_{jt}, m_{jt}, l_{jt}\} = 0.
    \end{split}
    \end{equation}
  \item
    The second step:

    \begin{equation}
    \begin{split}
    y_{jt} &= \beta_0 + \beta_l l_{jt} + \beta_k k_{jt} + g[\psi(k_{j, t - 1}, m_{j, t - 1}, l_{j, t - 1}) - \beta_0 - \beta_l l_{j, t - 1} - \beta_k k_{j, t - 1}] + \nu_{jt} + \eta_{jt}\\
    \Rightarrow & \mathbb{E}\{y_{jt} - \beta_0 - \beta_l l_{jt} - \beta_k k_{jt} - g[\psi(k_{j, t - 1}, m_{j, t - 1}, l_{j, t - 1}) - \beta_0 - \beta_l l_{j, t - 1} - \beta_k k_{j, t - 1}]|k_{j, t - 1}, i_{j, t - 1}, l_{j, t - 1}, m_{j, t - 1}\}
    \end{split}
    \end{equation}
  \item
    \(m_{jt}\) has to be excluded from the production function, i.e., it
    has to be a value-added production function. Otherwise,
    \(\beta_m m_{jt}\) and \(\beta_m m_{j, t - 1}\) appear in the second
    step. Because \(m_{jt}\) is correlated with \(\nu_{jt}\), the only
    hope is to vary \(m_{j, t - 1}\). But there is no additional
    variation in \(m_{j, t - 1}\) conditional on \(k_{j, t - 1}\),
    \(i_{j, t - 1}\), and \(l_{j, t - 1}\).
  \end{itemize}
\end{enumerate}

\subsection{Tackle Collinearity Problem: Share
Regression}\label{tackle-collinearity-problem-share-regression}

\begin{itemize}
\item
  How to avoid the peculiar assumptions on shocks and timing of
  decisions?
\item
  How to identify gross production function avoiding the third
  assumption by \citet{Ackerberg2015}?
\item
  Return to the old literature using the first-order condition.
\item
  Let \(w_t\) be wage and \(p_t\) be the product price.
\item
  Assume that the factor market is competitive.
\item
  Then, the first-order condition for profit maximization with respect
  to \(L_{jt}\) is:

  \begin{equation}
  \begin{split}
  &P_t F_L(L_{jt}, K_{jt})e^{\omega_{jt}} \mathbb{E} e^{\eta_{jt}} = w_t\\
  &\Leftrightarrow \frac{P_t F_L(L_{jt}, K_{jt})e^{\omega_{jt}} \mathbb{E} e^{\eta_{jt}}}{F(L_{jt}, K_{jt}) } = \frac{w_t}{F(L_{jt}, K_{jt}) }\\
  &\Leftrightarrow \frac{F_L(L_{jt}, K_{jt}) L_{jt}}{F(L_{jt}, K_{jt})  e^{\eta_{jt}} } = \frac{w_t L_{jt}}{P_t \underbrace{F(L_{jt}, K_{jt}) e^{\omega_{jt}} e^{\eta_{jt}}}_{Y_{jt}} },
  \end{split}
  \end{equation}

  where the right hand side is expenditure share to the labor, which is
  observed.
\item
  Furthermore, on the left hand side, we only have \(\eta_{jt}\), which
  is independent of inputs.
\item
  Let \(s_{jt}\) be the log of expenditure share to the labor, and take
  a log of the previous equation gives:

  \begin{equation}
  \begin{split}
  s_{jt} &= \log [F_L(L_{jt}, K_{jt}) L_{jt} \mathbb{E} e^{\eta_{jt}} / F(L_{jt}, K_{jt})] - \eta_{jt}\\
  & = \log(\beta_l) + \ln \mathbb{E} e^{\eta_{jt}} - \eta_{jt}.
  \end{split}
  \end{equation}
\item
  Remember that the coefficient in the Cobb-Douglas function is equal to
  the expenditure share.
\item
  In general, share regression provides additional variation to identify
  the elasticity of anticipated production with respect to the labor.
  Then we can follow the standard OP method to recover other parameters.
\end{itemize}

\section{Cost Function Estimation}\label{cost-function-estimation}

\subsection{Cost Function: Duality}\label{cost-function-duality}

\begin{itemize}
\item
  Given a function \(y = F(x)\) such that:

  \begin{itemize}
  \tightlist
  \item
    Add factor market structure.
  \item
    Add cost minimization.
  \end{itemize}
\item
  \(\rightarrow\) There exists a unique \textbf{cost function}
  \(c = C(y, p)\):

  \begin{itemize}
  \tightlist
  \item
    \textbf{Positivity}: positive for positive input prices and a
    positive.
  \item
    \textbf{Homogeneity}: homogeneous of degree one in the input prices.
  \item
    \textbf{Monotonicity}: increasing in the input prices and in the
    level of output.
  \item
    \textbf{Concavity}: concave in the input prices.
  \end{itemize}
\item
  Given a function \(c = C(y, p)\) such that:

  \begin{itemize}
  \tightlist
  \item
    \textbf{Positivity}: positive for positive input prices and a
    positive.
  \item
    \textbf{Homogeneity}: homogeneous of degree one in the input prices.
  \item
    \textbf{Monotonicity}: increasing in the input prices and in the
    level of output.
  \item
    \textbf{Concavity}: concave in the input prices.
  \end{itemize}
\item
  \(\rightarrow\) There exists a unique production function \(F(x)\)
  that yields \(C(y, p)\) as a solution to the cost minimization
  problem:

  \begin{equation}
  C(y, p) = \min_{x} p'x \text{   s.t.   } F(x) \ge y.
  \end{equation}
\item
  If the latter condition holds, the function \(C\) is said to be
  \textbf{integrable}.
\item
  It is rare that you can find a closed-form cost function of a
  production function.
\item
  It makes sense to start from cost function.
\item
  The duality ensures that there is a one-to-one mapping between a class
  of cost function and a class of production function.
\item
  If you accept competitive factor markets and cost minimization,
  identifying a cost function is equivalent to identifying a production
  function.
\item
  We used this idea in the last slides to identify the parameters
  regarding static decision variables.
\item
  See \citet{Jorgenson1986} for the literature in this topic up to the
  mid 80s.
\end{itemize}

\subsection{Translog Cost Function}\label{translog-cost-function}

\begin{itemize}
\tightlist
\item
  One of the popular specifications:

  \begin{equation}
  \begin{split}
  \ln c &= \alpha_0 + \alpha_p' \ln p + \alpha_y \ln y + \frac{1}{2} \ln p' B_{pp} \ln p\\
  & + \ln p' \beta_{py} \ln y + \frac{1}{2}\beta_{yy}(\ln y)^2.
  \end{split}
  \end{equation}
\item
  It assumes that the first and second order elasticities are constant.
\item
  A second-order (log) Taylor approximation of a general cost function.
\end{itemize}

\subsection{Translog Cost Function:
Integrability}\label{translog-cost-function-integrability}

\begin{itemize}
\tightlist
\item
  Translog cost function is known to be integrable if the following
  conditions hold:
\item
  \textbf{Homogeneity}: the cost shares and the cost flexibility are
  homogeneity of degree zero: \(B_{pp}1 = 0\), \(\beta_{py}'1 = 0\).
\item
  \textbf{Cost exhaustion}: the sum of cost shares is equal to unity:
  \(\alpha_p'1 = 1\), \(B_{pp}'1 = 0\), \(\beta_{py}'1 = 0\).
\item
  \textbf{Symmetry}: the matrix of share elasticities, biases of scale,
  and the cost flexibility elasticity is symmetric:

  \begin{equation}
  \begin{pmatrix}
  B_{pp} & \beta_{py}\\
  \beta_{py}' & \beta_{yy}
  \end{pmatrix}
  =
  \begin{pmatrix}
  B_{pp} & \beta_{py}\\
  \beta_{py}' & \beta_{yy}
  \end{pmatrix}'.
  \end{equation}
\item
  \textbf{Monotonicity}: The matrix of share elasticities
  \(B_{pp} + vv' - diag(v)\) is positive semi-definite.
\end{itemize}

\subsection{Two Approaches}\label{two-approaches}

\begin{enumerate}
\def\labelenumi{\arabic{enumi}.}
\tightlist
\item
  Cost data approach.

  \begin{itemize}
  \tightlist
  \item
    Use accounting cost data.
  \item
    It does not depend on behavioral assumption.
  \item
    One can impose restrictions of assuming cost minimization.
  \item
    The accounting cost data may not represent economic cost.
  \end{itemize}
\item
  Revealed preference approach.

  \begin{itemize}
  \tightlist
  \item
    Assume decision problem for firms.
  \item
    Assume profit maximization.
  \item
    Reveal the costs from firm's equilibrium strategy.
  \item
    It depends on structural assumptions.
  \item
    It reveals the cost as perceived by firms.
  \end{itemize}
\end{enumerate}

\subsection{Cost Data Approach}\label{cost-data-approach}

\begin{itemize}
\tightlist
\item
  Estimating a cost function using cost data from accounting data.
\item
  \citet{McElroy1987} is one of the most flexible and robust frameworks.
\item
  The approach is somewhat getting less popular in IO researchers.
\item
  Recently, the approach is not popular among IO researchers.
\item
  I \textit{conjecture} one of the reasons for this is that IO
  researchers believe cost data taken from accounting information does
  not capture all the costs firms face.
\item
  However, it is good to know the classical literature because it
  sometimes gives a new insight.
\item
  cf. \citet{Byrne2015} : Propose a novel method to combine accounting
  cost data to estimate demand and cost function jointly without using
  instrumental variable approach.
\end{itemize}

\subsection{Revealed Preference
Approach}\label{revealed-preference-approach}

\begin{itemize}
\tightlist
\item
  Another approach is to \textbf{reveal} the marginal cost from firm's
  price/quantity setting behavior assuming it is maximizing profit.

  \begin{itemize}
  \tightlist
  \item
    A parameter affects economic agent's action.
  \item
    Therefore, economic agent's action \textbf{reveals} the information
    about the parameter.
  \item
    See \citet{Bresnahan1981} and \citet{Bresnahan1989} for reference.
  \end{itemize}
\item
  We have shown that the assumption on the factor market and cost
  function minimization gives restriction on the cost parameters.
\item
  We may further assume the product market structure and profit
  maximization to identify cost parameters.
\item
  Example: In a competitive market, the equilibrium price is equal to
  the marginal cost. Therefore, the marginal cost is identified from
  prices.
\item
  What if the competition is imperfect?
\end{itemize}

\subsection{Single-product Monopolist}\label{single-product-monopolist}

\begin{itemize}
\item
  This approach requires researcher to specify the decision problem of a
  firm.
\item
  Assume that the firm is a single-product monopolist.
\item
  Let \(D(p)\) be the demand function.
\item
  Let \(C(q)\) be the cost function.
\item
  Temporarily, assume that we \textbf{know} the demand function.
\item
  We learn how to estimate demand functions in coming weeks.
\item
  The only unknown parameter is the cost function.
\item
  The monopolist solves:

  \begin{equation}
  \max_{p} D(p)p - C(D(p)).
  \end{equation}
\item
  The first-order condition w.r.t. \(p\) for profit maximization is:

  \begin{equation}
  \begin{split}
  &D(p) + pD'(p) - C'(D(p)) D'(p) = 0.\\
  &\Leftrightarrow C'(D(p)) = \underbrace{\frac{D(p) + pD'(p)}{D'(p)}}_{\text{$p$ is observed and $D(p)$ is known.}}
  \end{split}
  \end{equation}
\item
  This identifies the marginal cost
  \textit{at the equilibrium quantity}.
\item
  To trace out the entire marginal cost function, you need a demand
  shifter \(Z\) that changes the equilibrium: \(D(p, Z)\).

  \begin{equation}
  C'(D(p, z)) = \frac{D(p, z) + pD'(p, z)}{D'(p, z)}
  \end{equation}
\item
  This identifies the marginal cost function
  \textit{at the equilibrium quantity when $Z = z$}.
\item
  If the equilibrium quantities cover the domain of the marginal cost
  function when the demand shifter \(Z\) moves around, then it
  identifies the entire marginal cost function.
\end{itemize}

\subsection{Unobserved Heterogeneity in the Cost
Function}\label{unobserved-heterogeneity-in-the-cost-function}

\begin{itemize}
\item
  Previously we did not consider any unobserved heterogeneity in the
  cost function.
\item
  Now suppose that the cost function is given by:

  \begin{equation}
  C(q) = \tilde{C}(q) + q \epsilon + \mu,
  \end{equation}

  and \(\epsilon\) and \(\mu\) are not observed.
\item
  Moreover, because it includes anticipated shocks, it is likely to be
  correlated with input decisions and hence the output.
\item
  The first-order condition w.r.t. \(p\) for profit maximization is:

  \begin{equation}
  \begin{split}
  &D(p, z) + pD'(p, z) - [\tilde{C}'(D(p, z)) + \epsilon]D'(p, z) = 0.\\
  &\Leftrightarrow \tilde{C}'(D(p, z))  = \frac{D(p, z) + pD'(p, z)}{D'(p,z)} - \epsilon.
  \end{split}
  \end{equation}
\item
  Take the expectation conditional on \(Z = z\):

  \begin{equation}
  \tilde{C}'(D(p, z)) = \frac{D(p, z) + pD'(p, z)}{D'(p, z)} - \mathbb{E}\{\epsilon|Z = z\}.
  \end{equation}
\item
  If \(Z\) and \(\epsilon\) is independent, then the last term becomes
  zero and we can follow the same argument as before to trace out the
  marginal cost function.
\end{itemize}

\subsection{Multi-product Monopolist
Case}\label{multi-product-monopolist-case}

\begin{itemize}
\item
  Demand for good \(j\) is \(D_j(p)\) given a price vector \(p\).
\item
  Cost for producing a vector of good \(q\) is \(C(q)\).
\item
  Demand function is \textbf{known} but cost function is not known.
\item
  The monopolist solves:

  \begin{equation}
  \max_{p} \sum_{j = 1}^J p_j D_j(p) - C(D_1(p), \cdots, D_J(p)).
  \end{equation}
\item
  The first-order condition w.r.t. \(p_i\) for profit maximization is:

  \begin{equation}
  \begin{split}
  &D_i(p) + p_i \sum_{j = 1}^J \frac{\partial D_j(p)}{\partial p_i} = \sum_{j = 1}^J \frac{\partial C(D_1(p), \cdots, D_J(p))}{\partial q_j} \frac{\partial D_j(p)}{\partial p_i}.\\
  &= 
  \begin{pmatrix}
  \frac{\partial D_1(p)}{\partial p_i} & \cdots & \frac{\partial D_J(p)}{\partial p_i}
  \end{pmatrix}
  \begin{pmatrix}
  \frac{\partial C(D_1(p), \cdots, D_J(p))}{\partial q_1}\\
  \vdots\\
  \frac{\partial C(D_1(p), \cdots, D_J(p))}{\partial q_J}
  \end{pmatrix}
  \end{split}
  \end{equation}
\item
  Summing up, the first-order condition w.r.t. \(p\) is summarized as:

  \begin{equation}
  \begin{split}
  &\begin{pmatrix}
   D_1(p) + p_1 \sum_{j = 1}^J \frac{\partial D_j(p)}{\partial p_1}\\
   \vdots\\
   D_J(p) + p_J \sum_{j = 1}^J \frac{\partial D_j(p)}{\partial p_J}
  \end{pmatrix} 
  =
  \begin{pmatrix}
  \frac{\partial D_1(p)}{\partial p_1} & \cdots & \frac{\partial D_J(p)}{\partial p_1}\\
  \vdots\\
  \frac{\partial D_1(p)}{\partial p_J} & \cdots & \frac{\partial D_J(p)}{\partial p_J}
  \end{pmatrix}
  \begin{pmatrix}
  \frac{\partial C(D_1(p), \cdots, D_J(p))}{\partial q_1}\\
  \vdots\\
  \frac{\partial C(D_1(p), \cdots, D_J(p))}{\partial q_J}
  \end{pmatrix}\\
  &\Leftrightarrow
  \begin{pmatrix}
  \frac{\partial C(D_1(p), \cdots, D_J(p))}{\partial q_1}\\
  \vdots\\
  \frac{\partial C(D_1(p), \cdots, D_J(p))}{\partial q_J}
  \end{pmatrix} = 
  \underbrace{\begin{pmatrix}
  \frac{\partial D_1(p)}{\partial p_1} & \cdots & \frac{\partial D_J(p)}{\partial p_1}\\
  \vdots\\
  \frac{\partial D_1(p)}{\partial p_J} & \cdots & \frac{\partial D_J(p)}{\partial p_J}
  \end{pmatrix}^{-1}  
  \begin{pmatrix}
   D_1(p) + p_1 \sum_{j = 1}^J \frac{\partial D_j(p)}{\partial p_1}\\
   \vdots\\
   D_J(p) + p_J \sum_{j = 1}^J \frac{\partial D_j(p)}{\partial p_J}
  \end{pmatrix}.}_{\text{$p$ is observed and $D(p)$s are known.}}
  \end{split}
  \end{equation}
\item
  Hence, the cost function is identified.
\item
  Including unobserved heterogeneity in the cost function causes the
  same problem as in the previous case.
\end{itemize}

\subsection{Oligopoly}\label{oligopoly}

\begin{itemize}
\item
  There are firm \(j = 1, \cdots, J\) and they sell product
  \(j = 1, \cdots, J\), that is, firm = product (for simplicity).
\item
  Consider a price setting game. When the price vector is \(p\), demand
  for product \(j\) is given by \(D_j(p)\).
\item
  The cost function for firm \(j\) is \(C_j(q_j)\).
\item
  Given other firms' price \(p_{-j}\), firm \(j\) solves:

  \begin{equation}
  \max_{p_j} D_j(p) p_j - C_j(D_j(p)).
  \end{equation}
\item
  The first-order condition w.r.t. \(p_j\) for profit maximization is:

  \begin{equation}
  \begin{split}
  &D_j(p) + \frac{\partial D_j(p)}{\partial p_j} p_j = \frac{\partial C_j(D_j(p))}{\partial q_j} \frac{\partial D_j(p)}{\partial p_j}.\\
  &\frac{\partial C_j(D_j(p))}{\partial q_j} = \underbrace{\frac{\partial D_j(p)}{\partial p_j}^{-1}[D_j(p) + \frac{\partial D_j(p)}{\partial p_j} p_j ]}_{\text{$p$ is observed and $D_j(p)$ is known}}.
  \end{split}
  \end{equation}
\item
  In Nash equilibrium, these equations jointly hold for all firms
  \(j = 1, \cdots, J\).{]}
\item
  Including unobserved heterogeneity in the cost function causes the
  same problem as in the previous case.
\end{itemize}

\chapter{Demand Function Estimation}\label{demand}

\section{Motivations}\label{motivations-1}

\begin{itemize}
\item
  From demand function and utility maximization assumption, we can
  reveal the preference of the decision maker.
\item
  Thus, estimating demand function is necessary for \textbf{evaluating
  the consumer welfare}.
\item
  In IO, estimating the \textbf{price elasticity of demand} is
  specifically important, because it determines the \textbf{market
  power} of a monopolist and the size of the dead-weight loss.
\item
  In macroeconomics, estimating demand is in important to determine the
  \textbf{price level}, because the price level is the minimum
  expenditure for a consumer to achieve the certain level of utility.
\item
  In marketing, estimating demand is necessary to design the optimal
  pricing, advertising, and all the other marketing interventions.
\item
  In principle, the theory can be applied to whatever decisions other
  than the consumer choice.
\item
  \citet{Nevo2000c}:

  \begin{itemize}
  \tightlist
  \item
    How do the hypothetical mergers in the ready-to-eat cereal industry
    affect the market price, markup, and consumer surplus?
  \item
    To do so, the authors estimate the demand for ready-to-eat cereals
    and the cost functions for each product. Then, the authors conduct
    counterfactual simulations of mergers to quantify the effects.
  \end{itemize}
\item
  \citet{Chung2002}:

  \begin{itemize}
  \tightlist
  \item
    To what extent do firms go abroad to access technology available in
    other locations?
  \item
    To study this issue, the authors estimate the firms' locational
    choice when going abroad.
  \end{itemize}
\item
  \citet{Rysman2004}:

  \begin{itemize}
  \tightlist
  \item
    In Yellow pages, how do consumers evaluate the advertisement on it,
    and how do advertisers value consumer usage?
  \item
    To study this, the author simultaneously estimate the consumer
    demand for usage of a directory, advertiser demand for advertising,
    and a publisher's first-order condition.
  \end{itemize}
\item
  \citet{Gentzkow2004}:

  \begin{itemize}
  \tightlist
  \item
    Are online and print newspapers substitutes or complements?
  \item
    To study this, the author estimate a demand function in which online
    and print newspapers can be either substitutes or complements.
  \end{itemize}
\item
  \citet{Bayer2007}:

  \begin{itemize}
  \tightlist
  \item
    How is the preference of people for schools and neighborhoods? How
    is this capitalized into housing prices?
  \item
    To do so, the authors estimate the discrete choice of residents over
    locations. To deal with the endogeneity between the neighborhood and
    the unobserved attributes of the location, the authors use the
    discontinuity at the school attendance zone.
  \end{itemize}
\item
  \citet{Archak2011}:

  \begin{itemize}
  \tightlist
  \item
    How does the information embedded in product reviews the consumer
    choice?
  \item
    To study this, the authors estimate the discrete choice model of
    consumers in which the text information from the product reviews are
    included as the product attributes.
  \end{itemize}
\item
  \citet{Holmes2011}:

  \begin{itemize}
  \tightlist
  \item
    Wal-Mart maintain high store density. How large is the economy of
    density and the sales cannibalization?
  \item
    To study this, the author first estimate the demand function across
    neighborhood Wal-Mart to capture the sales cannibalization, and then
    estimate the cost structure from their entry and exit behaviors.
  \end{itemize}
\item
  \citet{Handbury2014}:

  \begin{itemize}
  \tightlist
  \item
    Urban and rural areas differ in available products. How does the
    price difference change if the heterogeneity in the product
    availability is incorporated?
  \item
    To do so, the authors estimate the demand function at each location,
    and the construct the spatial price index based on the available
    products at each location.
  \end{itemize}
\end{itemize}

\section{Analyzing Consumer
Behaviors}\label{analyzing-consumer-behaviors}

\begin{itemize}
\item
  \textbf{Alternative set}.
\item
  \textbf{Utility function}.

  \begin{itemize}
  \tightlist
  \item
    Add system of \textbf{choice sets}.
  \item
    Aad utility maximization.
  \end{itemize}
\item
  \(\rightarrow\) \textbf{Demand function}.
\item
  In case of producer behavior, there was a chance to directly observe
  the output of the most primitive function, the production function.
\item
  In case of consumer behavior, we never directly observe the output of
  the most primitive function, the utility function.
\item
  We can at most identify demand functions.
\item
  \textbf{Revealed preference theory}:

  \begin{itemize}
  \tightlist
  \item
    \citet{Samuelson1938}, \citet{Houthakker1950}, \citet{Richter1966},
    \citet{Afriat1967}, \citet{Varian1982}.
  \item
    If the demand function is derived from a preference by maximizing
    the preference, the demand function should satisfy some
    restrictions.
  \item
    If the assumption is true, we can recover part of the preference
    from the demand function.
  \end{itemize}
\end{itemize}

\section{Continuous Choice}\label{continuous-choice}

\begin{itemize}
\tightlist
\item
  The alternative set \(\mathcal{X}\) is a subset of \(\mathbb{R}^J\).
\item
  The utility function \(u\) is rational, monotone, and continuous on
  \(\mathcal{X}\).
\item
  The choice sets are given by a system of \textbf{linear budget set}:
  \[
  \mathcal{B}(p, x) = \{q \in \mathcal{X}: p \cdot q \le w\}.
  \]
\item
  If choice sets are non-linear, the following duality approach needs to
  be modified.
\end{itemize}

\subsection{Duality between Utility and Expenditure
Functions}\label{duality-between-utility-and-expenditure-functions}

\begin{itemize}
\tightlist
\item
  It is rather a special case that we can derive a closed form solution
  to a utility maximization problem.
\item
  We can use the first-order conditions as moment conditions for
  identification.

  \begin{equation}
  \frac{\partial u(q)}{\partial q_i} = \lambda p_i, i = 1, \cdots, J.
  \end{equation}
\item
  The derivation of a demand function from the identified utility
  function in general require a numerical simulation, which can be
  bothering.
\item
  As well as the duality between production and cost functions, we have
  the same duality theorem for utility and expenditure functions.
\item
  There is a one-to-one mapping between a class of utility functions and
  a class of expenditure functions.
\item
  Therefore, it is okay to start from an expenditure function.
\item
  It is rare that we can recover the utility function associated with an
  expenditure function in a closed form. But it is not often required
  for analysis.
\item
  Moreover, we can easily derive other important functions from the
  expenditure functions.
\item
  Let \(p\) be the price vector and \(u\) be the target utility level.
\item
  Let \(u(q)\) be a utility function.
\item
  An expenditure function associated with the utility function is
  defined by:

  \begin{equation}
  e(u, p) = \min_{q} p \cdot q, u(q) \ge u.
  \end{equation}
\item
  Let \(x\) be the total expenditure such that:

  \begin{equation}
  x = e(u, p).
  \end{equation}
\item
  We can start the analysis by specifying this function instead of the
  utility function.
\end{itemize}

\subsection{Deriving Other Functions}\label{deriving-other-functions}

\begin{itemize}
\tightlist
\item
  It is easy to derive other functions from an expenditure function.
\item
  \textbf{Indirect utility function}: invert the expenditure function to
  get:

  \begin{equation}
  u = e^{-1}(p, x) \equiv v(p, x).
  \end{equation}
\item
  \textbf{Hicksian demand function}: apply Shepard's lemma:

  \begin{equation}
  q_i = \frac{\partial e(u, p)}{\partial p_i} \equiv h_i(u, p).
  \end{equation}
\item
  \textbf{Marshallian demand function}: insert Hicksian demand function
  to the expenditure function:

  \begin{equation}
  q_i = h_i(v(p, x), p) \equiv d_i(p, x).
  \end{equation}
\end{itemize}

\subsection{Starting from an Indirect Utility
Function}\label{starting-from-an-indirect-utility-function}

\begin{itemize}
\tightlist
\item
  It is almost equivalent to start from an indirect utility function.
\item
  An indirect utility function with the utility function is defined by:

  \begin{equation}
  v(p, x) \equiv \max_{q} u(q), p'q \le x.
  \end{equation}
\item
  We can derive Marshallian demand function by Roy's identity:

  \begin{equation}
  q_i = \frac{- \partial v(p, x)/\partial p_i}{\partial v(p, x)/\partial x} \equiv d_i(p, x).
  \end{equation}
\end{itemize}

\subsection{Expenditure Share
Equation}\label{expenditure-share-equation}

\begin{itemize}
\tightlist
\item
  Let's start from an expenditure function \(e(p, x)\).
\item
  By Shepard's lemma, we have:

  \begin{equation}
  \frac{\partial \ln e(u, p)}{\partial \ln p_i} = \frac{\partial e(u, p)}{\partial p_i} \frac{p_i}{e(u, p)} = \frac{p_i q_i}{x} \equiv w_i.
  \end{equation}
\item
  We call this an expenditure share equation.
\item
  The estimation is based on the share equations.
\end{itemize}

\subsection{Almost Ideal Demand System
(AIDS)}\label{almost-ideal-demand-system-aids}

\begin{itemize}
\tightlist
\item
  Based on \citet{Deaton1980}.
\item
  See \citet{AngusDeaton1980} for further reference.
\item
  Consider an expenditure function that satisfies the following useful
  conditions:

  \begin{itemize}
  \tightlist
  \item
    It allows aggregation (this motivation is less important in recent
    days).
  \item
    It gives an arbitrary first-order approximation to any demand
    system.
  \item
    It can satisfy the restrictions of utility maximization.
  \item
    It can be used to test the restrictions of utility maximization.
  \end{itemize}
\end{itemize}

\subsection{PIGLOG Class}\label{piglog-class}

\begin{itemize}
\tightlist
\item
  PIGLOG (price-independent generalized logarithmic) class
  \citep{Muellbauer1976}.

  \begin{equation}
  \ln e(u, p) = (1 - u) \ln a(p) + u\ln b(p),
  \end{equation}

  where \(a(p)\) and \(b(p)\) are arbitrary linear homogeneous concave
  functions.
\item
  Consider households that differ in total income.
\item
  PIGLOC form ensures that the aggregate demand can be written in the
  same form where the total income is replaced with the sum of household
  total income.
\item
  The derivatives should be given free parameters for the model to be an
  arbitrary first-order approximation to any demand system.
\item
  In AIDS, we specify \(a(p)\) and \(b(p)\) as:

  \begin{equation}
  \begin{split}
  \ln a(p) &\equiv a_0 + \sum_{k} \alpha_k \ln p_k + \frac{1}{2}\sum_{k} \sum_{j} \gamma_{kj}^* \ln p_k \ln p_j\\
  \ln b(p) &\equiv \ln a(p) + \beta_0  \prod_{k} p_k^{\beta_k}.
  \end{split}
  \end{equation}
\end{itemize}

\subsection{Derive the Share Equation
I}\label{derive-the-share-equation-i}

\begin{itemize}
\tightlist
\item
  By Roy's identify, we can derive the associated share equation:

  \begin{equation}
  w_i \equiv \frac{\partial \ln e(u, p)}{\partial \ln p_i} = \alpha_i + \sum_{j} \gamma_{ij} + \beta_i u \beta_0 \prod_{k} p_k^{\beta_k},
  \end{equation}

  where

  \begin{equation}
  \gamma_{ij} = \frac{1}{2}(\gamma_{ij}^* + \gamma_{ji}^*).
  \end{equation}
\end{itemize}

\subsection{Derive the Share Equation
II}\label{derive-the-share-equation-ii}

\begin{itemize}
\tightlist
\item
  Insert indirect utility function \(u = v(p, x)\) to this to get:

  \begin{equation}
  w_i = \alpha_i + \sum_{j} \gamma_{ij} \ln p_j + \beta_i \ln \frac{x}{P},
  \end{equation}

  where

  \begin{equation}
  \ln P \equiv  \alpha_0 + \sum_{k} \alpha_k \ln p_k + \frac{1}{2} \sum_{j} \sum_{k} \gamma_{kj} \ln p_k \ln p_j.
  \end{equation}
\item
  \(P\) is a price index associated with the given preference.
\item
  With the specification of \citet{RichardStone1954}, it becomes
  \(\ln P = \sum_{j} x_j \ln p_j\).
\item
  It can be used as an approximation.
\end{itemize}

\subsection{Specify the Detail II}\label{specify-the-detail-ii}

\begin{itemize}
\tightlist
\item
  It can satisfy the restrictions of utility maximization.
\item
  It can be used to test the restrictions of utility maximization.

  \begin{itemize}
  \tightlist
  \item
    \(\sum_{j} x_j = 1\):

    \begin{equation}
    \sum_{j} \alpha_j = 1, \sum_{j} \gamma_{jk} = 0, \sum_{j} \beta_j = 0.
    \end{equation}
  \item
    \(e(u, p)\) is linear homogeneous in \(p\):

    \begin{equation}
    \sum_{j} \gamma_{ij} = 0.
    \end{equation}
  \item
    Symmetry:

    \begin{equation}
    \gamma_{ij} = \gamma_{ji}.
    \end{equation}
  \end{itemize}
\end{itemize}

\subsection{Estimation}\label{estimation}

\begin{itemize}
\tightlist
\item
  We can estimate parameters based on the share equations.
\item
  If we use aggregate data, the aggregate error term is correlated with
  the price vector.
\item
  Therefore, we need at least as many instrumental variables as the
  dimension of the price vector.
\item
  With valid instrumental variables, we can estimate the model with GMM.
\item
  If we use household-level data, the household-specific errors
  controlling for aggregate errors will not be correlated with the price
  vector if the price is determined in a competitive market.
\end{itemize}

\subsection{From Product Space Approach to Characteristics Space
Approach}\label{from-product-space-approach-to-characteristics-space-approach}

\begin{itemize}
\tightlist
\item
  The framework up to here is called \textbf{product space approach}
  because the utility has been defined over a product space.
\item
  When there are \(J\) goods, there are \(J^2\) parameters for prices.
\item
  One way to resolve this issue is to introduce a priori knowledge about
  the preference.
\item
  For example, we can introduce a priori segmentation with separability.
\item
  It is hard to evaluate the effect of introducing new product.
\item
  Again, we have to a priori decide which segment/product is similar to
  the new product.
\item
  This leads us to the \textbf{characteristics space approach}
  \citep{Lancaster1966, Muth1966}:

  \begin{itemize}
  \tightlist
  \item
    Consumption is an activity in which goods are inputs and in which
    the output is a collection of characteristics.
  \item
    Utility ranks collections of characteristics and only to rank
    collections of goods indirectly through the characteristics that
    they possesses.
  \item
    There are \(k = 1, \cdots, K\) activities.
  \item
    The activity \(y\) requires to consume \(x = A y\) products.
  \item
    The activity \(y\) generates \(z = B y\) characteristics.
  \item
    The budget constraint is \(p \cdot x \le 1\).
  \item
    The utility is defined over the characteristics \(u(z)\).
  \item
    The consumer's problem is: \[
    \max_y u(z)
    \] s.t. \[
    p \cdot x \le 1, x = Ay, z = By, x, y, z \ge 0.
    \]
  \end{itemize}
\item
  Then, only the dimension of characteristics matters and the value of
  new products can be evaluated by the contribution to the production of
  characteristics.
\item
  The early application includes \citet{Rosen1974},
  \citet{Muellbauer1974}, \citet{Gorman1980}.
\item
  The nonparametric analysis based on the reveals preference is
  \citet{Blow2008}.
\end{itemize}

\subsection{From Continuous Choice Approach to Discrete Choice
Approach}\label{from-continuous-choice-approach-to-discrete-choice-approach}

\begin{itemize}
\tightlist
\item
  The aggregate demand is a collection of choice across consumers and
  within consumers over time.
\item
  It makes sense to model individual choices and then aggregate rather
  than directly modeling the aggregate demand.
\item
  The resulting aggregate demand will satisfy restrictions that are
  consistent with the underlying consumer choice model.
\item
  If there is an interaction across choices, the aggregation is not
  trivial.
\item
  This is especially true when aggregating choices within consumers.
\item
  For now, assume that each choice is independent.
\end{itemize}

\section{Discrete Choice}\label{discrete-choice}

\subsection{Discrete Choice Approach}\label{discrete-choice-approach}

\begin{itemize}
\tightlist
\item
  Let \(u(q, z_i)\) be the utility of a consumer over \(J + 1\)
  dimensional consumption bundle \(q\) characterized by consumer
  characteristics \(z_i\).
\item
  The consumer solves:

  \begin{equation}
  V(p, y_i, z_i) = \max_{q}u(q, z_i), \text{   s.t.   } p'q \le y_i.
  \end{equation}
\item
  Alternative \(0\) is an \textbf{outside good}.
\item
  Normalize \(p_0 = 1\).
\item
  We call alternatives \(j = 1, \cdots, J\) \textbf{inside goods}.
\item
  The choice space is restricted on:

  \begin{equation}
  \begin{split}
  Q = \{q:& q_0 \in [0, M], q_j \in \{0, 1\}, j = 1, \cdots, J,\\
  & q_j q_k = 0, \forall j \neq k, j, k > 0, M < \infty\}.
  \end{split}
  \end{equation}
\end{itemize}

\subsection{Discrete Choice Approach}\label{discrete-choice-approach-1}

\begin{itemize}
\tightlist
\item
  The budget constraint reduces to:

  \begin{equation}
  \begin{cases}
  q_0 + p_j q_j = y &\text{   if   } q_j = 1, j > 0\\
  q_0 = y &\text{   otherwise}.
  \end{cases}
  \end{equation}
\item
  Hence,

  \begin{equation}
  q_0 = y - \sum_{j = 1}^J p_j q_j.
  \end{equation}
\end{itemize}

\subsection{Discrete Choice Approach}\label{discrete-choice-approach-2}

\begin{itemize}
\tightlist
\item
  The utility maximization problem can be written as:

  \begin{equation}
  V(p, y_i, z_i) = \max_{j = 0, 1, \cdots, j}  v_j(p_j, y_i, z_i),
  \end{equation}

  where

  \begin{equation}
  \begin{split}
  &v_j(p_j, y_i, z_i)\\
  & =
  \begin{cases}
  u(y_i - p_j, 0, \cdots, \underbrace{1}_{q_j}, \cdots, 0, z_i) &\text{   if  }j > 0,\\
  u(y_i, 0, \cdots, 0, z_i) &\text{   if   }j = 0,
  \end{cases}
  \end{split}
  \end{equation}

  is called the \textbf{choice-specific indirect utility}.
\end{itemize}

\subsection{Characteristics Space
Approach}\label{characteristics-space-approach}

\begin{itemize}
\item
  Preference is defined over the characteristics of alternatives,
  \(x_j\):
\item
  Car: vehicle, engine power, model-year, car maker, etc.
\item
  PC: CPU power, number of cores, memory, HDD volume, etc.
\item
  The choice-specific indirect utility is a function of the
  characteristics of the alternative:

  \begin{equation}
  \begin{split}
  v_j(p_j, y_i, z_i) &=u(y_i - p_j, 0, \cdots, \underbrace{1}_{q_j}, \cdots, 0, z_i)\\
  &= u^*(y_i - p_j, x_j, z_i)\\
  &\equiv v(p_j, x_j, y_i, z_i).
  \end{split}
  \end{equation}
\end{itemize}

\subsection{Weak Separability and Income
Effect}\label{weak-separability-and-income-effect}

\begin{itemize}
\tightlist
\item
  We usually focus on a particular product category such as cars, PCs,
  cereals, detergents, and so on.
\item
  Assume that the preference is separable between the category in
  question and other categories.
\item
  Then, call the category in question inside goods.
\item
  The outside good captures the income effect of the choice in this
  category.
\end{itemize}

\subsection{Weak Separability and Income
Effect}\label{weak-separability-and-income-effect-1}

\begin{itemize}
\tightlist
\item
  Thus, how the preference for the outside good is modeled determines
  how the individual income affects the choice.

  \begin{equation}
  \begin{split}
  &u^*(y_i - p_j, x_j, z_i) = u^{**}(x_j, z_i) + \alpha(y_i - p_j).\\
  &u^*(y_i - p_j, x_j, z_i) = u^{**}(x_j, z_i) + \alpha \ln (y_i - p_j).
  \end{split}
  \end{equation}
\item
  In the first example, the income level does not affect the choice
  because the term \(\alpha y_i\) is common and constant across choices
  (there is no income effect).
\item
  We often do not observe income of a consumer, \(y_i\).
\item
  Remember that the price of a product enters because we here consider
  \textbf{indirect} utility function.
\end{itemize}

\subsection{Utility Function
Normalization}\label{utility-function-normalization}

\begin{itemize}
\tightlist
\item
  The \textbf{location} of utility function is often normalized by
  setting:

  \begin{equation}
  u(y^*, 0, \cdots, 0, z^*) = 0,
  \end{equation}

  for certain choice of \((y^*, z^*)\).
\end{itemize}

\subsection{Aggregation of the Individual
Demand}\label{aggregation-of-the-individual-demand}

\begin{itemize}
\tightlist
\item
  Let \(q(p, x, y_i, z_i) = \{q_j(p, x, y_i, z_i)\}_{j = 0, \cdots, J}\)
  be the demand function of consumer \(i\), that is:

  \begin{equation}
  q_j(p, x, y_i, z_i) = 1 \Leftrightarrow j = \text{argmax}_{j = 0, 1, \cdots, j}  v(p_j, x_j, y_i, z_i).
  \end{equation}
\item
  Let \(f(y, z)\) be the joint distribution of the income and other
  consumer characteristics.
\item
  The aggregate demand for good \(j\) is:

  \begin{equation}
  D_j(p, x_j; f) \equiv N \int  q_j(p, x_j, y, z) f(y, z) dy dz,
  \end{equation}

  where \(N\) is the population.
\end{itemize}

\subsection{Horizontal Product
Differentiation}\label{horizontal-product-differentiation}

\begin{itemize}
\tightlist
\item
  \textbf{horizontal product differentiation}: consumers do not agree on
  the ranking of the choices.
\item
  There are two convenience stores \(j = 1, 2\) on a street \([0, 1]\).
\item
  Let \(z_i = L_i\), the location of consumer \(i\) and \(x_j = L_j\),
  the location of the choice on a street \([0, 1]\) with \(L_1 < L_2\).
\item
  A consumer has a preference such that:

  \begin{equation}
  v_{ij} \equiv v(p_j, x_j, y_i, z_i) \equiv s - t |L_i - L_j| - p_j.
  \end{equation}
\end{itemize}

\subsection{Horizontal Product
Differentiation}\label{horizontal-product-differentiation-1}

\begin{itemize}
\tightlist
\item
  Suppose that the prices are low enough that entire consumers on the
  street are willing to buy either from the stores.
\item
  Consumer \(i\) buys from store \(1\) if and only if:

  \begin{equation}
  \begin{split}
  &v(p_1, x_1, y_i, z_i) \ge v(p_2, x_2, y_i, z_i)\\
  &\Leftrightarrow s - t |L_i - L_1| - p_1 \ge s - t |L_i - L_2|- p_2\\
  &\Leftrightarrow L_i \le \frac{p_2 - p_1}{2 t} + \frac{L_1 + L_2}{2} \equiv \overline{L}_1(p_1, p_2).
  \end{split}
  \end{equation}
\item
  Let \(f(L_i)\) be \(U[0, 1]\). Then, the aggregate demand for store 1
  is:

  \begin{equation}
  \begin{split}
  \sigma_1(p, L_1, L_2) = N \int_{0}^{\overline{L}_1(p_1, p_2)} d L_i = N\overline{L}_1(p_1, p_2).
  \end{split}
  \end{equation}
\end{itemize}

\subsection{Vertical Product
Differentiation}\label{vertical-product-differentiation}

\begin{itemize}
\tightlist
\item
  \textbf{Vertical product differentiation}: Consumers agree on the
  ranking of the choices. Consumers can have different willingness to
  pay.
\item
  \citet{Bresnahan1987} analyzed automobile demand with this framework.
\item
  There are \(J\) goods and consumer \(i\) has a utility such as:

  \begin{equation}
  v_{ij} \equiv v(p_j, x_j, y_i, z_i) = z_i x_j - p_j,
  \end{equation}

  where \(x_j\) is a quality of product \(j\) and \(z_i\) is the
  consumer's willingness to pay for the quality with
  \(x_j < w_{j + 1}\).
\item
  Consumers' problem is:

  \begin{equation}
  \max\{0, z_i x_1 - p_1, \cdots, z_i x_j - p_J \}.
  \end{equation}
\end{itemize}

\subsection{Vertical Product
Differentiation}\label{vertical-product-differentiation-1}

\begin{itemize}
\tightlist
\item
  Consumer \(i\) prefers good \(j + 1\) to good \(j\) if and only if:

  \begin{equation}
  \begin{split}
  &v(p_{j + 1}, w_{j + 1}, y_i, z_i) \ge v(p_j, x_j, y_i, z_i)\\
  &\Leftrightarrow z_i w_{j + 1} - p_{j + 1} \ge z_i x_j - p_j\\
  &\Leftrightarrow z_i \ge \frac{p_{j + 1} - p_j}{w_{j + 1} - x_j} \equiv \Delta_j.
  \end{split}
  \end{equation}
\item
  So consumer \(i\) purchases good \(j\) if and only if
  \(z_i \in [\Delta_{j - 1}, \Delta_j)\) and buys nothing if:

  \begin{equation}
  z_i \le \Delta_0 \equiv \min\{p_1/x_1, \cdots p_J/x_j\}.
  \end{equation}
\item
  Letting \(F(z)\) be the distribution function of \(z\), the aggregate
  demand for good \(j\) is:

  \begin{equation}
  \sigma_j(p, x, z) = N[F(\Delta_{j}) - F(\Delta_{j - 1})].
  \end{equation}
\end{itemize}

\subsection{Econometric Models}\label{econometric-models}

\begin{itemize}
\tightlist
\item
  So far there was no econometrics.
\item
  Next we define what are observable and unobservable, and what are
  known and unknown.
\item
  Then consider how to identify and estimate the model.
\end{itemize}

\subsection{Multinomial Logit Model: Preference
Shock}\label{multinomial-logit-model-preference-shock}

\begin{itemize}
\tightlist
\item
  This originates at \citet{Mcfadden1974}.
\item
  See \citet{Train2009} for reference.
\item
  Suppose that there is some unobservable component in consumer
  characteristics.
\item
  In reality, consumers choice change somewhat randomly.
\item
  Let's capture such a \textbf{preference shock} by consider the
  following model:

  \begin{equation}
  v(p_j, x_j, y_i, z_i) + \epsilon_{ij},
  \end{equation}

  where I assumed that there is no other \(z_i\) for exposition
  simplicity, with some random vector:

  \begin{equation}
  \epsilon_i \equiv (\epsilon_{i0}, \cdots, \epsilon_{iJ})' \sim G.
  \end{equation}
\item
  At this point, \(G\) can be any distribution and the shocks can be
  dependent across \(j\) within \(i\).
\item
  \(p, x, y_i, z_i\) are \textbf{observed} but \(\epsilon_{ij}\) are
  \textbf{unobserved}.
\item
  When the realization of the preference shock is given, the consumer
  choice is: \[
  q_j(p, x, y_i, z_i, \epsilon_{i}) \equiv 1\{j = \text{argmax}_{k = 0, \cdots, J} v(p_k, x_k, y_i, z_i) + \epsilon_{ik}\}
  \] for \(k = 0, \cdots, J\).
\item
  The \textbf{choice probability} as observed by econometrician is: \[
  \sigma_j(p, x, y_i, z_i) \equiv \int q_j(p, x, y_i, z_i, \epsilon_{i}) dG(\epsilon_i).
  \]
\end{itemize}

\subsection{Multinomial Logit Model: Distributional
Assumption}\label{multinomial-logit-model-distributional-assumption}

\begin{itemize}
\item
  Now assume the followings:
\item
  \(\epsilon_{ij}\) are independent across \(j\):
  \(G(\epsilon_i) = \prod_{j = 0, \cdots, J} G_j(\epsilon_{ij})\).
\item
  \(\epsilon_{ij}\) are identical across \(j\):
  \(G_j(\epsilon_{ij}) = \overline{G}(\epsilon_{ij})\).
\item
  \(\overline{G}\) is a type-I extreme value.
\item
  \(\rightarrow\) The density
  \(g(\epsilon_{ij}) = \exp[-\exp(-\epsilon_{ij}) - \epsilon_{ij}]\).
\item
  This is called the (homoskedastic) \textbf{multinomial logit model}.
\item
  Setting the variance of \(\epsilon_{ij}\) at 1 for some \(j\) is a
  \textbf{scale} normalization.
\item
  By dropping some of the assumptions, we can have heteroskedastic
  multinomial logit model, generalized extreme value model, and so on.
\item
  Another popular distribution assumption is to assume a multivariate
  normal distribution of \(\epsilon_i\). This case is called the
  \textbf{multinomial probit model}.
\end{itemize}

\subsection{Multinomial Logit Model: Choice
Probability}\label{multinomial-logit-model-choice-probability}

\begin{itemize}
\tightlist
\item
  The \textbf{choice probability} of consumer \(i\) of good \(j\) is:

  \begin{equation}
  \begin{split}
  \sigma_j(p, x, y_i, z_i) & \equiv \mathbb{P}\{j = \text{argmax}_{k = 0, 1, \cdots, J} v(p, x_k, y_i, z_i) + \epsilon_{ik}  \}\\
  &=\mathbb{P}\{v(p, x_j, y_i, z_i) -  v(p_k, x_k, y_i, z_i) \ge \epsilon_{ik} - \epsilon_{ij}, \forall k \neq j\}\\
  & = \text{...after some algebra: leave as an exercise...}\\
  &= \frac{\exp[v(p_j, x_j, y_i, z_i) ]}{\sum_{k = 0}^J \exp[v(p_k, x_k, y_i, z_i)] }.
  \end{split}
  \end{equation}
\item
  For example, if:

  \begin{equation}
  v(p_k, x_k, y_i, z_i) = \beta_i'x_k + \alpha_i (y_i - p_k),
  \end{equation}
\end{itemize}

\begin{equation}
\begin{pmatrix}
\beta_i \\
\alpha_i
\end{pmatrix}
= 
\begin{pmatrix}
\beta_0 \\
\alpha_0
\end{pmatrix}
+
\begin{pmatrix}
\Gamma\\
\pi'
\end{pmatrix}
 z_i.
\end{equation}

\begin{itemize}
\tightlist
\item
  Then, we have:

  \begin{equation}
  \begin{split}
  \sigma_{j}(p, x, y_i, z_i) &= \frac{\exp[\beta_i'x_j + \alpha_i (y_i - p_j) ]}{\sum_{k = 0}^J \exp[\beta_i'x_k + \alpha_i (y_i - p_k) ]}\\
  &= \frac{\exp[\beta_i'x_j - \alpha_i p_j]}{\sum_{k = 0}^J \exp[\beta_i'x_k - \alpha_i p_k]}
  \end{split}
  \end{equation}
\item
  If we normalize the characteristics vector so that \(w_0 = 0\) holds
  for the outside option, it becomes: \[
  \sigma_{j}(p, x, y_i, z_i) = \frac{\exp[\beta_i'x_j - \alpha_i p_j]}{1 + \sum_{k = 1}^J \exp[\beta_i'x_k - \alpha_i p_k]}
  \]
\end{itemize}

\subsection{Multinomial Logit Model: Inclusive
Value}\label{multinomial-logit-model-inclusive-value}

\begin{itemize}
\tightlist
\item
  The expected utility for consumer \(i\) before the preference shocks
  are drawn under multinomial logit model is given by:

  \begin{equation}
  \begin{split}
  &\mathbb{E}\{\max_{j = 0, \cdots, J} v(p_j, x_j, y_i, z_i) + \epsilon_{ij}\} \\
  &= \text{   ...after some algebra: leave as an exercise...}\\
  &= \ln \Bigg\{\sum_{j = 0}^J \exp[v(p_j, x_j, y_i, z_i)] \Bigg\} + constant.
  \end{split} 
  \end{equation}
\item
  This is sometimes called the \textbf{inclusive value} of the choice
  set.
\end{itemize}

\subsection{Maximum Likelihood Estimation of Multinomial Logit
Model}\label{maximum-likelihood-estimation-of-multinomial-logit-model}

\begin{itemize}
\tightlist
\item
  Suppose we observe a sequence of income \(y_i\), consumer
  characteristics \(z_i\), choice \(q_{i}\), product characteristics
  \(x_j\) and price \(p_j\).
\item
  \(q_i = (q_{i0}, \cdots, q_{iJ})'\) and \(q_{ij} = 1\) if \(j\) is
  chosen and \(0\) otherwise.
\item
  The parameter of interest is the mean indirect utility function \(v\).
\item
  Then the log likelihood of \(\{q_i\}_{i = 1}^N\) conditional on
  \(\{y_i, z_i\}_{i = 1}^N\) and \(\{x_j,p_j\}_{j = 1}^J\) is:

  \begin{equation}
  \begin{split}
  l(v; q, y, z, w) &= \sum_{i = 1}^N \ln \mathbb{P}\{q_i = q(p, x, y_i, z_i)|p, x, y_i, z_i\}\\
  & = \sum_{i = 1}^N \log \Bigg\{ \prod_{j = 0}^{J} \sigma_{j}(p, x, y_i, z_i)^{q_{ij}} \Bigg\}\\
  &= \sum_{i = 1}^N \sum_{j = 0}^J \log \sigma_{j}(p, x, y_i, z_i)^{q_{ij}}.
  \end{split}
  \end{equation}
\item
  We can estimate the parameters by finding parameters that maximize the
  log likelihood.
\end{itemize}

\subsection{Nonlinear Least Square Estimation of Multinomial Logit
Model}\label{nonlinear-least-square-estimation-of-multinomial-logit-model}

\begin{itemize}
\tightlist
\item
  The multinomial logit model can be estimated by nonlinear least square
  method as well.
\item
  Suppose that the share of product \(j\) among consumers with
  characteristics \(z\) and income \(y\) was:

  \begin{equation}
  \sigma_j(p, x, y, z).
  \end{equation}
\item
  Note that:

  \begin{equation}
  \begin{split}
  \ln \sigma_{j}(p, x, y, z)  &= \ln \Bigg\{ \frac{\exp[v(p_j, x_j, y, z) ]}{\sum_{k = 0}^J \exp[v(p_k, x_k, y, z)] }  \Bigg\}\\
  &= v(p_j, x_j, y, z) - \ln\Bigg\{ \sum_{k = 0}^J \exp[v(p_k, x_k, y, z)]  \Bigg\}.
  \end{split}
  \end{equation}
\item
  Moreover, because of the location normalization of the utility
  function,

  \begin{equation}
  \sigma_{0}(p, x, y, z) = \frac{1}{\sum_{k = 0}^J \exp[v(p_j, x_k, y, z)] }.
  \end{equation}
\item
  Hence,

  \begin{equation}
  \ln \sigma_{j}(p, x, y, z) - \ln \sigma_{0}(p, x, y, z) = v(p, x_j, y, z).
  \end{equation}
\item
  The left-hand variables are observed in the data.
\item
  Let \(s_j(y, z)\) be the share of product \(j\) among consumers with
  characteristics \(z\) and income \(y\) \textbf{in the data}.
\item
  This can be calculated from the consumer-level data.
\item
  More importantly, if there is the total sales data for each
  demographic, we can use this approach.
\item
  Then, we can estimate the parameter by NLLS such that:

  \begin{equation}
  \min \sum_{(y, z)} \sum_{j = 1}^J \{\ln[s_{j}(y, z)/s_{0}(y, z)] - v(p_j, x_j, y, z)\}^2.
  \end{equation}
\item
  If \(v\) is linear in parameter, it is the ordinal least squares:

  \begin{equation}
  v(p_j, x_j, y_m) = \beta_i' x_j - \alpha_i p_j.
  \end{equation}
\end{itemize}

\begin{equation}
\ln[s_{j}(y, z)/s_{0}(y, z)] = \beta_i' x_j - \alpha_i p_j.
\end{equation}

\subsection{IIA Problem}\label{iia-problem}

\begin{itemize}
\tightlist
\item
  Multinomial logit problem is intuitive and easy to implement.
\item
  However, there are several problems in the model.
\item
  The most important problem is the
  \textbf{independence of irrelevant alternatives (IIA)} problem.
\item
  Notice that:

  \begin{equation}
  \frac{\sigma_j(p, x, y, z)}{\sigma_{k}(p, x, y, z)} = \frac{\exp[v(p_j, x_j, y, z)]}{\exp[v(p_k, x_k, y, z)]}.
  \end{equation}
\item
  The ratio of choice probabilities between two alternatives depend only
  on the mean indirect utility of these two alternatives and
  \textbf{independent of irrelevant alternatives (IIA)}.
\item
  Why is this a problem?
\end{itemize}

\subsection{Blue Bus and Red Bus
Problem}\label{blue-bus-and-red-bus-problem}

\begin{itemize}
\tightlist
\item
  Suppose that you can go to a town by bus or by train.
\item
  Half of commuters use a bus and the other half use a train.
\item
  The existing bus was blue. Now, the county introduced a red bus, which
  is identical to the existing blue bus.
\item
  No one take care of the color of bus. So the mean indirect utility of
  blue bus and red bus are equal.
\item
  What is the new share across blue bus, red bus, and train?
\item
  IIA \(\to\) share of blue bus = share of train.
\item
  Buses are identical \(\to\) share of blue bus = share of red bus.
\item
  Therefore, shares have to be 1/3, respectively.
\item
  But shouldn't it be that train keeps half share and bus have half
  share in total?
\end{itemize}

\subsection{Restrictive Price
Elasticity}\label{restrictive-price-elasticity}

\begin{itemize}
\tightlist
\item
  IIA property restrict price elasticities in an unfavorable manner.
\item
  This is a serious problem because the main purpose for us to estimate
  demand functions is to identify the price elasticity.
\item
  Let \(v(p_j, x_j, y, z) = \beta_z'x_j - \alpha_z p_j\). Then, we have:

  \begin{equation}
  e_{jk} =
  \begin{cases}
  -\alpha p_{j} (1 - \sigma_j(p, x, y, z)) &\text{   if   } k = j\\
  \alpha p_{k} \sigma_k(p_k, x_k, y, z) &\text{   if   } k \neq j.
  \end{cases}
  \end{equation}
\item
  The price elasticity is completely determined by the existing choice
  probabilities of the relevant alternatives.
\item
  Suppose that there are coca cola, Pepsi cola, and a coffee.
\item
  The shares were 1/2, 1/6, 1/3, respectively.
\item
  Suppose that the price of coca cola increased.
\item
  We expect that they instead purchase Pepsi cola because Pepsi cola is
  more similar to coca cola than coffee.
\item
  However, according to the previous result, twice more consumers
  substitute to coffee rather than to Pepsi cola.
\end{itemize}

\subsection{Monotonic Inclusive Value}\label{monotonic-inclusive-value}

\begin{itemize}
\tightlist
\item
  Suppose that there is a good whose mean indirect utility is \(v\).
\item
  The inclusive value for this choice set is \(\ln[1 + \exp(v)]\).
\item
  Suppose that we put \(J\) same goods on the shelf and consumer can
  choose any of them.
\item
  The inclusive value is \(\ln[1 + J \exp(v)]\).
\item
  We just added the same goods. But the expected utility of consumer
  increases monotonically in the number of alternatives.
\end{itemize}

\subsection{The Source of the Problem}\label{the-source-of-the-problem}

\begin{itemize}
\tightlist
\item
  \textbf{The source of the problem is that there is no correlation in
  the preference shock across products}.
\item
  When the preference shock to coca cola is high, the preference shock
  to Pepsi cola should be high, while the preference shock to coffee
  should be relatively independent.
\item
  Because the expected value of the maximum of the preference shocks
  increases according to the number of alternatives, the inclusive value
  becomes increasing in the number of alternatives.
\item
  However, the preference shocks should be the same for the same good.
  Then, the the expected value of the maximum of the preference shock
  should not increase even if we add the same products on the shelf.
\end{itemize}

\subsection{Correlation in Preference
Shocks}\label{correlation-in-preference-shocks}

\begin{itemize}
\tightlist
\item
  Therefore, the preference shock should be such that: preference shocks
  between two alternative should be more correlated when they are closer
  in the characteristics space.
\item
  So we have to allow the covariance matrix of the preference shock to
  be free parameters.
\item
  If we allow flexible covariance matrix, the curse of dimensionality in
  the number of alternatives comes back: The dimensionality of the
  covariance matrix is \(J^2\).
\item
  Another way is to remove \(\epsilon_{ij}\): it is called a
  \textbf{pure characteristics model} \citep{Berry2007}.
\item
  But the pure characteristics model is computationally not
  straightforward.
\item
  We explore the way of introducing mild correlation across similar
  products in the preference shocks.
\end{itemize}

\subsection{Observed and Unobserved Consumer
Heterogeneity}\label{observed-and-unobserved-consumer-heterogeneity}

\begin{itemize}
\tightlist
\item
  Consider beverage demand and let \(x_j = \text{carbonated}_j\) and
  \(z_i = \text{teenager}_i\).
\item
  Suppose that the mean indirect utility is:

  \begin{equation}
  v(p_j, x_j, y_i, z_i) = \beta_i (\text{carbonated})_j - \alpha_i p_j,
  \end{equation}
\end{itemize}

\begin{equation}
\beta_i = 0.1 + 0.2 \cdot (\text{carbonated})_i.
\end{equation}

\begin{itemize}
\item
  The mean utility of a carbonated drink for a teenager is 0.3 but only
  0.1 for others.
\item
  When coca cola was not available, teenager will substitute more to
  Pepsi cola than non-teenagers.
\item
  IIA holds at the market-segment level but not at the market level.
\item
  How to avoid IIA at the market-segment level?: Introduce unobserved
  consumer heterogeneity.
\item
  Suppose that the mean indirect utility is:

  \begin{equation}
  \beta_i = 0.1 + 0.2 \cdot (\text{carbonated})_i + \nu_i.
  \end{equation}
\item
  Consumers with high \(\nu_i\) values carbonated drinks more than those
  with low \(\nu_i\) values.
\item
  When coca cola was not available, consumers with high \(\nu_i\) will
  substitute more to Pepsi cola than those with low \(\nu_i\) values.
\item
  IIA holds at the market-segment-\(\nu\) level but not at the
  market-segment level.
\item
  In the above example, ``\(0.2 \cdot (\text{carbonated})_i\)'' captures
  the consumer heterogeneity by observed characteristics and
  ``\(\nu_i\)'' by unobserved characteristics.
\end{itemize}

\subsection{Mixed Logit Model}\label{mixed-logit-model}

\begin{itemize}
\tightlist
\item
  Suppose that the mean indirect utility is:

  \begin{equation}
  v(p_j, x_j, y_i, z_i, \beta_i, \alpha_i) = \beta_i' x_j - \alpha_i p_j,
  \end{equation}

  with

  \begin{equation}
  (\beta_i, \alpha_i) \sim f(\beta_i, \alpha_i|y_i, z_i).
  \end{equation}
\item
  If \(\epsilon_{ij}\) is drawn i.i.d. from type-I extreme value
  distribution, the choice probability of good \(j\) by consumer \(i\)
  conditional on \(p, x, y_i, z_i\) is:

  \begin{equation}
  s_{j}(p, x, y_i, z_i) = \int_{\beta_i, \alpha_i} \frac{\exp[v(p_j, x_j, y_i, z_i, \beta_i, \alpha_i)]}{\sum_{k = 0}^J \exp[v(p_j, x_j, y_i, z_i, \beta_i, \alpha_i)]} f(\beta_i, \alpha_i|y_i, z_i) d\beta_i d\alpha_i.
  \end{equation}
\item
  This is called the \textbf{mixed-logit model}.
\item
  If the distribution of \(\epsilon_{ij}\) is different, it is no longer
  mixed logit.
\item
  Conditional on \((\beta_i, \alpha_i)\) the choice probability is
  written in the same way with the multinomial logit model.
\item
  \(\beta_i, \alpha_i\) are marginal out, because econometrician does
  not observe them.
\end{itemize}

\subsection{Mixed Logit Model : Parametric
Assumptions}\label{mixed-logit-model-parametric-assumptions}

\begin{itemize}
\tightlist
\item
  It is often assumed that:

  \begin{equation}
  v(p_j, x_j, y_i, z_i, \beta_i, \alpha_i) = \beta_i' x_j - \alpha_i p_j.
  \end{equation}
\item
  \citet{Mcfadden2000} showed that any discrete choice models that are
  consistent with the random utility maximization can be arbitrarily
  closely approximated by this class of mixed-logit model.
\item
  The distribution of \(\beta_i\) and \(\alpha_i\) is often assumed to
  be:

  \begin{equation}
  \begin{split}
  &\beta_i = \beta_0 + \Gamma z_i + \Sigma \nu_i,\\
  &\alpha_i = \alpha_0 + \pi' z_i + \Omega \upsilon_i,
  \end{split}
  \end{equation}

  where \(\nu_i\) and \(\upsilon_i\) are i.i.d. standard normal random
  vectors.
\end{itemize}

\subsection{Mixed Logit Model: IIA}\label{mixed-logit-model-iia}

\begin{itemize}
\tightlist
\item
  There is no IIA at the market-segment level:

  \begin{equation}
  \frac{\sigma_{j}(p, x, y, z)}{\sigma_{l}(p, x, y_i, z_i)} = \frac{\int_{\beta_i, \alpha_i} \frac{\exp[v(p_j, x_j, y_i, z_i, \beta_i, \alpha_i)]}{\sum_{k = 0}^J \exp[v(p_k, x_k, y_i, z_i, \beta_i, \alpha_i)]} f(\beta_i, \alpha_i|y_i) d\beta_i d\alpha_i}{\int_{\beta_i, \alpha_i} \frac{\exp[v(p_l, x_l, y_i, z_i, \beta_i, \alpha_i)]}{\sum_{k = 0}^J \exp[v(p_k, x_k, y_i, z_i, \beta_i, \alpha_i)]} f(\beta_i, \alpha_i|y_i) d\beta_i d\alpha_i}.
  \end{equation}
\item
  The share ratio depends on the price and characteristics of all the
  other products.
\end{itemize}

\subsection{Mixed Logit Moel: Price
Elasticities}\label{mixed-logit-moel-price-elasticities}

\begin{itemize}
\tightlist
\item
  Let:

  \begin{equation}
  v(p_j, x_j, y_i, z_i, \beta_i, \alpha_i) = \beta_i' x_j - \alpha_i p_j.
  \end{equation}
\item
  The price elasticities of the choice probabilities conditional on
  \(p, x, y_i, z_i\) is:

  \begin{equation}
  e_{jk} = 
  \begin{cases}
  -\frac{p_j}{\sigma_j} \int \alpha_i \sigma_{ij}(1 - \sigma_{ij})f(\beta_i, \alpha_i|y_i, z_i) d\beta_i d\alpha_i &\text{   if   } j = k\\
  \frac{p_k}{\sigma_j} \int \alpha_i \sigma_{ij} \sigma_{ik} f(\beta_i, \alpha_i|y_i, z_i) d\beta_i d\alpha_i &\text{   otherwise},
  \end{cases} 
  \end{equation}

  where

  \begin{equation}
  \sigma_{ij} = \frac{\exp(\beta_i'x_j - \alpha_i p_j)}{\sum_{k = 0}^J \exp(\beta_i'x_k - \alpha_i p_k)}.
  \end{equation}
\item
  The price elasticity depends on the density of unobserved consumer
  types.
\end{itemize}

\subsection{Simulated Maximum Likelihood Estimation of the Mixed Logit
Model}\label{simulated-maximum-likelihood-estimation-of-the-mixed-logit-model}

\begin{itemize}
\tightlist
\item
  The choice probability of the mixed logit model is an integration of
  the multinomial logit choice probability.
\item
  This is not derived analytically in general.
\item
  We can use simulation to evaluate the choice probability:
\item
  Draw \(R\) values of \(\beta\) and \(\alpha\),
  \(\{\beta^r, \alpha^r \}_{r = 1}^R\).
\item
  Compute the multinomial choice probabilities associated with
  \((\beta^r, \alpha^r)\) for each \(r = 1, \cdots, R\).
\item
  Approximate the choice probability with the mean of the simulated
  multinomial choice share:

  \begin{equation}
  \sigma_{j}(p, x, y_i, z_i) \approx \hat{\sigma}_{j}(p, x, y_i, z_i) \equiv \frac{1}{R} \sum_{r = 1}^R  \frac{\exp[v(p_j, x_j, y_i, z_i, \beta^r, \alpha^r)]}{\sum_{k = 0}^J \exp[v(p_k, x_k, y_i, z_i, \beta^r, \alpha^r)]}.
  \end{equation}
\item
  This is one of the numerical integration: \textbf{Monte Carlo
  integration}.
\item
  Another approach is to use \textbf{quadrature}. See \citet{Judd1998}
  for reference.
\end{itemize}

\subsection{Simulated Maximum Likelihood Estimation of the Mixed Logit
Model}\label{simulated-maximum-likelihood-estimation-of-the-mixed-logit-model-1}

\begin{itemize}
\tightlist
\item
  There are \(t = 1, \cdots, T\) markets and there \(i = 1, \cdots, N\)
  consumers in each market.
\item
  Let \(\mathcal{J}_t\) be the set of products that are available in
  market \(t\).
\item
  Suppose that we observe income \(y_{it}\), characteristics \(z_{it}\),
  and choice \(q_{it}\) for each consumer in a market.
\item
  Suppose that we observe product characteristics \(x_{jt}\) and price
  \(p_{jt}\) of each product in each market.
\item
  The simulated conditional log likelihood is:

  \begin{equation}
  \begin{split}
  &\sum_{i = 1}^N \sum_{t = 1}^T \ln \mathbb{P}\{q_{it}\\
  &= q(p_t, w_t, y_{it}, z_{it})|p_t, w_t, y_{it}, z_{it}\} \approx \sum_{i = 1}^N \ln \Bigg\{ \prod_{j \in \mathcal{J}_t \cup \{0\}} \hat{\sigma}_{j}(p_t, w_t, y_{it}, z_{it})^{q_{itj}} \Bigg\}.
  \end{split}
  \end{equation}
\item
  We find parameters that maximize the simulated conditional log
  likelihood.
\end{itemize}

\subsection{Simulated Non-linear Least Square Estimation of the Mixed
Logit
Model}\label{simulated-non-linear-least-square-estimation-of-the-mixed-logit-model}

\begin{itemize}
\tightlist
\item
  Suppose that we only know the sales or share at the market-segment
  level.
\item
  That is, we only observe the share of product \(j\) in market \(t\)
  among consumers of characteristics \(z\) and income \(y\),
  \(s_{jt}(y, z)\).
\item
  Then we can estimate the parameter by:

  \begin{equation}
  \min \sum_{t = 1}^T \sum_{j \in \mathcal{J}_t \cup \{0\}} \sum_{(y, z) \in \mathcal{Y} \times \mathcal{Z}} \{s_{jt}(y, z) - \hat{\sigma}_{j}(p_t, w_t, y, z)\}^2.
  \end{equation}
\end{itemize}

\subsection{Nested Logit Model: A Special Case of Mixed Logit
Model}\label{nested-logit-model-a-special-case-of-mixed-logit-model}

\begin{itemize}
\tightlist
\item
  Let \(w_{j1}, \cdots, w_{jG}\) be the indicator of product category,
  i.e., \(w_{jg}\) takes value 1 if good \(j\) belong to category \(g\)
  and 0 otherwise.
\item
  e.g., car category = \{Sports, Luxury, Large, Midsize, Small\}.
\item
  We have:

  \begin{equation}
  v(p, x_j, y_i, z_i) = \beta'x_j - \alpha_i p_j + \sum_{g = 1}^G \zeta_{ig} w_{jg} + \epsilon_{ij}.
  \end{equation}
\item
  If \(\zeta_{it}\) takes high value, the consumer attaches higher value
  to the category.
\item
  When a product in category \(g\) was not available, consumers with
  high \(\zeta_{ig}\) will substitute more to the other products in the
  same category than consumers with low \(\zeta_{ig}\).
\end{itemize}

\subsection{Nested Logit Model: Distributional
Assumption}\label{nested-logit-model-distributional-assumption}

\begin{itemize}
\tightlist
\item
  Let

  \begin{equation}
  \varepsilon_{ij} \equiv \sum_{g = 1}^G \zeta_{ig} w_{jg} + \epsilon_{ij}.
  \end{equation}
\item
  Under certain distributional assumption on \(\zeta_{ig}\) and
  \(\epsilon_{ij}\), the term \(\varepsilon_{ij}\) have a cumulative
  distribution {[}I forgot the reference! Let me know if you find it{]}:

  \begin{equation}
  F(\varepsilon_i) = \exp\Bigg\{- \sum_{g = 1}^G \Bigg(\sum_{j \in \text{   category   } g} \exp[-\varepsilon_{ij}/\lambda_g] \Bigg)^{\lambda_g}  \Bigg\}.
  \end{equation}
\end{itemize}

\subsection{Nested Logit Model: Choice
Probability}\label{nested-logit-model-choice-probability}

\begin{itemize}
\tightlist
\item
  Under this distributional assumption, the choice probability is:

  \begin{equation}
  \sigma_{j}(p, x, y_i, z_i) = \frac{\exp[v(p, x_j, y_i, z_i)/\lambda_g] \Bigg(\sum_{k \in \text{   category   } g} \exp[v(p, x_k, y_i, z_i)/\lambda_g]\Bigg)^{\lambda_g - 1}}{\sum_{g = 1}^G \Bigg(\sum_{k \in \text{   category   } g} \exp[v(p, x_k, y_i, z_i)/\lambda_g]\Bigg)^{\lambda_g}},
  \end{equation}

  if good \(j\) belongs to category \(g\).
\item
  The higher \(\lambda_g \in [0, 1]\) implies lower correlation within
  category \(g\).
\item
  \(\lambda_g = 1\) for all \(g\) coincides with the multinomial logit
  model.
\end{itemize}

\subsection{Nested Logit Model: Decomposition of the Choice
Probability}\label{nested-logit-model-decomposition-of-the-choice-probability}

\begin{itemize}
\tightlist
\item
  The choice probability can be decomposed into two parts:

  \begin{equation}
  \sigma_{j}(p, x, y_i, z_i) = \frac{\exp[v(p, x_j, y_i, z_i)/\lambda_g]}{\sum_{k \in \text{   category   } g} \exp[v(p, x_k, y_i, z_i)/\lambda_g]} \frac{\sum_{k \in \text{   category   } g} \exp[v(p, x_k, y_i, z_i)/\lambda_g]^{\lambda_g}}{\sum_{g = 1}^G \Bigg(\sum_{k \in \text{   category   } g} \exp[v(p, x_k, y_i, z_i)/\lambda_g]\Bigg)^{\lambda_g}}.
  \end{equation}
\item
  Letting: \[
  I_{g}(p, x, y_i, z_i) \equiv \log \sum_{k \in \text{   category   } g} \exp[v(p, x_k, y_i, z_i)/\lambda_g],
  \] we have:

  \begin{equation}
  \sigma_{j}(p, x, y_i, z_i) = \frac{\exp[v(p, x_j, y_i, z_i)/\lambda_g]}{\sum_{k \in \text{   category   } g} \exp[v(p, x_k, y_i, z_i)/\lambda_g]} \frac{\exp[\lambda_g I_{g}(p, x, y_i, z_i)]}{\sum_{g = 1}^G \exp[\lambda_g I_{g}(p, x, y_i, z_i)]}.
  \end{equation}
\item
  The second first term can be interpreted as the probability of
  choosing product \(j\) conditional on choosing category \(g\) and the
  second term as the probability of choosing category \(g\).
\end{itemize}

\subsection{Discrete Choice Model with Unobserved Fixed
Effects}\label{discrete-choice-model-with-unobserved-fixed-effects}

\begin{itemize}
\tightlist
\item
  We have assumed that good \(j\) is characterized by a vector of
  observed characteristics \(x_j\).
\item
  Can econometrician observe all the relevant characteristics of the
  products in the choice set? Maybe no. For example, econometrician may
  not observe brand values that are created by advertisement and
  recognized by consumers.
\item
  Such unobserved product characteristics is likely to be correlated
  with the price.
\item
  This can cause \textbf{endogeneity problems}.
\item
  In the following, we consider the situation where only market-segment
  level share data is available.
\item
  Because we can construct the market-share level data from individual
  choice level data, all the arguments should go through with the
  individual choice level data.
\end{itemize}

\subsection{Unobserved Fixed Effects in Multinomial Logit
Model}\label{unobserved-fixed-effects-in-multinomial-logit-model}

\begin{itemize}
\tightlist
\item
  To fix the idea, let's revisit the multinomial logit model.
\item
  For now, we do not consider either observed or unobserved consumer
  heterogeneity.
\item
  Including observed heterogeneity is straightforward.
\item
  We discuss how to include unobserved heterogeneity in the subsequent
  sections.
\item
  Suppose that the indirect utility function of good \(j\) for consumer
  \(i\) in market \(t\) is:

  \begin{equation}
  \beta' x_{jt}  - \alpha p_{jt} - \xi_{jt} + \epsilon_{ik},
  \end{equation}
\item
  \(\epsilon_{ik}\) is i.i.d. Type-I extreme value.
\item
  \(\xi_{jt}\) is the \emph{unobserved product-market-specific fixed
  effect} of product \(j\) in market \(t\), which can be correlated with
  \(p_{jt}\).
\item
  We hold the assumption that \(x_{jt}\) is uncorrelated with
  \(\xi_{jt}\).
\item
  The choice probability of good \(j\) for this consumer and hence the
  choice share in this market is:
\end{itemize}

\begin{equation}
\sigma_j(p_t, w_t, \xi_t) = \frac{\exp(\beta' x_j - \alpha p_{jt} + \xi_{jt})}{1 + \sum_{k = 1}^J\exp(\beta' x_k - \alpha p_{kt} +  \xi_{kt} ) }.
\end{equation}

\begin{itemize}
\tightlist
\item
  How to deal with the endogeneity between \(p_{jt}\) and \(\xi_{jt}\)?
\end{itemize}

\subsection{Instrumental Variables and
Inversion}\label{instrumental-variables-and-inversion}

\begin{itemize}
\tightlist
\item
  Suppose that we have a vector of instrumental variables \(w_{jt}\)
  such that:

  \begin{equation}
  \mathbb{E}\{\xi_{jt}|w_{jt}\} = 0.
  \end{equation}
\item
  In a liner model, we \textbf{invert} the model for the unobserved
  fixed effects:

  \begin{equation}
  \xi_{jt} = y_{jt} - \beta'x_{jt},
  \end{equation}
\item
  Notice that the unobserved fixed effect is written as a function of
  parameters and data.
\item
  Then we exploit the moment condition by:

  \begin{equation}
  \begin{split}
  &\mathbb{E}\{\xi_{jt}|w_{jt}\} = 0,\\
  &\Rightarrow \mathbb{E}\{ \xi_{jt} w_{jt}\} = 0,\\
  &\Leftrightarrow \mathbb{E}\{(y_{jt} - \beta'x_{jt}) w_{jt} \} = 0
  \end{split}
  \end{equation}
\item
  We can estimate \(\beta\) by finding the value that makes the sample
  analogue of the above expectation zero.
\end{itemize}

\subsection{Inversion in Multinomial Logit
Model}\label{inversion-in-multinomial-logit-model}

\begin{itemize}
\tightlist
\item
  Can we invert the multinomial model for \(\xi_{jt}\)?
\item
  We have:

  \begin{equation}
  \begin{split}
  &\ln [\sigma_{jt}(p_t, x_t, \xi_t) / \sigma_{0t}(p_t, x_t, \xi_t)] = \beta' x_j - \alpha p_{jt} + \xi_{jt}\\
  &\Leftrightarrow \xi_{jt} = \ln [\sigma_j(p_t, x_t, \xi_t) / \sigma_0(p_t, x_t, \xi_t)] - [\beta' x_j - \alpha p_{jt}].
  \end{split}
  \end{equation}
\item
  Therefore, the moment condition can be written as:

  \begin{equation}
  \begin{split}
  &\mathbb{E}\{\xi_{jt}|w_{jt}\} = 0,\\
  &\Rightarrow \mathbb{E}\{\xi_{jt} w_{jt}\} = 0,\\
  &\Leftrightarrow \mathbb{E}\{(\ln [\sigma_{jt}(p_t, x_t, \xi_t) / \sigma_{0t}(p_t, x_t, \xi_t)] - [\beta' x_j - \alpha p_{jt}]) w_{jt}  \} = 0.
  \end{split}
  \end{equation}
\item
  We can evaluate the sample analogue of the expectation by replacing
  the theoretical choice probability \(\sigma\) with the observed share
  \(s\).
\item
  At the end, it is no different from the linear model where the
  dependent variable is \(\ln s_{jt}/s_{0t}\).
\end{itemize}

\subsection{Market-invariant Product-specific Fixed
Effects}\label{market-invariant-product-specific-fixed-effects}

\begin{itemize}
\tightlist
\item
  Furthermore, if you can assume \(\xi_{jt} = \xi_j\), then

  \begin{equation}
  \ln [\sigma_j(p_t, x_t, \xi_t) / \sigma_0(p_t, x_t, \xi_t)] = \beta' x_{jt} - \alpha p_{jt} + \xi_{j}.
  \end{equation}
\item
  This is nothing but a linear regression on \(x_j\) and \(p_{jt}\) with
  product-specific unobserved fixed effect.
\item
  This can be estimated by a within-estimator.
\item
  This specification is a good starting point: we better start with the
  simplest specification and use the estimate as the initial guess for
  the following specifications.
\end{itemize}

\subsection{Unobserved Consumer Heterogeneity and Unobserved Fixed
Effects in Mixed-logit
Model}\label{unobserved-consumer-heterogeneity-and-unobserved-fixed-effects-in-mixed-logit-model}

\begin{itemize}
\tightlist
\item
  So far we abstracted away from the unobserved consumer heterogeneity.
\item
  Next, suppose that the indirect utility function of good \(j\) for
  consumer \(i\) in market \(t\) is:

  \begin{equation}
  \beta_i' x_{jt}  - \alpha_i p_{jt} - \xi_{jt} + \epsilon_{ik},
  \end{equation}

  where \(\epsilon_{ik}\) is i.i.d. Type-I extreme value.
\item
  The coefficient are drawn according to:

  \begin{equation}
  \begin{split}
  &\beta_{it} = \beta_0 + \Sigma \nu_{it},\\
  &\alpha_{it} = \alpha_0 + \Omega \upsilon_{it},
  \end{split}
  \end{equation}
\item
  \(\nu_i\) are i.i.d. standard normal random variables.
\item
  Then the indirect utility of good \(j\) for consumer \(i\) in market
  \(t\) is written as:

  \begin{equation}
  \underbrace{\beta_0' x_{jt} - \alpha_0 p_{jt} + \xi_{jt}}_{\text{(conditional) mean}} + \underbrace{\nu_{it}' \Sigma x_{jt} - \upsilon_{it}' \Omega p_{jt}}_{\text{deviation from the mean}} 
  \end{equation}
\item
  We refer to \(\beta_0, \alpha_0\) as \textbf{linear parameters} and
  \(\Sigma, \Omega\) as \textbf{non-linear parameters}, because of the
  reason I explain in the subsequent section.
\item
  Let \(\theta_1\) be the linear parameters and \(\theta_2\) the
  non-linear parameters and let \(\theta = (\theta_1', \theta_2')'\).
\end{itemize}

\subsection{Unobserved Fixed Effects in Mixed-logit
Model}\label{unobserved-fixed-effects-in-mixed-logit-model}

\begin{itemize}
\tightlist
\item
  The choice share of good \(j\) in market \(t\) is:

  \begin{equation}
  \begin{split}
  &\sigma_{j}(p_t, x_t, \xi_t, y, z; \theta)\\
  &= \int \frac{\exp[\beta_0' x_{jt} - \alpha_0 p_{jt} + \xi_{jt} + \nu_{it}' \Sigma x_{jt} - \upsilon_{it}' \Omega p_{jt}]}{1 + \sum_{k \in \mathcal{J}_t} \exp[\exp[\beta_0' x_{kt} - \alpha_0 p_{kt} + \xi_{kt} + \nu_{it}' \Sigma x_{kt} - \upsilon_{it}' \Omega p_{kt}]]} f(\nu, \upsilon) d \nu d \upsilon.
  \end{split}
  \end{equation}
\item
  How can we represent \(\xi_{jt}\) as a function of parameters of
  interest to exploit the moment condition?
\end{itemize}

\subsection{\texorpdfstring{Representing \(\xi_{jt}\) as a Function of
Parameters of
Interest}{Representing \textbackslash{}xi\_\{jt\} as a Function of Parameters of Interest}}\label{representing-xi_jt-as-a-function-of-parameters-of-interest}

\begin{itemize}
\tightlist
\item
  Let \(s_{jt}\) be the share of product \(j\) in market \(t\).
\item
  The following system of equations implicitly determines \(\xi_{jt}\)
  as a function of parameters of interest:

  \begin{equation}
  s_{jt} = \sigma_j(p_t, x_t, \xi_t; \theta).
  \end{equation}
\item
  Let \(\xi_{jt}(\theta)\) is the solution to the system of equations
  above given parameter \(\theta\).
\item
  If it exists, it is the unobserved heterogeneity as a function of
  parameters and data.
\item
  Does this solution exist?
\item
  Is it unique?
\item
  Is there efficient method to find the solution?
\end{itemize}

\subsection{Summarizing the Conditional Mean
Term}\label{summarizing-the-conditional-mean-term}

\begin{itemize}
\tightlist
\item
  Now, let \(\delta_{jt}\) be the conditional mean term in the indirect
  utility:

  \begin{equation}
  \delta_{jt} \equiv \beta_0' x_{jt} - \alpha_0 p_{jt} + \xi_{jt}.
  \end{equation}
\item
  I call it the average utility of the product in the market.
\item
  Then, the choice share of product \(j\) in market \(t\) is written as:

  \begin{equation}
  \begin{split}
  &\sigma_{jt}(\delta_t, \theta_2) \\
  &\equiv \int \frac{\exp\Bigg(\delta_{jt} + \nu' \Sigma x_{jt} - \upsilon' \Omega p_{jt}\Bigg)}{1 + \sum_{k \in \mathcal{J}_t} \exp\Bigg(\delta_{kt} + \nu' \Sigma x_{kt} - \upsilon' \Omega p_{kt}\Bigg)} f(\nu, \upsilon) d\nu d\upsilon,
  \end{split}
  \end{equation}

  for \(j = 1, \cdots, J, t = 1, \cdots, T\).
\end{itemize}

\subsection{\texorpdfstring{Contraction Mapping for
\(\delta_t\).}{Contraction Mapping for \textbackslash{}delta\_t.}}\label{contraction-mapping-for-delta_t.}

\begin{itemize}
\tightlist
\item
  Now, fix \(\theta_2\) and define an operator \(T\) such that:

  \begin{equation}
  T_t(\delta_t) = \delta_t + \ln \underbrace{s_{jt}}_{\text{data}} - \ln \underbrace{\sigma_{jt}(\delta_t, \theta_2)}_{\text{model}}.
  \end{equation}
\item
  Let
  \(\delta_t^{(0)} = (\delta_{1t}^{(0)}, \cdots, \delta_{Jt}^{(0)})'\)
  be an arbitrary starting vector of average utility of products in a
  market.
\item
  Using the operator above, we update \(\delta_{t}^{(r)}\) by:

  \begin{equation}
  \delta_{t}^{(r + 1)} = T_t(\delta_{t}^{(r)}) = \delta_t^{(r)} + \ln s_{jt} - \ln \sigma_{jt}(\delta_t^{(r)}, \theta_2),
  \end{equation}

  for \(r = 0, 1, \cdots\).
\item
  \citet{Berry1995a} proved that \(T_t\) as specified above is a
  \textbf{contraction mapping with modulus less than one}.
\item
  This means that:

  \begin{enumerate}
  \def\labelenumi{\arabic{enumi}.}
  \tightlist
  \item
    \(T_t\) has a unique fixed point;
  \item
    For arbitrary \(\delta_t^{(r)}\),
    \(\lim_{r \to \infty} T_t^r(\delta_t^{(0)})\) is the unique fixed
    point.
  \end{enumerate}
\item
  The fixed point of \(T_t\) is \(\delta_t^*\) such that
  \(\delta_t^* = T_t(\delta_t^*)\), i.e.,

  \begin{equation}
  \begin{split}
  &\delta_t^* = \delta_t^* + \ln s_{jt} - \ln \sigma_{jt}(\delta_t^*, \theta_2),\\
  &\Leftrightarrow s_{jt} = \sigma_{jt}(\delta_t^*, \theta_2).
  \end{split}
  \end{equation}
\item
  So, the fixed point \(\delta_t^*\) is the conditional mean indirect
  utility that solves the equality given non-linear parameter
  \(\theta_2\).
\item
  Moreover, the solution is unique.
\item
  Moreover, it can be found by iterating the operator.
\item
  Let \(\delta_t(\theta_2)\) be the solution to this equation, i.e., the
  limit of this operation.
\item
  The above result is useful because it ensures the inversion and
  provides the algorithm to find the solution.
\item
  The invertibility itself holds under more general settings
  \citep{Berry2013}.
\end{itemize}

\subsection{\texorpdfstring{Solving for
\(\xi_{jt}(\theta)\)}{Solving for \textbackslash{}xi\_\{jt\}(\textbackslash{}theta)}}\label{solving-for-xi_jttheta}

\begin{itemize}
\tightlist
\item
  We defined the average utility as:

  \begin{equation}
  \delta_{jt} =  \beta_0' x_{jt} - \alpha_0 p_{jt} + \xi_{jt}.
  \end{equation}
\item
  Hence, if we set:

  \begin{equation}
  \xi_{jt}(\theta) \equiv \delta_t(\theta_2) - \Bigg[\sum_{l = 1}^L \beta_{l} w_{jl} - \alpha_0 p_{jt} \Bigg],
  \end{equation}

  the \(\xi_{jt}(\theta)\) solves the equality:

  \begin{equation}
  s_{jt} = \sigma_{j}(p_t, x, \xi_t; \theta).
  \end{equation}
\end{itemize}

\subsection{\texorpdfstring{Solving for \(\xi_{jt}(\theta)\):
Summary}{Solving for \textbackslash{}xi\_\{jt\}(\textbackslash{}theta): Summary}}\label{solving-for-xi_jttheta-summary}

\begin{itemize}
\tightlist
\item
  In summary, \(\xi_{jt}\) that solves the equality exists and unique,
  and can be computed by:
\item
  Fix \(\theta = \{\theta_1, \theta_2\}\).
\item
  Fix arbitrary starting value \(\delta_t^{(0)}\) for
  \(t = 1, \cdots, T\).
\item
  Let \(\delta_t(\theta_2)\) be the limit of \(T_t^r(\delta_t^{(0)})\)
  for \(r = 0, 1, \cdots\) for each \(t = 1, \cdots, T\).
\item
  Stop the iteration if
  \(|\delta_t(\theta_2)^{(r + 1)} - \delta_t(\theta_2)^{(r)}|\) is below
  a threshold.
\item
  Let \(\xi_{jt}(\theta)\) be such that:

  \begin{equation}
  \xi_{jt}(\theta) = \delta_{jt}(\theta_2) - \beta_0' x_{jt} - \alpha_0 p_{jt}.
  \end{equation}
\item
  Then we can evaluate the moment at \(\theta\) by:

  \begin{equation}
  \mathbb{E}\{\xi_{jt}(\theta)|w_{jt}\} = 0.
  \end{equation}
\item
  We run this algorithm every time we evaluate the moment condition at a
  parameter value.
\end{itemize}

\subsection{GMM Objective Function}\label{gmm-objective-function}

\begin{itemize}
\tightlist
\item
  Find \(\theta\) that solves:

  \begin{equation}
  \min_{\theta} \xi(\theta)' W \Phi^{-1} W' \xi(\theta),
  \end{equation}

  where \(\Phi\) is a weight matrix,

  \begin{equation}
  \xi(\theta) = 
  \begin{pmatrix}
  \xi_{11}(\theta)\\
  \vdots\\
  \xi_{J_1 1}(\theta)\\
  \vdots\\
  \xi_{1T} \\
  \vdots\\
  \xi_{J_T T}
  \end{pmatrix},
  W = 
  \begin{pmatrix}
  w_{11}' \\
  \vdots \\
  w_{J_11}' \\
  \vdots \\
  w_{1T}' \\
  \vdots \\
  w_{J_TT}' \\
  \end{pmatrix}.
  \end{equation}
\item
  There are \(J \to \infty\) and \(T \to \infty\) asymptotics. Either is
  fine to consistently estimate the parameters.
\item
  \(w_{jt} = (x_{jt}', w_{jt}^*)'\) where \(w_{jt}^*\) is an excluded
  instrument that is relevant to \(p_{jt}\).
\end{itemize}

\subsection{Estimating Linear
Parameters}\label{estimating-linear-parameters}

\begin{itemize}
\tightlist
\item
  The first-order condition for \(\theta_1\) is:

  \begin{equation}
  \theta_1 = (X_1'W \Phi^{-1} W'X_1)^{-1} X_1' W \Phi^{-1} W' \delta(\theta_2),
  \end{equation}

  where

  \begin{equation}
  X_1 = 
  \begin{pmatrix}
  x_{11}' & - p_{11}\\
  \vdots & \vdots \\
  x_{J_1 1}' & - p_{J_1 1}\\
  \vdots & \vdots \\
  x_{1T}' & - p_{1T}\\
  \vdots & \vdots \\
  x_{J_T T} & - p_{J_T T}
  \end{pmatrix},
  \delta(\theta_2) =
  \begin{pmatrix}
  \delta_1(\theta_2)\\
  \vdots\\
  \delta_T(\theta_2)
  \end{pmatrix}
  \end{equation}

  .
\item
  If \(\theta_2\) is given, the optimal \(\theta_1\) is computed by the
  above formula.
\item
  \(\rightarrow\) We only have to search over \(\theta_2\).
\item
  This is the reason why we called \(\theta_1\) linear parameters and
  \(\theta_2\) non-linear parameters.
\end{itemize}

\subsection{BLP Algorithm}\label{blp-algorithm}

\begin{itemize}
\tightlist
\item
  Find \(\theta_2\) that maximizes the GMM objective function.
\item
  To do so:
\end{itemize}

\begin{enumerate}
\def\labelenumi{\arabic{enumi}.}
\tightlist
\item
  Pick up \(\theta_2\).
\item
  Compute \(\delta(\theta_2)\) by the fixed-point algorithm.
\item
  Compute associated \(\theta_1\) by the formula:

  \begin{equation}
  \theta_1 = (X_1'W \Phi^{-1} W'X_1)^{-1} X_1' W \Phi^{-1} W' \delta(\theta_2),
  \end{equation}
\item
  Compute \(\xi(\theta)\) from the above \(\delta(\theta_2)\) and
  \(\theta_1\).
\item
  Evaluate the GMM objective function with the \(\xi(\theta)\).
\end{enumerate}

\subsection{Mathematical Program with Equilibrium Constraints
(MPEC)}\label{mathematical-program-with-equilibrium-constraints-mpec}

\begin{itemize}
\tightlist
\item
  In the BLP algorithm, for each parameter \(\theta\), find
  \(\xi(\theta)\) that solve:

  \begin{equation}
  s = \sigma(p, x, \xi; \theta)
  \end{equation}

  by the fixed-point algorithm and then evaluate the GMM objective
  function.
\item
  This inner loop takes time if the stopping criterion is tight.
\item
  If the stopping criterion is loose, the loop may stop earlier but the
  error may be unacceptably large.
\item
  \citet{Dube2012} suggest to minimize the GMM objective function with
  the above equation as the constraints.

  \begin{equation}
  \min_{\theta} \xi(\theta)' W \Phi^{-1} W' \xi(\theta) \text{   s.t.   } s = \sigma(p, x, \xi; \theta).
  \end{equation}
\item
  To enjoy the benefit of this approach, we have to analytically derive
  the gradient and hessian of the objective function and the
  constraints, which are anyway needed if we estimate the standard error
  with the plug-in method.
\item
  If the problem is of small scale, BLP algorithm will be fast enough
  and easier to implement.
\item
  If the problem is of large scale, you may better use the MPEC
  approach.
\end{itemize}

\subsection{Instrumental Variables}\label{instrumental-variables}

\begin{itemize}
\tightlist
\item
  The remaining problem is how to choose the excluded instrumental
  variable \(w_{jt}^*\) for each product/market.
\item
  \textbf{Cost shifters}:

  \begin{itemize}
  \tightlist
  \item
    Traditional instruments.
  \end{itemize}
\item
  \textbf{Hausman-type IV} \citep{Hausman1994}:

  \begin{itemize}
  \tightlist
  \item
    Assume that demand shocks are independent across markets, whereas
    the cost shocks are correlated.
  \item
    The latter will be true if the product is produced by the same
    manufacturer.
  \item
    Then, the price of the same product in the other markets
    \(p_{j, -t}\) will be valid instruments for the price of the product
    in a given market, \(p_{jt}\).
  \end{itemize}
\item
  \textbf{BLP-type IV} \citep{Berry1995a}:

  \begin{itemize}
  \tightlist
  \item
    In oligopoly, the price of a good in a market depends on the market
    structure, i.e., what kind of products are available in the market.
  \item
    For example, if there are similar products in the market, the price
    will tend to be lower.
  \item
    Then, the product characteristics of other products in the market ,
    will be valid instrument for the price of goods in a given market,
    \(p_{jt}\).
  \item
    If there are multi-product firms, whether the other good is owned by
    the same company will also affect the price.
  \item
    Specifically, \citet{Berry1995a} use:
  \end{itemize}
\end{itemize}

\begin{equation}
\sum_{k \neq j \in \mathcal{J}_t \cap \mathcal{F}_{f}} x_{kt},
\end{equation}

\begin{equation}
\sum_{k \neq j \in \mathcal{J}_t \setminus \mathcal{F}_{f}} x_{kt}.
\end{equation}

\begin{itemize}
\tightlist
\item
  \(f\) is the firm that owns product \(j\) and \(\mathcal{F}_{f}\) is
  the set of products firm \(f\) owns.
\item
  \textbf{Differentiation IV} \citep{Gandhi2015a}:

  \begin{itemize}
  \tightlist
  \item
    Let \(d_{jkt} = d(x_{jt}, x_{kt})\) be some distance between product
    characteristics.
  \item
    They showed that under certain conditions the optimal BLP-type IV is
    a function \(d_{-jt}\{d_{jkt}\}_{k \neq j \in \mathcal{J}_t}\).
  \item
    The suggest to use the moments of \(d_{-jt}\) as the excluded
    instrument variables.
  \end{itemize}
\item
  Weak instruments problem of BLP-type IV:

  \begin{itemize}
  \tightlist
  \item
    \citet{Armstrong2016b} argued that estimates based on BLP-type IV
    may be inconsistent when \(J \times \infty\) asymptotics is
    considered, because then the market approaches the competitive
    market and the correlation between the markup and the product
    characteristics of the rivals disappear.
  \item
    Specifically, the estimator is inconsistent if all of the following
    conditions are met:
  \end{itemize}

  \begin{enumerate}
  \def\labelenumi{\arabic{enumi}.}
  \tightlist
  \item
    \(J \to \infty\) but \(T\) is fixed;
  \item
    The demand/cost functions are such that the correlation between
    markups and characteristics of other products decreases quickly
    enough as \(J \to \infty\).
  \item
    There is no cost instruments or other sources of identification.
  \end{enumerate}
\end{itemize}

\chapter{Assignment 1: Basic Programming in R}\label{assignment1}

The deadline is the \textbf{start time of February 18 class}.

Report the following results in html format using R markdown. In other
words, replicate this document. You write functions in a separate R file
and put in \texttt{R} folder in the project folder. Build the project as
a package and load it from the R markdown file. The execution code
sholuld be written in R markdown file.

You submit:

\begin{itemize}
\tightlist
\item
  R file containing functions.
\item
  R markdown file containing your answers and executing codes.
\item
  HTML report generated from the R markdown.
\end{itemize}

\section{Simulate data}\label{simulate-data}

Consider to simulate data from the following model and estimate the
parameters from the simulated data.

\[
y_{ij} = 1\{j = \text{argmax}_{k = 1, 2} \beta x_k + \epsilon_{ik} \},
\] where \(\epsilon_{ik}\) follows i.i.d. type-I extreme value
distribution, \(\beta = 0.2\), \(x_1 = 0\) and \(x_2 = 1\).

\begin{enumerate}
\def\labelenumi{\arabic{enumi}.}
\tightlist
\item
  To simulate data, first make a data frame as follows:
\end{enumerate}

\begin{verbatim}
## # A tibble: 2,000 x 3
##        i     k     x
##    <int> <int> <dbl>
##  1     1     1     0
##  2     1     2     1
##  3     2     1     0
##  4     2     2     1
##  5     3     1     0
##  6     3     2     1
##  7     4     1     0
##  8     4     2     1
##  9     5     1     0
## 10     5     2     1
## # ... with 1,990 more rows
\end{verbatim}

\begin{enumerate}
\def\labelenumi{\arabic{enumi}.}
\setcounter{enumi}{1}
\tightlist
\item
  Second, draw type-I extreme value random variables. Set the seed at 1.
  You can use \texttt{evd} package to draw the variables. You should get
  exactly the same realization if the seed is correctly set.
\end{enumerate}

\begin{verbatim}
## # A tibble: 2,000 x 4
##        i     k     x       e
##    <int> <int> <dbl>   <dbl>
##  1     1     1     0  0.281 
##  2     1     2     1 -0.167 
##  3     2     1     0  1.93  
##  4     2     2     1  1.97  
##  5     3     1     0  0.830 
##  6     3     2     1 -1.06  
##  7     4     1     0 -0.207 
##  8     4     2     1  0.617 
##  9     5     1     0  0.0444
## 10     5     2     1  1.92  
## # ... with 1,990 more rows
\end{verbatim}

\begin{enumerate}
\def\labelenumi{\arabic{enumi}.}
\setcounter{enumi}{2}
\tightlist
\item
  Third, compute the latent value of each option to obtain the following
  data frame:
\end{enumerate}

\begin{verbatim}
## # A tibble: 2,000 x 5
##        i     k     x       e  latent
##    <int> <int> <dbl>   <dbl>   <dbl>
##  1     1     1     0  0.281   0.281 
##  2     1     2     1 -0.167   0.0331
##  3     2     1     0  1.93    1.93  
##  4     2     2     1  1.97    2.17  
##  5     3     1     0  0.830   0.830 
##  6     3     2     1 -1.06   -0.863 
##  7     4     1     0 -0.207  -0.207 
##  8     4     2     1  0.617   0.817 
##  9     5     1     0  0.0444  0.0444
## 10     5     2     1  1.92    2.12  
## # ... with 1,990 more rows
\end{verbatim}

\begin{enumerate}
\def\labelenumi{\arabic{enumi}.}
\setcounter{enumi}{3}
\tightlist
\item
  Finally, compute \(y\) by comparing the latent values of \(k = 1, 2\)
  for each \(i\) to obtain the following result:
\end{enumerate}

\begin{verbatim}
## # A tibble: 2,000 x 6
##        i     k     x       e  latent     y
##    <int> <int> <dbl>   <dbl>   <dbl> <dbl>
##  1     1     1     0  0.281   0.281      1
##  2     1     2     1 -0.167   0.0331     0
##  3     2     1     0  1.93    1.93       0
##  4     2     2     1  1.97    2.17       1
##  5     3     1     0  0.830   0.830      1
##  6     3     2     1 -1.06   -0.863      0
##  7     4     1     0 -0.207  -0.207      0
##  8     4     2     1  0.617   0.817      1
##  9     5     1     0  0.0444  0.0444     0
## 10     5     2     1  1.92    2.12       1
## # ... with 1,990 more rows
\end{verbatim}

\section{Estimate the parameter}\label{estimate-the-parameter}

Now you generated simulated data. Suppose you observe \(x_k\) and
\(y_{ik}\) for each \(i\) and \(k\) and estimate \(\beta\) by a maximum
likelihood estimator. The likelihood for \(i\) to choose \(k\)
(\(y_{ik} = 1\)) can be shown to be: \[
p_{ik}(\beta) = \frac{\exp(\beta x_k)}{\exp(\beta x_1) + \exp(\beta x_2)}.
\]

Then, the likelihood of observing \(\{y_{ik}\}_{i, k}\) is: \[
L(\beta) = \prod_{i = 1}^{1000} p_{i1}(\beta)^{y_{i1}} [1 - p_{i1}(\beta)]^{1 - y_{i1}},
\] and the log likelihood is: \[
l(\beta) = \sum_{i = 1}^{1000}\{y_{i1}\log p_{i1}(\beta) + (1 - y_{i1})\log [1 - p_{i1}(\beta)\}.
\]

\begin{enumerate}
\def\labelenumi{\arabic{enumi}.}
\item
  Write a function to compute the livelihood for a given \(\beta\) and
  data and name the function \texttt{loglikelihood\_A1}.
\item
  Compute the value of log likelihood for \(\beta = 0, 0.1, \cdots, 1\)
  and plot the result using \texttt{ggplot2} packages. You can use
  \texttt{latex2exp} package to use LaTeX math symbol in the label:
\end{enumerate}

\begin{center}\includegraphics{lecture_files/figure-latex/unnamed-chunk-31-1} \end{center}

\begin{enumerate}
\def\labelenumi{\arabic{enumi}.}
\tightlist
\item
  Find and report \(\beta\) that maximizes the log likelihood for the
  simulated data. You can use \texttt{optim} function to achieve this.
  You will use \texttt{Brent} method and set the lower bound at -1 and
  upper bound at 1 for the parameter search.
\end{enumerate}

\begin{verbatim}
## $par
## [1] 0.2371046
## 
## $value
## [1] -0.6861689
## 
## $counts
## function gradient 
##       NA       NA 
## 
## $convergence
## [1] 0
## 
## $message
## NULL
\end{verbatim}

\chapter{Assignment 2: Production Function
Estimation}\label{assignment2}

The deadline is the \textbf{February 28 1:30pm}.

\section{Simulate data}\label{simulate-data-1}

Consider the following production and investment process for
\(j = 1, \cdots, 1000\) firms across \(t = 1, \cdots, 10\) periods.

The log production function is of the form: \[
y_{jt} = \beta_0 + \beta_l l_{jt} + \beta_k k_{jt} + \omega_{jt} + \eta_{jt},
\] where \(\omega_{jt}\) is an anticipated shock and \(\eta_{jt}\) is an
ex post shock.

The anticipated shocks evolve as: \[
\omega_{jt} = \alpha \omega_{j, t - 1} + \nu_{jt},
\] where \(\nu_{jt}\) is an i.i.d. normal random variable with mean 0
and standard deviation \(\sigma_\nu\). The ex post shock is an i.i.d.
normal random variable with mean 0 and standard deviation
\(\sigma_{\eta}\).

The product price the same across firms and normalized at 1. The price
is normalized at 1. The wage \(w_t\) is constant at 0.5.

Finally, the capital accumulate according to: \[
K_{j, t + 1} = (1 - \delta) K_{jt} + I_{jt}.
\]

We set the parameters as follows:

\begin{longtable}[]{@{}lll@{}}
\toprule
parameter & variable & value\tabularnewline
\midrule
\endhead
\(\beta_0\) & \texttt{beta\_0} & 1\tabularnewline
\(\beta_l\) & \texttt{beta\_l} & 0.2\tabularnewline
\(\beta_k\) & \texttt{beta\_k} & 0.7\tabularnewline
\(\alpha\) & \texttt{alpha} & 0.7\tabularnewline
\(\sigma_{\eta}\) & \texttt{sigma\_eta} & 0.2\tabularnewline
\(\sigma_{\nu}\) & \texttt{sigma\_nu} & 0.5\tabularnewline
\(\sigma_{w}\) & \texttt{sigma\_w} & 0.1\tabularnewline
\(\delta\) & \texttt{delta} & 0.05\tabularnewline
\bottomrule
\end{longtable}

\begin{enumerate}
\def\labelenumi{\arabic{enumi}.}
\item
  Define the parameter variables as above.
\item
  Write a function that returns the log output given \(l_{jt}\),
  \(k_{jt}\), \(\omega_{jt}\), and \(\eta_{jt}\) under the given
  parameter values according to the above production function and name
  it
  \texttt{log\_production(l,\ k,\ omega,\ eta,\ beta\_0,\ beta\_l,\ beta\_k)}.
\end{enumerate}

Suppose that the labor is determined after \(\omega_{jt}\) is observed,
but before \(\eta_{jt}\) is observed, given the log capital level
\(k_{jt}\).

\begin{enumerate}
\def\labelenumi{\arabic{enumi}.}
\setcounter{enumi}{2}
\tightlist
\item
  Derive the optimal log labor as a function of \(\omega_{jt}\),
  \(\eta_{jt}\), \(k_{jt}\), and wage. Write a function to return the
  optimal log labor given the variables and parameters and name it
  \texttt{log\_labor\_choice(k,\ wage,\ omega,\ beta\_0,\ beta\_l,\ beta\_k,\ sigma\_eta)}.
\end{enumerate}

As discussed in the class, if there is no additional variation in labor,
the coefficient on the labor \(\beta_l\) is not identified. Thus, if we
generate labor choice from the previous function, \(\beta_l\) will not
be identified from the simulated data. To see this, we write a modified
version of the previous function in which \(\omega_{jt}\) is replaced
with \(\omega_{jt} + \iota_{jt}\), where \(\iota_{jt}\) is an
optimization error that follows an i.i.d. normal distribution with mean
0 and standard deviation 0.05. That is, the manager of the firm
perceives as if the shock is \(\omega_{jt} + \iota_{jt}\), even though
the true shock is \(\omega_{jt}\).

\begin{enumerate}
\def\labelenumi{\arabic{enumi}.}
\setcounter{enumi}{3}
\tightlist
\item
  Modify the previous function by including \(\iota_{jt}\) as an
  additional input and name it
  \texttt{log\_labor\_choice\_error(k,\ wage,\ omega,\ beta\_0,\ beta\_l,\ beta\_k,\ iota,\ sigma\_eta)}.
\end{enumerate}

Consider an investment process such that: \[
I_{jt} = (\delta + \gamma \omega_{jt}) K_{jt},
\] where \(I_{jt}\) and \(K_{jt}\) are investment and capital in level.
Set \(\gamma = 0.1\), i.e., the investment is strictly increasing in
\(\omega_{jt}\). The investment function should be derived by solving
the dynamic problem of a firm. But here, we just specify it in a
reduced-form.

\begin{enumerate}
\def\labelenumi{\arabic{enumi}.}
\setcounter{enumi}{4}
\tightlist
\item
  Define variable \(\gamma\) and assign it the value. Write a function
  that returns the investment given \(K_{jt}\), \(\omega_{jt}\), and
  parameter values, according to the previous equation, and name it
  \texttt{investment\_choice(k,\ omega,\ gamma,\ delta)}.
\end{enumerate}

Simulate the data first using the labor choice without optimization
error and second using the labor choice with optimization error. To do
so, we specify the initial values for the state variables \(k_{jt}\) and
\(\omega_{jt}\) as follows.

\begin{enumerate}
\def\labelenumi{\arabic{enumi}.}
\setcounter{enumi}{5}
\tightlist
\item
  Draw \(k_{j1}\) from an i.i.d. normal distribution with mean 1 and
  standard deviation 0.5. Draw \(\omega_{j1}\) from its stationary
  distribution (check the stationary distribution of AR(1) process).
  Draw a wage. Before simulating the rest of the data, set the seed at
  1.
\end{enumerate}

\begin{verbatim}
## # A tibble: 1,000 x 5
##        j     t     k   omega  wage
##    <int> <dbl> <dbl>   <dbl> <dbl>
##  1     1     1 0.687  0.795    0.5
##  2     2     1 1.09   0.779    0.5
##  3     3     1 0.582 -0.610    0.5
##  4     4     1 1.80   0.148    0.5
##  5     5     1 1.16   0.0486   0.5
##  6     6     1 0.590 -1.16     0.5
##  7     7     1 1.24   0.568    0.5
##  8     8     1 1.37  -1.34     0.5
##  9     9     1 1.29  -0.873    0.5
## 10    10     1 0.847  0.699    0.5
## # ... with 990 more rows
\end{verbatim}

\begin{enumerate}
\def\labelenumi{\arabic{enumi}.}
\setcounter{enumi}{6}
\tightlist
\item
  Draw optimization error \(\iota_{jt}\) and compute the labor and
  investment choice of period 1. For labor choice, compute both types of
  labor choices.
\end{enumerate}

\begin{verbatim}
## # A tibble: 1,000 x 9
##        j     t     k   omega  wage     iota      l l_error       I
##    <int> <dbl> <dbl>   <dbl> <dbl>    <dbl>  <dbl>   <dbl>   <dbl>
##  1     1     1 0.687  0.795    0.5 -0.0443   1.72   1.67    0.257 
##  2     2     1 1.09   0.779    0.5 -0.0961   2.06   1.94    0.381 
##  3     3     1 0.582 -0.610    0.5  0.0810  -0.123 -0.0218 -0.0196
##  4     4     1 1.80   0.148    0.5  0.0260   1.89   1.92    0.391 
##  5     5     1 1.16   0.0486   0.5 -0.00279  1.21   1.21    0.176 
##  6     6     1 0.590 -1.16     0.5  0.0348  -0.809 -0.766  -0.120 
##  7     7     1 1.24   0.568    0.5  0.00268  1.93   1.93    0.370 
##  8     8     1 1.37  -1.34     0.5 -0.0655  -0.346 -0.428  -0.330 
##  9     9     1 1.29  -0.873    0.5 -0.106    0.165  0.0327 -0.135 
## 10    10     1 0.847  0.699    0.5 -0.0104   1.74   1.73    0.280 
## # ... with 990 more rows
\end{verbatim}

\begin{enumerate}
\def\labelenumi{\arabic{enumi}.}
\setcounter{enumi}{7}
\tightlist
\item
  Draw ex post shock and compute the output according to the production
  function for both labor without optimization error and with
  optimization error. Name the output without optimization error
  \texttt{y} and the one with optimization error \texttt{y\_error}.
\end{enumerate}

\begin{verbatim}
## # A tibble: 1,000 x 12
##        j     t     k   omega  wage     iota      l l_error       I     eta
##    <int> <dbl> <dbl>   <dbl> <dbl>    <dbl>  <dbl>   <dbl>   <dbl>   <dbl>
##  1     1     1 0.687  0.795    0.5 -0.0443   1.72   1.67    0.257   0.148 
##  2     2     1 1.09   0.779    0.5 -0.0961   2.06   1.94    0.381   0.0773
##  3     3     1 0.582 -0.610    0.5  0.0810  -0.123 -0.0218 -0.0196  0.259 
##  4     4     1 1.80   0.148    0.5  0.0260   1.89   1.92    0.391  -0.161 
##  5     5     1 1.16   0.0486   0.5 -0.00279  1.21   1.21    0.176  -0.321 
##  6     6     1 0.590 -1.16     0.5  0.0348  -0.809 -0.766  -0.120   0.187 
##  7     7     1 1.24   0.568    0.5  0.00268  1.93   1.93    0.370   0.361 
##  8     8     1 1.37  -1.34     0.5 -0.0655  -0.346 -0.428  -0.330  -0.0113
##  9     9     1 1.29  -0.873    0.5 -0.106    0.165  0.0327 -0.135   0.377 
## 10    10     1 0.847  0.699    0.5 -0.0104   1.74   1.73    0.280   0.316 
## # ... with 990 more rows, and 2 more variables: y <dbl>, y_error <dbl>
\end{verbatim}

\begin{enumerate}
\def\labelenumi{\arabic{enumi}.}
\setcounter{enumi}{8}
\tightlist
\item
  Repeat this procedure for \(t = 1, \cdots 10\) by updating the capital
  and anticipated shocks, and name the resulting data frame
  \texttt{df\_T}.
\end{enumerate}

\begin{verbatim}
## # A tibble: 10,000 x 13
##        j     t     k   omega  wage     iota      l l_error       I     eta
##    <int> <dbl> <dbl>   <dbl> <dbl>    <dbl>  <dbl>   <dbl>   <dbl>   <dbl>
##  1     1     1 0.687  0.795    0.5 -0.0443   1.72   1.67    0.257   0.148 
##  2     2     1 1.09   0.779    0.5 -0.0961   2.06   1.94    0.381   0.0773
##  3     3     1 0.582 -0.610    0.5  0.0810  -0.123 -0.0218 -0.0196  0.259 
##  4     4     1 1.80   0.148    0.5  0.0260   1.89   1.92    0.391  -0.161 
##  5     5     1 1.16   0.0486   0.5 -0.00279  1.21   1.21    0.176  -0.321 
##  6     6     1 0.590 -1.16     0.5  0.0348  -0.809 -0.766  -0.120   0.187 
##  7     7     1 1.24   0.568    0.5  0.00268  1.93   1.93    0.370   0.361 
##  8     8     1 1.37  -1.34     0.5 -0.0655  -0.346 -0.428  -0.330  -0.0113
##  9     9     1 1.29  -0.873    0.5 -0.106    0.165  0.0327 -0.135   0.377 
## 10    10     1 0.847  0.699    0.5 -0.0104   1.74   1.73    0.280   0.316 
## # ... with 9,990 more rows, and 3 more variables: y <dbl>, y_error <dbl>,
## #   nu <dbl>
\end{verbatim}

\begin{enumerate}
\def\labelenumi{\arabic{enumi}.}
\setcounter{enumi}{9}
\tightlist
\item
  Check the simulated data by making summary table.
\end{enumerate}

\begin{tabular}{l|r|r|r|r|r}
\hline
  & N & Mean & Sd & Min & Max\\
\hline
j & 10000 & 500.5000000 & 288.6894251 & 1.0000000 & 1000.0000000\\
\hline
t & 10000 & 5.5000000 & 2.8724249 & 1.0000000 & 10.0000000\\
\hline
k & 10000 & 0.9797900 & 0.5838949 & -1.2822534 & 3.2332312\\
\hline
omega & 10000 & -0.0055826 & 0.7025102 & -2.5894171 & 2.6281307\\
\hline
wage & 10000 & 0.5000000 & 0.0000000 & 0.5000000 & 0.5000000\\
\hline
iota & 10000 & -0.0000696 & 0.0502883 & -0.1841453 & 0.1715419\\
\hline
l & 10000 & 0.9799746 & 1.0965108 & -3.3281023 & 4.9679634\\
\hline
l\_error & 10000 & 0.9798876 & 1.0971595 & -3.3765433 & 4.9520674\\
\hline
I & 10000 & 0.1793502 & 0.3006526 & -1.2722627 & 3.2975332\\
\hline
eta & 10000 & 0.0015825 & 0.2001539 & -0.7650371 & 0.7455922\\
\hline
y & 10000 & 1.8778479 & 1.1171035 & -2.4680251 & 6.1228291\\
\hline
y\_error & 10000 & 1.8778305 & 1.1169266 & -2.4777133 & 6.1196499\\
\hline
nu & 10000 & -0.0021155 & 0.4984324 & -2.1513907 & 1.8253882\\
\hline
\end{tabular}

\section{Estimate the parameters}\label{estimate-the-parameters}

For now, use the labor choice with optimization error.

\begin{enumerate}
\def\labelenumi{\arabic{enumi}.}
\tightlist
\item
  First, simply regress \(y_{jt}\) on \(l_{jt}\) and \(k_{jt}\) using
  the least square method. This is likely to give an upwardly biased
  estimates on \(\beta_l\) and \(\beta_k\). Why is it?
\end{enumerate}

\begin{verbatim}
## 
## Call:
## lm(formula = y_error ~ l_error + k, data = df_T)
## 
## Residuals:
##      Min       1Q   Median       3Q      Max 
## -0.73002 -0.14117 -0.00071  0.13743  0.87983 
## 
## Coefficients:
##             Estimate Std. Error t value Pr(>|t|)    
## (Intercept) 0.892542   0.004058 219.966   <2e-16 ***
## l_error     0.997913   0.002396 416.454   <2e-16 ***
## k           0.007599   0.004503   1.688   0.0915 .  
## ---
## Signif. codes:  0 '***' 0.001 '**' 0.01 '*' 0.05 '.' 0.1 ' ' 1
## 
## Residual standard error: 0.2068 on 9997 degrees of freedom
## Multiple R-squared:  0.9657, Adjusted R-squared:  0.9657 
## F-statistic: 1.408e+05 on 2 and 9997 DF,  p-value: < 2.2e-16
\end{verbatim}

\begin{enumerate}
\def\labelenumi{\arabic{enumi}.}
\setcounter{enumi}{1}
\tightlist
\item
  Second, take within-transformation on \(y_{jt}\), \(l_{jt}\), and
  \(k_{jt}\) and let \(\Delta y_{jt}\), \(\Delta l_{jt}\), and
  \(\Delta k_{jt}\) denote them. Then, regress \(\Delta y_{jt}\) on
  \(\Delta l_{jt}\), and \(\Delta k_{jt}\) by the least squares method.
\end{enumerate}

\begin{verbatim}
## 
## Call:
## lm(formula = dy_error ~ -1 + dl_error + dk, data = df_T_within)
## 
## Residuals:
##      Min       1Q   Median       3Q      Max 
## -0.72450 -0.13285 -0.00244  0.12931  0.77657 
## 
## Coefficients:
##            Estimate Std. Error t value Pr(>|t|)    
## dl_error  0.9910916  0.0029548 335.413   <2e-16 ***
## dk       -0.0009029  0.0127539  -0.071    0.944    
## ---
## Signif. codes:  0 '***' 0.001 '**' 0.01 '*' 0.05 '.' 0.1 ' ' 1
## 
## Residual standard error: 0.1961 on 9998 degrees of freedom
## Multiple R-squared:  0.9184, Adjusted R-squared:  0.9184 
## F-statistic: 5.629e+04 on 2 and 9998 DF,  p-value: < 2.2e-16
\end{verbatim}

Next, we attempt to estimate the parameters using Olley-Pakes method.
Estimate the first-step model of Olley-Pakes method: \[
y_{jt} = \beta_0 + \beta_1 l_{jt} + \phi(k_{jt}, I_{jt}) + \eta_{jt},
\] by approximating \(\phi_t\) by a kernel function.

Remark that \(\phi\) in general depends on observed and unobserved state
variables. For this reason, in theory, \(\phi\) should be estimated for
each period. In this exercise, we assume \(\phi\) is common across
periods because we know that there is no unobserved state variables in
the true data generating process. Moreover, we do not include \(w_t\)
because we know that it is time -invariant. Do not forget to consider
them in the actual data analysis.

You can use \texttt{npplreg} function of \texttt{np} package to estimate
a partially linear model with a multivariate kernel. You first use
\texttt{npplregbw} to obtain the optimal band width and then use
\texttt{npplreg} to estimate the model under the optimal bandwidth. The
computation of the optimal bandwidth is time consuming.

\begin{enumerate}
\def\labelenumi{\arabic{enumi}.}
\setcounter{enumi}{2}
\tightlist
\item
  Return the summary of the first stage estimation and plot the fitted
  values against the data points.
\end{enumerate}

\begin{verbatim}
## 
## Partially Linear Model
## Regression data: 10000 training points, in 5 variable(s)
## With 3 linear parametric regressor(s), 2 nonparametric regressor(s)
## 
##                     y(z)           
## Bandwidth(s): 0.07355058 0.01435558
## 
##                     x(z)           
## Bandwidth(s): 0.03908594 0.01191551
##               0.01397428 3.74952954
##               0.83529394 0.00398329
## 
##                   l_error        k        I
## Coefficient(s): 0.2485295 2.355522 5.345144
## 
## Kernel Regression Estimator: Local-Constant
## Bandwidth Type: Fixed
## 
## Residual standard error: 0.1934585
## R-squared: 0.970064
## 
## Continuous Kernel Type: Second-Order Gaussian
## No. Continuous Explanatory Vars.: 2
\end{verbatim}

\begin{center}\includegraphics{lecture_files/figure-latex/unnamed-chunk-47-1} \end{center}

\begin{enumerate}
\def\labelenumi{\arabic{enumi}.}
\setcounter{enumi}{3}
\tightlist
\item
  Check that \(\beta_l\) is not identified with the data without
  optimization error. Estimate the first stage model of Olley-Pakes with
  the labor choice without optimization error and report the result.
\end{enumerate}

\begin{verbatim}
## 
## Partially Linear Model
## Regression data: 10000 training points, in 5 variable(s)
## With 3 linear parametric regressor(s), 2 nonparametric regressor(s)
## 
##                     y(z)           
## Bandwidth(s): 0.07347226 0.01437256
## 
##                     x(z)            
## Bandwidth(s): 0.02960021 0.009986945
##               0.01397428 3.749529542
##               0.83529394 0.003983290
## 
##                        l        k         I
## Coefficient(s): 1.180628 2.034182 0.7805077
## 
## Kernel Regression Estimator: Local-Constant
## Bandwidth Type: Fixed
## 
## Residual standard error: 0.1932285
## R-squared: 0.970116
## 
## Continuous Kernel Type: Second-Order Gaussian
## No. Continuous Explanatory Vars.: 2
\end{verbatim}

Then, we estimate the second stage model of Olley-Pakes method: \[
y_{jt} - \hat{\beta_l} l_{jt} = \beta_0 + \beta_k k_{jt} + \alpha[\hat{\phi}(k_{j, t - 1}, I_{j, t - 1}) - \beta_0 - \beta_k k_{jt}] + \nu_{jt} + \eta_{jt}.
\]

In this model, we do not have to non-parametetrically estimate the
conditional expectation of \(\omega_{jt}\) on \(\omega_{j, t - 1}\),
because we know that the anticipated shock follows an AR(1) process.
Remark that we in general have to non-parametrically estimate it.

The model is non-linear in parameters, because of the term
\(\alpha \beta_0\) and \(\alpha \beta_k\). We estimate \(\alpha\),
\(\beta_0\), and \(\beta_k\) by a GMM estimator. The moment is: \[
g_{JT}(\alpha, \beta_0, \beta_k) \equiv \frac{1}{JT}\sum_{j = 1}^J \sum_{t = 1}^T \{y_{jt} - \hat{\beta_l} l_{jt} - \beta_0 - \beta_k k_{jt} - \alpha[\hat{\phi}(k_{j, t - 1}, I_{j, t - 1}) - \beta_0 - \beta_k k_{jt}]\} 
\begin{bmatrix}
k_{jt} \\
k_{j, t - 1} \\
I_{j, t - 1}
\end{bmatrix}.
\]

\begin{enumerate}
\def\labelenumi{\arabic{enumi}.}
\setcounter{enumi}{4}
\tightlist
\item
  Using the estimates in the first step, compute: \[
  y_{jt} - \hat{\beta_l} l_{jt},
  \] and: \[
  \hat{\phi}(k_{j, t - 1}, I_{j, t - 1}),
  \] for each \(j\) and \(t\) and save it as a data frame names
  \texttt{df\_T\_1st}.
\end{enumerate}

\begin{verbatim}
## # A tibble: 10,000 x 4
##        j     t y_error_tilde phi_t_1
##    <int> <dbl>         <dbl>   <dbl>
##  1     1     1         2.34   NA    
##  2     1     2         1.37    2.21 
##  3     1     3         0.621   1.49 
##  4     1     4         0.447   0.882
##  5     1     5         0.878   0.611
##  6     1     6         1.62    0.926
##  7     1     7         0.558   1.40 
##  8     1     8         0.684   0.439
##  9     1     9         0.939   0.520
## 10     1    10         1.49    0.836
## # ... with 9,990 more rows
\end{verbatim}

\begin{enumerate}
\def\labelenumi{\arabic{enumi}.}
\setcounter{enumi}{5}
\tightlist
\item
  Compute a function that returns the value of
  \(g_{JT}(\alpha, \beta_0, \beta_k)\) given parameter values, data, and
  \texttt{df\_T\_1st}, and name it \texttt{moment\_OP\_2nd}. Show the
  values of the moments evaluated at the true parameters.
\end{enumerate}

\begin{verbatim}
## [1] -0.018507303 -0.019038229 -0.003867714
\end{verbatim}

Based on the moment, we can define the objective function of a
generalized method of moments estimator with a weighting matrix \(W\)
as: \[
Q_{JT}(\alpha, \beta_0, \beta_k) \equiv g_{JT}(\alpha, \beta_0, \beta_k)' W g_{JT}(\alpha, \beta_0, \beta_k).
\]

\begin{enumerate}
\def\labelenumi{\arabic{enumi}.}
\setcounter{enumi}{6}
\tightlist
\item
  Write a function that returns the value of
  \(Q_{JT}(\alpha, \beta_0, \beta_k)\) given the vector of parameter
  values, data, and \texttt{df\_T\_1st}, and name it
  \texttt{objective\_OP\_2nd}. Setting \(W\) at the identity matrix,
  show the value of the objective function evaluated at the true
  parameters.
\end{enumerate}

\begin{verbatim}
##              [,1]
## [1,] 0.0007199336
\end{verbatim}

\begin{enumerate}
\def\labelenumi{\arabic{enumi}.}
\setcounter{enumi}{7}
\tightlist
\item
  Draw the graph of the objective function when one of \(\alpha\),
  \(\beta_0\), and \(\beta_k\) are changed from 0 to 1 by 0.1 while the
  others are set at the true value. Is the objective function minimized
  at around the true value?
\end{enumerate}

\begin{center}\includegraphics{lecture_files/figure-latex/unnamed-chunk-54-1} \end{center}

\begin{center}\includegraphics{lecture_files/figure-latex/unnamed-chunk-54-2} \end{center}

\begin{center}\includegraphics{lecture_files/figure-latex/unnamed-chunk-54-3} \end{center}

\begin{enumerate}
\def\labelenumi{\arabic{enumi}.}
\setcounter{enumi}{8}
\tightlist
\item
  Find the parameters that minimize the objective function using
  \texttt{optim}. You may use \texttt{L-BFGS-B} method to solve it.
\end{enumerate}

\begin{verbatim}
## $par
## [1] 0.7020260 0.9766308 0.6693945
## 
## $value
## [1] 1.994601e-07
## 
## $counts
## function gradient 
##       10       10 
## 
## $convergence
## [1] 0
## 
## $message
## [1] "CONVERGENCE: REL_REDUCTION_OF_F <= FACTR*EPSMCH"
\end{verbatim}

\bibliography{library.bib,packages.bib}


\end{document}
